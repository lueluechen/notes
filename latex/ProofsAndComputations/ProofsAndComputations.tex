% Created 2021-05-21 五 05:39
% Intended LaTeX compiler: pdflatex
\documentclass[11pt]{article}
\usepackage[utf8]{inputenc}
\usepackage[T1]{fontenc}
\usepackage{graphicx}
\usepackage{grffile}
\usepackage{longtable}
\usepackage{wrapfig}
\usepackage{rotating}
\usepackage{amsmath}
\usepackage{textcomp}
\usepackage{amssymb}
\usepackage{capt-of}
\usepackage{hyperref}
%%%%%%%%%%%%%%%%%%%%%%%%%%%%%%%%%%%%%%
%% TIPS                                 %%
%%%%%%%%%%%%%%%%%%%%%%%%%%%%%%%%%%%%%%
% \substack{a\\b} for multiple lines text

\usepackage[utf8]{inputenc}

\usepackage[B1,T1]{fontenc}

% pdfplots will load xolor automatically without option
\usepackage[dvipsnames]{xcolor}
%%%%%%%%%%%%%%%%%%%%%%%%%%%%%%%%%%%%%%%
%% MATH related pacakge                  %%
%%%%%%%%%%%%%%%%%%%%%%%%%%%%%%%%%%%%%%%
% \usepackage{amsmath} mathtools loads the amsmath
\usepackage{amsmath}
\usepackage{mathtools}


\usepackage{amsthm}
\usepackage{amsbsy}

%\usepackage{commath}

\usepackage{amssymb}
\usepackage{mathrsfs}
%\usepackage{mathabx}
\usepackage{stmaryrd}
\usepackage{empheq}

%for \not\ll
\usepackage{centernot}

\usepackage{scalerel}
\usepackage{stackengine}
\usepackage{stackrel}

\usepackage{nicematrix}
\usepackage{tensor}
\usepackage{blkarray}
\usepackage{siunitx}
\usepackage[f]{esvect}

\usepackage{unicode-math}
\setmainfont{TeX Gyre Pagella}
% \setmathfont{STIX}
%\setmathfont{texgyrepagella-math.otf}
%\setmathfont{Libertinus Math}
\setmathfont{Latin Modern Math}

 
% \setmathfont[range={\smwhtdiamond,\enclosediamond,\varlrtriangle}]{Latin Modern Math}
 \setmathfont[range={\rightrightarrows,\twoheadrightarrow,\leftrightsquigarrow,\triangledown,\vartriangle}]{XITS Math}
 \setmathfont[range={\int,\setminus}]{Libertinus Math}
 \setmathfont[range={\mathalpha}]{TeX Gyre Pagella Math}
% unicode is not good at this!
%\let\nmodels\nvDash


%%%%%%%%%%%%%%%%%%%%%%%%%%%%%%%%%%%%%%%
%% TIKZ related packages                 %%
%%%%%%%%%%%%%%%%%%%%%%%%%%%%%%%%%%%%%%%

\usepackage{pgfplots}
\pgfplotsset{compat=1.15}
\usepackage{tikz}
\usepackage{tikz-cd}
\usepackage{tikz-qtree}

\usetikzlibrary{arrows,positioning,calc,fadings,decorations,matrix,decorations,shapes.misc}
%setting from geogebra
\definecolor{ccqqqq}{rgb}{0.8,0,0}


%%%%%%%%%%%%%%%%%%%%%%%%%%%%%%%%%%%%%%%
%% MISCLELLANEOUS packages               %%
%%%%%%%%%%%%%%%%%%%%%%%%%%%%%%%%%%%%%%%
\usepackage[most]{tcolorbox}
\usepackage{threeparttable}
\usepackage{tabularx}

\usepackage{enumitem}

% wrong with preview
\usepackage{subcaption}
\usepackage{caption}
% {\aunclfamily\Huge}
\usepackage{auncial}

\usepackage{float}

\usepackage{fancyhdr}

\usepackage{ifthen}
\usepackage{xargs}


\usepackage{imakeidx}
\usepackage{hyperref}
\usepackage{soul}


%\usepackage[xetex]{preview}
%%%%%%%%%%%%%%%%%%%%%%%%%%%%%%%%%%%%%%%
%% USEPACKAGES end                       %%
%%%%%%%%%%%%%%%%%%%%%%%%%%%%%%%%%%%%%%%

% \setlist{nosep}
% \numberwithin{equation}{subsection}
% \fancyhead{} % Clear the headers
% \renewcommand{\headrulewidth}{0pt} % Width of line at top of page
% \fancyhead[R]{\slshape\leftmark} % Mark right [R] of page with Chapter name [\leftmark]
% \pagestyle{fancy} % Set default style for all content pages (not TOC, etc)


% \newlength\shlength
% \newcommand\vect[2][0]{\setlength\shlength{#1pt}%
%   \stackengine{-5.6pt}{$#2$}{\smash{$\kern\shlength%
%     \stackengine{7.55pt}{$\mathchar"017E$}%
%       {\rule{\widthof{$#2$}}{.57pt}\kern.4pt}{O}{r}{F}{F}{L}\kern-\shlength$}}%
%       {O}{c}{F}{T}{S}}


\indexsetup{othercode=\small}
\makeindex[columns=2,options={-s /media/wu/file/stuuudy/notes/index_style.ist},intoc]
\makeatletter
\def\@idxitem{\par\hangindent 0pt}
\makeatother


%\newcounter{dummy} \numberwithin{dummy}{section}
\newtheorem{dummy}{dummy}[section]
\theoremstyle{definition}
\newtheorem{definition}[dummy]{Definition}
\theoremstyle{plain}
\newtheorem{corollary}[dummy]{Corollary}
\newtheorem{lemma}[dummy]{Lemma}
\newtheorem{proposition}[dummy]{Proposition}
\newtheorem{theorem}[dummy]{Theorem}
\theoremstyle{definition}
\newtheorem{examplle}{Example}[section]
\theoremstyle{remark}
\newtheorem*{remark}{Remark}
\newtheorem{exercise}{Exercise}[subsection]
\newtheorem{observation}{Observation}[section]


\newenvironment{claim}[1]{\par\noindent\textbf{Claim:}\space#1}{}

\makeatletter
\DeclareFontFamily{U}{tipa}{}
\DeclareFontShape{U}{tipa}{m}{n}{<->tipa10}{}
\newcommand{\arc@char}{{\usefont{U}{tipa}{m}{n}\symbol{62}}}%

\newcommand{\arc}[1]{\mathpalette\arc@arc{#1}}

\newcommand{\arc@arc}[2]{%
  \sbox0{$\m@th#1#2$}%
  \vbox{
    \hbox{\resizebox{\wd0}{\height}{\arc@char}}
    \nointerlineskip
    \box0
  }%
}
\makeatother

\setcounter{MaxMatrixCols}{20}
%%%%%%% ABS
\DeclarePairedDelimiter\abss{\lvert}{\rvert}%
\DeclarePairedDelimiter\normm{\lVert}{\rVert}%

% Swap the definition of \abs* and \norm*, so that \abs
% and \norm resizes the size of the brackets, and the
% starred version does not.
\makeatletter
\let\oldabs\abss
%\def\abs{\@ifstar{\oldabs}{\oldabs*}}
\newcommand{\abs}{\@ifstar{\oldabs}{\oldabs*}}
\newcommand{\norm}[1]{\left\lVert#1\right\rVert}
%\let\oldnorm\normm
%\def\norm{\@ifstar{\oldnorm}{\oldnorm*}}
%\renewcommand{norm}{\@ifstar{\oldnorm}{\oldnorm*}}
\makeatother

% \newcommand\what[1]{\ThisStyle{%
%     \setbox0=\hbox{$\SavedStyle#1$}%
%     \stackengine{-1.0\ht0+.5pt}{$\SavedStyle#1$}{%
%       \stretchto{\scaleto{\SavedStyle\mkern.15mu\char'136}{2.6\wd0}}{1.4\ht0}%
%     }{O}{c}{F}{T}{S}%
%   }
% }

% \newcommand\wtilde[1]{\ThisStyle{%
%     \setbox0=\hbox{$\SavedStyle#1$}%
%     \stackengine{-.1\LMpt}{$\SavedStyle#1$}{%
%       \stretchto{\scaleto{\SavedStyle\mkern.2mu\AC}{.5150\wd0}}{.6\ht0}%
%     }{O}{c}{F}{T}{S}%
%   }
% }

% \newcommand\wbar[1]{\ThisStyle{%
%     \setbox0=\hbox{$\SavedStyle#1$}%
%     \stackengine{.5pt+\LMpt}{$\SavedStyle#1$}{%
%       \rule{\wd0}{\dimexpr.3\LMpt+.3pt}%
%     }{O}{c}{F}{T}{S}%
%   }
% }

\newcommand{\bl}[1] {\boldsymbol{#1}}
\newcommand{\Wt}[1] {\stackrel{\sim}{\smash{#1}\rule{0pt}{1.1ex}}}
\newcommand{\wt}[1] {\widetilde{#1}}
\newcommand{\tf}[1] {\textbf{#1}}


%For boxed texts in align, use Aboxed{}
%otherwise use boxed{}

\DeclareMathSymbol{\widehatsym}{\mathord}{largesymbols}{"62}
\newcommand\lowerwidehatsym{%
  \text{\smash{\raisebox{-1.3ex}{%
    $\widehatsym$}}}}
\newcommand\fixwidehat[1]{%
  \mathchoice
    {\accentset{\displaystyle\lowerwidehatsym}{#1}}
    {\accentset{\textstyle\lowerwidehatsym}{#1}}
    {\accentset{\scriptstyle\lowerwidehatsym}{#1}}
    {\accentset{\scriptscriptstyle\lowerwidehatsym}{#1}}
  }


\newcommand{\cupdot}{\mathbin{\dot{\cup}}}
\newcommand{\bigcupdot}{\mathop{\dot{\bigcup}}}

\usepackage{graphicx}

\usepackage[toc,page]{appendix}

% text on arrow for xRightarrow
\makeatletter
%\newcommand{\xRightarrow}[2][]{\ext@arrow 0359\Rightarrowfill@{#1}{#2}}
\makeatother

% Arbitrary long arrow
\newcommand{\Rarrow}[1]{%
\parbox{#1}{\tikz{\draw[->](0,0)--(#1,0);}}
}

\newcommand{\LRarrow}[1]{%
\parbox{#1}{\tikz{\draw[<->](0,0)--(#1,0);}}
}


\makeatletter
\providecommand*{\rmodels}{%
  \mathrel{%
    \mathpalette\@rmodels\models
  }%
}
\newcommand*{\@rmodels}[2]{%
  \reflectbox{$\m@th#1#2$}%
}
\makeatother







\newcommand{\trcl}[1]{%
  \mathrm{trcl}{(#1)}
}



% Roman numerals
\makeatletter
\newcommand*{\rom}[1]{\expandafter\@slowromancap\romannumeral #1@}
\makeatother
% \\def \\b\([a-zA-Z]\) {\\boldsymbol{[a-zA-z]}}
% \\DeclareMathOperator{\\b\1}{\\textbf{\1}}


\DeclareMathOperator{\bx}{\textbf{x}}
\DeclareMathOperator{\bz}{\textbf{z}}
\DeclareMathOperator{\bff}{\textbf{f}}
\DeclareMathOperator{\ba}{\textbf{a}}
\DeclareMathOperator{\bk}{\textbf{k}}
\DeclareMathOperator{\bs}{\textbf{s}}
\DeclareMathOperator{\bh}{\textbf{h}}
\DeclareMathOperator{\bc}{\textbf{c}}
\DeclareMathOperator{\br}{\textbf{r}}
\DeclareMathOperator{\bi}{\textbf{i}}
\DeclareMathOperator{\bj}{\textbf{j}}
\DeclareMathOperator{\bn}{\textbf{n}}
\DeclareMathOperator{\be}{\textbf{e}}
\DeclareMathOperator{\bo}{\textbf{o}}
\DeclareMathOperator{\bU}{\textbf{U}}
\DeclareMathOperator{\bL}{\textbf{L}}
\DeclareMathOperator{\bV}{\textbf{V}}
\def \bzero {\mathbf{0}}
\def \btwo {\mathbf{2}}
\DeclareMathOperator{\bv}{\textbf{v}}
\DeclareMathOperator{\bp}{\textbf{p}}
\DeclareMathOperator{\bI}{\textbf{I}}
\DeclareMathOperator{\bM}{\textbf{M}}
\DeclareMathOperator{\bN}{\textbf{N}}
\DeclareMathOperator{\bK}{\textbf{K}}
\DeclareMathOperator{\bt}{\textbf{t}}
\DeclareMathOperator{\bb}{\textbf{b}}
\DeclareMathOperator{\bA}{\textbf{A}}
\DeclareMathOperator{\bX}{\textbf{X}}
\DeclareMathOperator{\bu}{\textbf{u}}
\DeclareMathOperator{\bS}{\textbf{S}}
\DeclareMathOperator{\bZ}{\textbf{Z}}
\DeclareMathOperator{\bJ}{\textbf{J}}
\DeclareMathOperator{\by}{\textbf{y}}
\DeclareMathOperator{\bw}{\textbf{w}}
\DeclareMathOperator{\bT}{\textbf{T}}
\DeclareMathOperator{\bF}{\textbf{F}}
\DeclareMathOperator{\bmm}{\textbf{m}}
\DeclareMathOperator{\bW}{\textbf{W}}
\DeclareMathOperator{\bR}{\textbf{R}}
\DeclareMathOperator{\bC}{\textbf{C}}
\DeclareMathOperator{\bD}{\textbf{D}}
\DeclareMathOperator{\bE}{\textbf{E}}
\DeclareMathOperator{\bQ}{\textbf{Q}}
\DeclareMathOperator{\bP}{\textbf{P}}
\DeclareMathOperator{\bY}{\textbf{Y}}
\DeclareMathOperator{\bH}{\textbf{H}}
\DeclareMathOperator{\bB}{\textbf{B}}
\DeclareMathOperator{\bG}{\textbf{G}}
\def \blambda {\symbf{\lambda}}
\def \boldeta {\symbf{\eta}}
\def \balpha {\symbf{\alpha}}
\def \bbeta {\symbf{\beta}}
\def \bgamma {\symbf{\gamma}}
\def \bxi {\symbf{\xi}}
\def \bLambda {\symbf{\Lambda}}
\def \bGamma {\symbf{\Gamma}}

\newcommand{\bto}{{\boldsymbol{\to}}}
\newcommand{\Ra}{\Rightarrow}
\newcommand\und[1]{\underline{#1}}
\newcommand\ove[1]{\overline{#1}}
\def \bPhi {\boldsymbol{\Phi}}
\def \btheta {\boldsymbol{\theta}}
\def \bTheta {\boldsymbol{\Theta}}
\def \bmu {\boldsymbol{\mu}}
\def \bphi {\boldsymbol{\phi}}
\def \bSigma {\boldsymbol{\Sigma}}
\def \lb {\left\{}
\def \rb {\right\}}
\def \la {\langle}
\def \ra {\rangle}
\def \caln {\mathcal{N}}
\def \dissum {\displaystyle\Sigma}
\def \dispro {\displaystyle\prod}
\def \E {\mathbb{E}}
\def \Q {\mathbb{Q}}
\def \N {\mathbb{N}}
\def \V {\mathbb{V}}
\def \R {\mathbb{R}}
\def \P {\mathbb{P}}
\def \A {\mathbb{A}}
\def \F {\mathbb{F}}
\def \Z {\mathbb{Z}}
\def \I {\mathbb{I}}
\def \C {\mathbb{C}}
\def \cala {\mathcal{A}}
\def \cale {\mathcal{E}}
\def \calb {\mathcal{B}}
\def \calq {\mathcal{Q}}
\def \calp {\mathcal{P}}
\def \cals {\mathcal{S}}
\def \calx {\mathcal{X}}
\def \caly {\mathcal{Y}}
\def \calg {\mathcal{G}}
\def \cald {\mathcal{D}}
\def \caln {\mathcal{N}}
\def \calr {\mathcal{R}}
\def \calt {\mathcal{T}}
\def \calm {\mathcal{M}}
\def \calw {\mathcal{W}}
\def \calc {\mathcal{C}}
\def \calv {\mathcal{V}}
\def \calf {\mathcal{F}}
\def \calk {\mathcal{K}}
\def \call {\mathcal{L}}
\def \calu {\mathcal{U}}
\def \calo {\mathcal{O}}
\def \calh {\mathcal{H}}
\def \cali {\mathcal{I}}

\def \bcup {\bigcup}

% set theory

\def \zfcc {\textbf{ZFC}^-}
\def \ac  {\textbf{AC}}
\def \gl  {\textbf{L }}
\def \gll {\textbf{L}}
\newcommand{\zfm}{$\textbf{ZF}^-$}

%\def \zfm {$\textbf{ZF}^-$}
\def \zfmm {\textbf{ZF}^-}
\def \wf {\textbf{WF }}
\def \on {\textbf{On }}
\def \cm {\textbf{M }}
\def \cn {\textbf{N }}
\def \cv {\textbf{V }}
\def \zc {\textbf{ZC }}
\def \zcm {\textbf{ZC}}
\def \zff {\textbf{ZF}}
\def \wfm {\textbf{WF}}
\def \onm {\textbf{On}}
\def \cmm {\textbf{M}}
\def \cnm {\textbf{N}}
\def \cvm {\textbf{V}}
\def \gchh {\textbf{GCH}}
\renewcommand{\restriction}{\mathord{\upharpoonright}}
\def \pred {\text{pred}}

\def \rank {\text{rank}}
\def \con {\text{Con}}
\def \deff {\text{Def}}


\def \uin {\underline{\in}}
\def \oin {\overline{\in}}
\def \uR {\underline{R}}
\def \oR {\overline{R}}
\def \uP {\underline{P}}
\def \oP {\overline{P}}

\def \dsum {\displaystyle\sum}

\def \Ra {\Rightarrow}

\def \e {\enspace}

\def \sgn {\operatorname{sgn}}
\def \gen {\operatorname{gen}}
\def \Hom {\operatorname{Hom}}
\def \hom {\operatorname{hom}}
\def \Sub {\operatorname{Sub}}

\def \supp {\operatorname{supp}}

\def \epiarrow {\twoheadarrow}
\def \monoarrow {\rightarrowtail}
\def \rrarrow {\rightrightarrows}

% \def \minus {\text{-}}
% \newcommand{\minus}{\scalebox{0.75}[1.0]{$-$}}
% \DeclareUnicodeCharacter{002D}{\minus}


\def \tril {\triangleleft}

\def \ACF {\text{ACF}}
\def \GL {\text{GL}}
\def \PGL {\text{PGL}}
\def \equal {=}
\def \deg {\text{deg}}
\def \degree {\text{degree}}
\def \app {\text{App}}
\def \FV {\text{FV}}
\def \conv {\text{conv}}
\def \cont {\text{cont}}
\DeclareMathOperator{\cl}{\textbf{CL}}
\DeclareMathOperator{\sg}{sg}
\DeclareMathOperator{\trdeg}{trdeg}
\def \Ord {\text{Ord}}

\DeclareMathOperator{\cf}{cf}
\DeclareMathOperator{\zfc}{ZFC}

%\DeclareMathOperator{\Th}{Th}
%\def \th {\text{Th}}
% \newcommand{\th}{\text{Th}}
\DeclareMathOperator{\type}{type}
\DeclareMathOperator{\zf}{\textbf{ZF}}
\def \fa {\mathfrak{a}}
\def \fb {\mathfrak{b}}
\def \fc {\mathfrak{c}}
\def \fd {\mathfrak{d}}
\def \fe {\mathfrak{e}}
\def \ff {\mathfrak{f}}
\def \fg {\mathfrak{g}}
\def \fh {\mathfrak{h}}
%\def \fi {\mathfrak{i}}
\def \fj {\mathfrak{j}}
\def \fk {\mathfrak{k}}
\def \fl {\mathfrak{l}}
\def \fm {\mathfrak{m}}
\def \fn {\mathfrak{n}}
\def \fo {\mathfrak{o}}
\def \fp {\mathfrak{p}}
\def \fq {\mathfrak{q}}
\def \fr {\mathfrak{r}}
\def \fs {\mathfrak{s}}
\def \ft {\mathfrak{t}}
\def \fu {\mathfrak{u}}
\def \fv {\mathfrak{v}}
\def \fw {\mathfrak{w}}
\def \fx {\mathfrak{x}}
\def \fy {\mathfrak{y}}
\def \fz {\mathfrak{z}}
\def \fA {\mathfrak{A}}
\def \fB {\mathfrak{B}}
\def \fC {\mathfrak{C}}
\def \fD {\mathfrak{D}}
\def \fE {\mathfrak{E}}
\def \fF {\mathfrak{F}}
\def \fG {\mathfrak{G}}
\def \fH {\mathfrak{H}}
\def \fI {\mathfrak{I}}
\def \fJ {\mathfrak{J}}
\def \fK {\mathfrak{K}}
\def \fL {\mathfrak{L}}
\def \fM {\mathfrak{M}}
\def \fN {\mathfrak{N}}
\def \fO {\mathfrak{O}}
\def \fP {\mathfrak{P}}
\def \fQ {\mathfrak{Q}}
\def \fR {\mathfrak{R}}
\def \fS {\mathfrak{S}}
\def \fT {\mathfrak{T}}
\def \fU {\mathfrak{U}}
\def \fV {\mathfrak{V}}
\def \fW {\mathfrak{W}}
\def \fX {\mathfrak{X}}
\def \fY {\mathfrak{Y}}
\def \fZ {\mathfrak{Z}}

\def \sfA {\textsf{A}}
\def \sfB {\textsf{B}}
\def \sfC {\textsf{C}}
\def \sfD {\textsf{D}}
\def \sfE {\textsf{E}}
\def \sfF {\textsf{F}}
\def \sfG {\textsf{G}}
\def \sfH {\textsf{H}}
\def \sfI {\textsf{I}}
\def \sfj {\textsf{J}}
\def \sfK {\textsf{K}}
\def \sfL {\textsf{L}}
\def \sfM {\textsf{M}}
\def \sfN {\textsf{N}}
\def \sfO {\textsf{O}}
\def \sfP {\textsf{P}}
\def \sfQ {\textsf{Q}}
\def \sfR {\textsf{R}}
\def \sfS {\textsf{S}}
\def \sfT {\textsf{T}}
\def \sfU {\textsf{U}}
\def \sfV {\textsf{V}}
\def \sfW {\textsf{W}}
\def \sfX {\textsf{X}}
\def \sfY {\textsf{Y}}
\def \sfZ {\textsf{Z}}
\def \sfa {\textsf{a}}
\def \sfb {\textsf{b}}
\def \sfc {\textsf{c}}
\def \sfd {\textsf{d}}
\def \sfe {\textsf{e}}
\def \sff {\textsf{f}}
\def \sfg {\textsf{g}}
\def \sfh {\textsf{h}}
\def \sfi {\textsf{i}}
\def \sfj {\textsf{j}}
\def \sfk {\textsf{k}}
\def \sfl {\textsf{l}}
\def \sfm {\textsf{m}}
\def \sfn {\textsf{n}}
\def \sfo {\textsf{o}}
\def \sfp {\textsf{p}}
\def \sfq {\textsf{q}}
\def \sfr {\textsf{r}}
\def \sfs {\textsf{s}}
\def \sft {\textsf{t}}
\def \sfu {\textsf{u}}
\def \sfv {\textsf{v}}
\def \sfw {\textsf{w}}
\def \sfx {\textsf{x}}
\def \sfy {\textsf{y}}
\def \sfz {\textsf{z}}



%\DeclareMathOperator{\ker}{ker}
\DeclareMathOperator{\im}{im}

\DeclareMathOperator{\inn}{Inn}
\DeclareMathOperator{\AC}{\textbf{AC}}
\DeclareMathOperator{\cod}{cod}
\DeclareMathOperator{\dom}{dom}
\DeclareMathOperator{\ran}{ran}
\DeclareMathOperator{\textd}{d}
\DeclareMathOperator{\td}{d}
\DeclareMathOperator{\id}{id}
\DeclareMathOperator{\LT}{LT}
\DeclareMathOperator{\Mat}{Mat}
\DeclareMathOperator{\Eq}{Eq}
\DeclareMathOperator{\irr}{irr}
\DeclareMathOperator{\Fr}{Fr}
\DeclareMathOperator{\Gal}{Gal}
\DeclareMathOperator{\lcm}{lcm}
\DeclareMathOperator{\alg}{\text{alg}}
\DeclareMathOperator{\Th}{Th}

\DeclareMathOperator{\DAG}{DAG}
\DeclareMathOperator{\ODAG}{ODAG}

% \varprod
\DeclareSymbolFont{largesymbolsA}{U}{txexa}{m}{n}
\DeclareMathSymbol{\varprod}{\mathop}{largesymbolsA}{16}
% \DeclareMathSymbol{\tonm}{\boldsymbol{\to}\textbf{Nm}}
\def \tonm {\bto\textbf{Nm}}
\def \tohm {\bto\textbf{Hm}}

% Category theory
\DeclareMathOperator{\Ab}{\textbf{Ab}}
\DeclareMathOperator{\Alg}{\textbf{Alg}}
\DeclareMathOperator{\Rng}{\textbf{Rng}}
\DeclareMathOperator{\Sets}{\textbf{Sets}}
\DeclareMathOperator{\Met}{\textbf{Met}}
\DeclareMathOperator{\BA}{\textbf{BA}}
\DeclareMathOperator{\Mon}{\textbf{Mon}}
\DeclareMathOperator{\Top}{\textbf{Top}}
\DeclareMathOperator{\Aut}{\textbf{Aut}}
\DeclareMathOperator{\RMod}{R-\textbf{Mod}}
\DeclareMathOperator{\RAlg}{R-\textbf{Alg}}
\DeclareMathOperator{\LF}{LF}
\DeclareMathOperator{\op}{op}
% Model theory
\DeclareMathOperator{\tp}{tp}
\DeclareMathOperator{\Diag}{Diag}
\DeclareMathOperator{\el}{el}
\DeclareMathOperator{\depth}{depth}
\DeclareMathOperator{\FO}{FO}
\DeclareMathOperator{\fin}{fin}
\DeclareMathOperator{\qr}{qr}
\DeclareMathOperator{\Mod}{Mod}
\DeclareMathOperator{\TC}{TC}
\DeclareMathOperator{\KH}{KH}
\DeclareMathOperator{\Part}{Part}
\DeclareMathOperator{\Infset}{\textsf{Infset}}
\DeclareMathOperator{\DLO}{\textsf{DLO}}
\DeclareMathOperator{\sfMod}{\textsf{Mod}}
\DeclareMathOperator{\AbG}{\textsf{AbG}}
\DeclareMathOperator{\sfACF}{\textsf{ACF}}
% Computability Theorem
\DeclareMathOperator{\Tot}{Tot}
\DeclareMathOperator{\graph}{graph}
\DeclareMathOperator{\Fin}{Fin}
\DeclareMathOperator{\Cof}{Cof}
\DeclareMathOperator{\lh}{lh}
% Commutative Algebra
\DeclareMathOperator{\ord}{ord}
\DeclareMathOperator{\Idem}{Idem}
\DeclareMathOperator{\zdiv}{z.div}
\DeclareMathOperator{\Frac}{Frac}
\DeclareMathOperator{\rad}{rad}
\DeclareMathOperator{\nil}{nil}
\DeclareMathOperator{\Ann}{Ann}
\DeclareMathOperator{\End}{End}
\DeclareMathOperator{\coim}{coim}
\DeclareMathOperator{\coker}{coker}
\DeclareMathOperator{\Bil}{Bil}
\DeclareMathOperator{\Tril}{Tril}
% Topology
\newcommand{\interior}[1]{%
  {\kern0pt#1}^{\mathrm{o}}%
}

% \makeatletter
% \newcommand{\vect}[1]{%
%   \vbox{\m@th \ialign {##\crcr
%   \vectfill\crcr\noalign{\kern-\p@ \nointerlineskip}
%   $\hfil\displaystyle{#1}\hfil$\crcr}}}
% \def\vectfill{%
%   $\m@th\smash-\mkern-7mu%
%   \cleaders\hbox{$\mkern-2mu\smash-\mkern-2mu$}\hfill
%   \mkern-7mu\raisebox{-3.81pt}[\p@][\p@]{$\mathord\mathchar"017E$}$}

% \newcommand{\amsvect}{%
%   \mathpalette {\overarrow@\vectfill@}}
% \def\vectfill@{\arrowfill@\relbar\relbar{\raisebox{-3.81pt}[\p@][\p@]{$\mathord\mathchar"017E$}}}

% \newcommand{\amsvectb}{%
% \newcommand{\vect}{%
%   \mathpalette {\overarrow@\vectfillb@}}
% \newcommand{\vecbar}{%
%   \scalebox{0.8}{$\relbar$}}
% \def\vectfillb@{\arrowfill@\vecbar\vecbar{\raisebox{-4.35pt}[\p@][\p@]{$\mathord\mathchar"017E$}}}
% \makeatother
% \bigtimes

\DeclareFontFamily{U}{mathx}{\hyphenchar\font45}
\DeclareFontShape{U}{mathx}{m}{n}{
      <5> <6> <7> <8> <9> <10>
      <10.95> <12> <14.4> <17.28> <20.74> <24.88>
      mathx10
      }{}
\DeclareSymbolFont{mathx}{U}{mathx}{m}{n}
\DeclareMathSymbol{\bigtimes}{1}{mathx}{"91}
% \odiv
\DeclareFontFamily{U}{matha}{\hyphenchar\font45}
\DeclareFontShape{U}{matha}{m}{n}{
      <5> <6> <7> <8> <9> <10> gen * matha
      <10.95> matha10 <12> <14.4> <17.28> <20.74> <24.88> matha12
      }{}
\DeclareSymbolFont{matha}{U}{matha}{m}{n}
\DeclareMathSymbol{\odiv}         {2}{matha}{"63}


\newcommand\subsetsim{\mathrel{%
  \ooalign{\raise0.2ex\hbox{\scalebox{0.9}{$\subset$}}\cr\hidewidth\raise-0.85ex\hbox{\scalebox{0.9}{$\sim$}}\hidewidth\cr}}}
\newcommand\simsubset{\mathrel{%
  \ooalign{\raise-0.2ex\hbox{\scalebox{0.9}{$\subset$}}\cr\hidewidth\raise0.75ex\hbox{\scalebox{0.9}{$\sim$}}\hidewidth\cr}}}

\newcommand\simsubsetsim{\mathrel{%
  \ooalign{\raise0ex\hbox{\scalebox{0.8}{$\subset$}}\cr\hidewidth\raise1ex\hbox{\scalebox{0.75}{$\sim$}}\hidewidth\cr\raise-0.95ex\hbox{\scalebox{0.8}{$\sim$}}\cr\hidewidth}}}
\newcommand{\stcomp}[1]{{#1}^{\mathsf{c}}}

\setlength{\baselineskip}{0.8in}

\stackMath
\newcommand\yrightarrow[2][]{\mathrel{%
  \setbox2=\hbox{\stackon{\scriptstyle#1}{\scriptstyle#2}}%
  \stackunder[0pt]{%
    \xrightarrow{\makebox[\dimexpr\wd2\relax]{$\scriptstyle#2$}}%
  }{%
   \scriptstyle#1\,%
  }%
}}
\newcommand\yleftarrow[2][]{\mathrel{%
  \setbox2=\hbox{\stackon{\scriptstyle#1}{\scriptstyle#2}}%
  \stackunder[0pt]{%
    \xleftarrow{\makebox[\dimexpr\wd2\relax]{$\scriptstyle#2$}}%
  }{%
   \scriptstyle#1\,%
  }%
}}
\newcommand\yRightarrow[2][]{\mathrel{%
  \setbox2=\hbox{\stackon{\scriptstyle#1}{\scriptstyle#2}}%
  \stackunder[0pt]{%
    \xRightarrow{\makebox[\dimexpr\wd2\relax]{$\scriptstyle#2$}}%
  }{%
   \scriptstyle#1\,%
  }%
}}
\newcommand\yLeftarrow[2][]{\mathrel{%
  \setbox2=\hbox{\stackon{\scriptstyle#1}{\scriptstyle#2}}%
  \stackunder[0pt]{%
    \xLeftarrow{\makebox[\dimexpr\wd2\relax]{$\scriptstyle#2$}}%
  }{%
   \scriptstyle#1\,%
  }%
}}

\newcommand\altxrightarrow[2][0pt]{\mathrel{\ensurestackMath{\stackengine%
  {\dimexpr#1-7.5pt}{\xrightarrow{\phantom{#2}}}{\scriptstyle\!#2\,}%
  {O}{c}{F}{F}{S}}}}
\newcommand\altxleftarrow[2][0pt]{\mathrel{\ensurestackMath{\stackengine%
  {\dimexpr#1-7.5pt}{\xleftarrow{\phantom{#2}}}{\scriptstyle\!#2\,}%
  {O}{c}{F}{F}{S}}}}

\newenvironment{bsm}{% % short for 'bracketed small matrix'
  \left[ \begin{smallmatrix} }{%
  \end{smallmatrix} \right]}

\newenvironment{psm}{% % short for ' small matrix'
  \left( \begin{smallmatrix} }{%
  \end{smallmatrix} \right)}

\newcommand{\bbar}[1]{\mkern 1.5mu\overline{\mkern-1.5mu#1\mkern-1.5mu}\mkern 1.5mu}

\newcommand{\bigzero}{\mbox{\normalfont\Large\bfseries 0}}
\newcommand{\rvline}{\hspace*{-\arraycolsep}\vline\hspace*{-\arraycolsep}}

\font\zallman=Zallman at 40pt
\font\elzevier=Elzevier at 40pt

\newcommand\isoto{\stackrel{\textstyle\sim}{\smash{\longrightarrow}\rule{0pt}{0.4ex}}}
\newcommand\embto{\stackrel{\textstyle\prec}{\smash{\longrightarrow}\rule{0pt}{0.4ex}}}

% from http://www.actual.world/resources/tex/doc/TikZ.pdf

\tikzset{
modal/.style={>=stealth’,shorten >=1pt,shorten <=1pt,auto,node distance=1.5cm,
semithick},
world/.style={circle,draw,minimum size=0.5cm,fill=gray!15},
point/.style={circle,draw,inner sep=0.5mm,fill=black},
reflexive above/.style={->,loop,looseness=7,in=120,out=60},
reflexive below/.style={->,loop,looseness=7,in=240,out=300},
reflexive left/.style={->,loop,looseness=7,in=150,out=210},
reflexive right/.style={->,loop,looseness=7,in=30,out=330}
}


\makeatletter
\newcommand*{\doublerightarrow}[2]{\mathrel{
  \settowidth{\@tempdima}{$\scriptstyle#1$}
  \settowidth{\@tempdimb}{$\scriptstyle#2$}
  \ifdim\@tempdimb>\@tempdima \@tempdima=\@tempdimb\fi
  \mathop{\vcenter{
    \offinterlineskip\ialign{\hbox to\dimexpr\@tempdima+1em{##}\cr
    \rightarrowfill\cr\noalign{\kern.5ex}
    \rightarrowfill\cr}}}\limits^{\!#1}_{\!#2}}}
\newcommand*{\triplerightarrow}[1]{\mathrel{
  \settowidth{\@tempdima}{$\scriptstyle#1$}
  \mathop{\vcenter{
    \offinterlineskip\ialign{\hbox to\dimexpr\@tempdima+1em{##}\cr
    \rightarrowfill\cr\noalign{\kern.5ex}
    \rightarrowfill\cr\noalign{\kern.5ex}
    \rightarrowfill\cr}}}\limits^{\!#1}}}
\makeatother

% $A\doublerightarrow{a}{bcdefgh}B$

% $A\triplerightarrow{d_0,d_1,d_2}B$


\usepackage{ebproof}
\DeclareMathOperator{\Efq}{Efq}
\def \texists {\tilde{\exists}}
\def \tvee {\tilde{\vee}}
\DeclareMathOperator{\Stab}{Stab}
\author{Helmut Schwichtenberg \& Stanley S. Wainer}
\date{\today}
\title{Proofs and Computations}
\hypersetup{
 pdfauthor={Helmut Schwichtenberg \& Stanley S. Wainer},
 pdftitle={Proofs and Computations},
 pdfkeywords={},
 pdfsubject={},
 pdfcreator={Emacs 27.1 (Org mode 9.3)}, 
 pdflang={English}}
\begin{document}

\maketitle
\tableofcontents

\section{Logic}
\label{sec:org8e3244c}

\subsection{Natural Deduction}
\label{sec:org94529aa}
Negation is defined by
\begin{equation*}
\neg A:=(A\to\bot)
\end{equation*}

\begin{definition}[Gentzen]
\textbf{Subformulas} of \(A\) are defined by
\begin{enumerate}
\item \(A\) is a subformula of \(A\)
\item if \(B\circ C\) is a subformula of \(A\) then so are \(B,C\) for \(\circ=\to,\wedge,\vee\)
\item if \(\forall_xB(x)\) or \(\exists_xB(x)\) is a subformula of \(A\), then
so is \(B(r)\)
\end{enumerate}
\end{definition}



\begin{definition}[]
The notions of \textbf{positive}, \textbf{negative}, \textbf{strictly positive} subformula are defined
in a similar style
\begin{enumerate}
\item \(A\) is a positive and a strictly positive subformula of itself
\item if \(B\wedge C\) or \(B\vee C\) is a positive (negative, strictly
positive) subformula of \(A\), then so are \(B, C\)
\item if \(\forall_xB(x)\) or \(\exists_xB(x)\) is a positive (negative,
strictly positive) subformula of \(A\), then so is \(B(r)\)
\item if \(B\to C\) is a positive (negative) subformula of
\(A\), then \(B\) is a negative (positive)subformula of \(A\), and \(C\)
is a positive (negative)subformula of \(A\)
\item if \(B\to C\) is a strictly subformula of \(A\), then so is \(C\)
\end{enumerate}


A strictly positive subformula of \(A\) is also called a \textbf{strictly positive
part} (\textbf{s.p.p.}) of \(A\)
\end{definition}


\begin{equation*}
\begin{prooftree}[center=false]
\hypo{[u:A]}
\ellipsis{D}{B}
\infer1[\(\to^+u\)]{A\to B}
\end{prooftree}
\hspace{1cm}
\begin{prooftree}[center=false]
\hypo{}
\ellipsis{M}{A\to B}
\hypo{}
\ellipsis{N}{A}
\infer2[\(\to^-\)]{B}
\end{prooftree}
\end{equation*}


The rule \(\forall^+x\) with conclusion \(\forall_xA\) is subject to the
following \textbf{(eigen-)variable condition}: the derivation \(M\) of the premise
\(A\) should not contain any open assumption having \(x\) as a free variable

\begin{equation*}
\begin{prooftree}[center=false]
\hypo{}
\ellipsis{M}{A}
\infer1[\(\forall^+x\)]{\forall_xA}
\end{prooftree}
\hspace{1cm}
\begin{prooftree}[center=false]
\hypo{}
\ellipsis{M}{\forall_xA(x)}
\hypo{r}
\infer2[\(\forall^-\)]{A(r)}
\end{prooftree}
\end{equation*}

\begin{equation*}
\begin{prooftree}[center=false]
\hypo{}
\ellipsis{M}{A}
\infer1[\(\vee_0^+\)]{A\vee B}
\end{prooftree}\hspace{1cm}
\begin{prooftree}[center=false]
\hypo{}
\ellipsis{M}{B}
\infer1[\(\vee_1^+\)]{A\vee B}
\end{prooftree}\hspace{1cm}
\begin{prooftree}[center=false]
\hypo{}
\ellipsis{M}{A\vee B}
\hypo{[u:A]}
\ellipsis{N}{C}
\hypo{[v:B]}
\ellipsis{K}{C}
\infer3[\(\vee^-u,v\)]{C}
\end{prooftree}
\end{equation*}

\begin{equation*}
\begin{prooftree}[center=false]
\hypo{}
\ellipsis{M}{A}
\hypo{}
\ellipsis{N}{B}
\infer2[\(\wedge^+\)]{A\wedge B}
\end{prooftree}\hspace{1cm}
\begin{prooftree}[center=false]
\hypo{}
\ellipsis{M}{A\wedge B}
\hypo{[u:A]\quad [v:B]}
\ellipsis{N}{C}
\infer2[\(\wedge^-u,v\)]{C}
\end{prooftree}
\end{equation*}

\begin{equation*}
\begin{prooftree}[center=false]
\hypo{r}
\hypo{}
\ellipsis{M}{A(r)}
\infer2[\(\exists^+\)]{\exists_xA(x)}
\end{prooftree}\hspace{1cm}
\begin{prooftree}[center=false]
\hypo{}
\ellipsis{M}{\exists_xA}
\hypo{[u:A]}
\ellipsis{N}{B}
\infer2[\(\exists^- x,u(\text{var.cond.})\)]{B}
\end{prooftree}
\end{equation*}

Rule \(\exists^-x,u\) is subject to an \textbf{(eigen-)variable condition}: in the
derivation \(N\) the variable \(x\)
\begin{enumerate}
\item should not occur free in the formula of any open assumption other than \(u:A\)
\item should not occur free in \(B\)
\end{enumerate}


For each of the connectives \(\wedge, \vee, \exists\) the rules and the
following axioms are equivalent over minimal logic
\begin{equation*}
\exists^+:A\to\exists_xA,\quad \exists^-:\exists_xA\to\forall_x(A\to B)\to B(x\not\in FV(B))
\end{equation*}

\begin{lemma}[]
The following are derivable
\begin{align*}
&(A\wedge B\to C)\leftrightarrow(A\to B\to C)\\
&(A\to B\wedge C)\leftrightarrow(A\to B)\wedge(A\to C)\\
&(A\vee B\to C)\leftrightarrow(A\to C)\wedge(B\to C)\\
&(A\to B)\vee(A\to C)\to (A\to B\vee C)\\
&\exists_x(A\to B)\to(\forall_xA\to B)\quad\text{if }x\not\in FV(B)\\
&\forall_x(A\to B)\leftrightarrow(A\to\forall_xB)\quad\text{if }x\not\in FV(A)\\
&\forall_x(A\to B)\leftrightarrow(\exists_xA\to B)\quad\text{if }x\not\in FV(B)\\
&\exists_x(A\to B)\to(A\to\exists_xB)\quad\text{if }x\not\in FV(A)
\end{align*}
\end{lemma}

\begin{proof}
\begin{equation*}
\begin{prooftree}[center=false]
\hypo{u:\exists_x(A\to B)}
\hypo{x}
\hypo{w:A\to B}\hypo{v:A}
\infer2{B}
\infer2{\exists_xB}
\infer2[\(\exists^-x,w\)]{\exists_xB}
\infer1[\(\to^+v\)]{A\to\exists_xB}
\infer1[\(\to^+u\)]{\exists_x(A\to B)to A\to\exists_xB}
\end{prooftree}
\end{equation*}
\end{proof}

\textbf{weak disjuction} and \textbf{weak existence}
\begin{equation*}
A\tilde{\vee}B:=\neg A\to\neg B\to\bot,\quad
\tilde{\exists}_xA:=\neg\forall_x\neg A
\end{equation*}
These weak variants are no stronger than the proper ones
\begin{equation*}
A\vee B\to A\tilde{\vee}B,\quad\exists_xA\to\tilde{\exists}_xA
\end{equation*}
by putting \(C:=\bot\) in \(\vee^-\) and \(B:=\bot\) in \(\exists^-\)

Moreover
\begin{align*}
\tilde{\exists}_{x_1,\dots,x_n}A&:=\forall_{x_1,\dots,x_n}(A\to\bot)\to\bot\\
\tilde{\exists}_{x_1,\dots,x_n}(A_1\tilde{\wedge}\cdots\tilde{\wedge}A_m)&:=
\forall_{x_1,\dots,x_n}(A_1\to\cdots\to A_m\to\bot)\to\bot
\end{align*}

In the previous contexts falsity \(\bot\) plays no role. We may change this
and require \textbf{ex-falso-quodlibet} axioms of the form
\begin{equation*}
\forall_{\vec{x}}(\bot\to R\vec{x})
\end{equation*}
with \(R\) a relation symbol distinct from \(\bot\). Let Efq denote the set
of all such axioms. A formula \(A\) is called \textbf{intuitionistically derivable},
written \(\vdash_iA\) if \(\Efq\vdash A\). We write \(\Gamma\vdash_i B\) for
\(\Gamma\cup\Efq\vdash B\)

We may even go further and require \textbf{stability} axioms, of the form
\begin{equation*}
\forall_{\vec{x}}(\neg\neg R\vec{x}\to R\vec{x})
\end{equation*}
with \(R\) again a relation distinct from \(\bot\). Let Stab denote the set
of all these axioms. A formula \(A\) is called \textbf{classically derivable}, written
\(\vdash_c A\), if \(\Stab\vdash A\). We write \(\Gamma\vdash_cB\) for
\(\Gamma\cup\Stab\vdash B\)

\begin{theorem}[Stability, or principle of indirect proof]
\begin{enumerate}
\item \(\vdash(\neg\neg A\to A)\to(\neg\neg B\to B)\to\neg\neg(A\wedge B)\to
      A\wedge B\)
\item \(\vdash(\neg\neg B\to B)\to\neg\neg(A\to B)\to A\to B\)
\item \(\vdash(\neg\neg A\to A)\to\neg\neg\forall_xA\to A\)
\item \(\vdash_c\neg\neg A\to A\) for every formula \(A\) without \(\vee,\exists\)
\end{enumerate}
\end{theorem}

\begin{proof}
\begin{enumerate}
\item \((\neg\neg A\to A)\to\neg\neg A\to A\)
\setcounter{enumi}{1}
\item \begin{equation*}
\begin{prooftree}[center=false]
\hypo{u:\neg\neg B\to B}
\hypo{v:\neg\neg(A\to B)}
\hypo{u_1:\neg B}
\hypo{u_2:A\to B}
\hypo{w:A}
\infer2{B}
\infer2{\bot}
\infer1[\(\to^+u_2\)]{\neg(A\to B)}
\infer2{
\begin{prooftree}[center=false]
\hypo{\bot}\infer1[\(\to^+u_1\)]{\neg\neg B}
\end{prooftree}
}
\infer2{B}
\end{prooftree}
\end{equation*}
\item \begin{equation*}
\begin{prooftree}[center=false]
\hypo{u:\neg\neg A\to A}
\hypo{v:\neg\neg\forall_xA}
\hypo{u_1:\neg A}
\hypo{u_2:\forall_xA}
\hypo{x}
\infer2{A}
\infer2{
\begin{prooftree}[center=false]
\hypo{\bot}\infer1[\(\to^+u_2\)]{\neg\forall_xA}
\end{prooftree}
}
\infer2{\begin{prooftree}[center=false]
\hypo{\bot}\infer1[\(\to^+u_1\)]{\neg\neg A}
\end{prooftree}}
\infer2{A}
\end{prooftree}
\end{equation*}
\item Induction on \(A\). The case \(R\vec{t}\) with \(R\) distinct from
\(\bot\) is given by Stab. In the case \(\bot\) the desired derivation is
\begin{equation*}
\begin{prooftree}[center=false]
\hypo{v:(\bot\to\bot)\to\bot}
\hypo{u:\bot}
\infer1[\(\to^+u\)]{\bot\to\bot}
\infer2{\bot}
\end{prooftree}
\end{equation*}
In the case \(A\wedge B,A\to B\) and \(\forall_xA\) use 1,2,3 respectively
\end{enumerate}
\end{proof}

\begin{lemma}[]
The following are derivable
\begin{align*}
(\tilde{\exists}_xA\to B)\to\forall_x(A\to B)&\quad\text{ if }x\not\in FV(B)\\
(\neg\neg B\to B)\to\forall_x(A\to B)\to\tilde{\exists}_xA\to B&\quad\text{ if }x\not\in FV(B)\\
(\bot\to B[x:=c])\to(A\to\tilde{\exists}_xB)\to\texists_x(A\to B)&\quad\text{ if }x\not\in FV(A)\\
\texists_x(A\to B)\to A\to\texists_xB&\quad\text{ if }x\not \in FV(A)
\end{align*}
The last two items can also be seen as simplifying a weakly existentially
quantified implication whose premise doesn't contain the quantified variable.
In case the conclusion does not contain the quantified variable we have
align
\begin{align*}
(\neg\neg B\to B)\to\texists_x(A\to B)\to\forall_xA\to B&\quad\text{ if }x\not \in FV(A)\\
\forall_x(\neg\neg A\to A)\to(\forall_xA\to B)\to\texists_x(A\to B)&\quad\text{ if }x\not \in FV(A)
\end{align*}
\end{lemma}

\begin{proof}
\begin{enumerate}
\setcounter{enumi}{2}
\item Writing \(B_0\) for \(B[x:=c]\) we have
\begin{equation*}
\resizebox{0.9\textwidth}{!}{
\begin{prooftree}[center=false]
\hypo{\forall_x\neg(A\to B)}
\hypo{c}
\infer2{\neg(A\to B_0)}
\hypo{\bot\to B_0}
\hypo{A\to\texists_xB}
\hypo{u_2:A}
\infer2{\texists_xB}
\hypo{\forall_x\neg(A\to B)}
\hypo{x}
\infer2{\neg(A\to B)}
\hypo{u_1:B}
\infer1{A\to B}
\infer2{
\begin{prooftree}
\hypo{\bot}
\infer1[\(\to^+u\)]{\neg B}
\infer1{\forall_x\neg B}
\end{prooftree}
}
\infer2{\bot}
\infer2{B_0}
\infer1[\(\to^+u_2\)]{A\to B_0}
\infer2{\bot}
\end{prooftree}}
\end{equation*}
\item \begin{equation*}
\begin{prooftree}[center=false]
\hypo{\texists_x(A\to B)}
\hypo{\forall_x\neg B}
\hypo{x}
\infer2{\neg B}
\hypo{u_1:A\to B}
\hypo{A}
\infer2{B}
\infer2{\bot}
\infer1[\(\to^+u_1\)]{\neg(A\to B)}
\infer1{\forall_x\neg(A\to B)}
\infer2{\bot}
\end{prooftree}
\end{equation*}
\end{enumerate}
\end{proof}

An immediate consequence of 6 is the classical derivability of the "drinker
formula" \(\texists_x(Px\to\forall_xPx)\) to be read "in every non-empty bar
there is a person s.t. if this person drinks, then everybody drinks"

\begin{corollary}[]
\begin{alignat*}{2}
&\vdash_c(\texists_xA\to B)\leftrightarrow\forall_x(A\to B)\quad
&&\text{ if }x\not\in FV(B)\text{ and }B\text{ without }\forall,\exists\\
&\vdash_i(A\to\texists_xB)\leftrightarrow\texists_x(A\to B)
&&\text{ if }x\not\in FV(A)\\
&\vdash_c\texists_x(A\to B)\leftrightarrow(\forall_xA\to B)\quad
&&\text{ if }x\not\in FV(B)\text{ and }A,B\text{ without }\forall,\exists
\end{alignat*}
\end{corollary}

\begin{lemma}[]
The following are derivable
\begin{align*}
&(A\tvee B\to C)\to (A\to C)\wedge(B\to C)\\
&(\neg\neg C\to C)\to(A\to C)\to(B\to C)\to A\tvee B\to C\\
&(\bot\to B)\to(A\to B \tvee C)\to(A\to B)\tvee(A\to C)\\
&(A\to B)\tvee(A\to C)\to A\to B\tvee C\\
&(\neg\neg C\to C)\to(A\to C)\tvee(B\to C)\to A\to B\to C\\
&(\bot\to C)\to(A\to B\to C)\to(A\to C)\tvee(B\to C)
\end{align*}
\end{lemma}

\begin{proof}
\begin{enumerate}
\setcounter{enumi}{5}
\item \begin{equation*}
\begin{prooftree}[center=false]
\hypo{\neg(B\to C)}
\hypo{\bot\to C}
\hypo{\neg(A\to C)}
\hypo{A\to B\to C}
\hypo{u_1:A}
\infer2{B\to C}
\hypo{u_2:B}
\infer2{C}
\infer1[\(\to^+u_1\)]{A\to C}
\infer2{\bot}
\infer2{C}
\infer1[\(\to^+u_2\)]{B\to C}
\infer2{\bot}
\end{prooftree}
\end{equation*}
\end{enumerate}
\end{proof}

\begin{corollary}[]
\begin{alignat*}{2}
&\vdash_c(A\tvee B\to C)\leftrightarrow(A\to C)\wedge(B\to C)\quad&&\text{ for }C
\text{ without }\forall,\exists\\
&\vdash_i(A\to B\tvee C)\leftrightarrow(A\to B)\tvee(A\to C)\\
&\vdash_c(A\to C)\tvee(B\to C)\leftrightarrow(A\to B\to C)
&&\text{for }C\text{ without }\forall,\exists
\end{alignat*}
\end{corollary}

\begin{remark}
It is easy to see that weak disjuction and the weak existential quantifier
satisfy the same axioms as the strong variants, if one restricts the
conclusion of the elimination axioms to formulas without \(\forall,
   \exists\). In fact we have
\begin{align*}
& \vdash A\to A\tvee B,\quad\vdash B\to A\tvee B\\
&\vdash_c A\tvee B\to(A\to C)\to(B\to C)\to C\quad(C\text{ without }\forall,\exists)\\
&\vdash A\to\tvee_xA\\
&\vdash_c \texists_xA\to\forall_x(A\to B)\to B\quad(x\not\in FV(B),B\text{ without }\forall,\exists)
\end{align*}
\end{remark}

\begin{proof}
\begin{enumerate}
\setcounter{enumi}{1}
\item \begin{equation*}
\resizebox{0.9\textwidth}{!}{
\begin{prooftree}[center=false]
\hypo{\neg\neg C\to C}
\hypo{\neg A\to\neg B\to\bot}
\hypo{u_1:\neg C}
\hypo{A\to C}
\hypo{u_2:A}
\infer2{C}
\infer2{\bot}
\infer1[\(\to^+u_2\)]{\neg A}
\infer2{\neg B\to\bot}
\hypo{u_1:\neg C}
\hypo{B\to C}
\hypo{u_3:B}
\infer2{C}
\infer2{\bot}
\infer1[\(\to^+u_3\)]{\neg B}
\infer2{\bot}
\infer1[\(\to^+u_1\)]{\neg\neg C}
\infer2{C}
\end{prooftree}}
\end{equation*}
\end{enumerate}
\end{proof}

\(A\) is derivable in classical logic iff its translation \(A^g\) is
derivable in minimal logic

\begin{definition}[Gödel-Gentzen translation $A^g$]
\begin{align*}
(R\vec{t})^g&:=\neg\neg R\vec{t}\quad\text{ for }R\text{ distinct from }\bot\\
\bot^g&:=\bot\\
(A\vee B)^g&:=A^g\tvee B^g\\
(\exists_xA)^g&:=\texists_xA^g\\
(A\circ B)^g&:=A^g\circ B^g\quad\text{ for }\circ=\to,\wedge\\
(\forall_xA)^g&:=\forall_xA^g
\end{align*}
\end{definition}

\begin{lemma}[]
\(\vdash\neg\neg A^g\to A^g\)
\end{lemma}

\begin{proof}
Induction on \(A\)

\emph{Case} \(R\vec{t}\) with \(R\) distinct from \(\bot\). We must show
\(\neg\neg\neg\neg R\vec{t}\to\neg\neg R\vec{t}\), which is a special case of
\(\vdash\neg\neg\neg B\to\neg B\)

\emph{Case} \(\bot\). Use \(\vdash\neg\neg\bot\to\bot\)

\emph{Case} \(A\vee B\). \(\vdash\neg\neg(A^g\tvee B^g)\to A^g\tvee B^g\) is a special
case of \(\vdash\neg\neg(\neg C\to \neg D\to\bot)\to\neg C\to\neg D\to\bot\)
\begin{equation*}
\begin{prooftree}[center=false]
\hypo{\neg\neg(\neg C\to\neg D\to\bot)}
\hypo{u_1:\neg C\to\neg D\to\bot}
\hypo{\neg C}
\infer2{\neg D\to\bot}
\hypo{\neg D}
\infer2{\bot}
\infer1[\(\to^+u_1\)]{\neg(\neg C\to\neg D\to\bot)}
\infer2{\bot}
\end{prooftree}
\end{equation*}

\emph{Case} \(\exists_xA\). We need to show
\(\vdash\neg\neg\texists_xA^g\to\texists_xA^g\), and this is a special case
of \(\vdash\neg\neg\neg B\to\neg B\)
\end{proof}

\begin{theorem}[]
\begin{enumerate}
\item \(\Gamma\vdash_c A\) implies \(\Gamma^g\vdash A^g\)
\item \(\Gamma^g\vdash A^g\) implies \(\Gamma\vdash_c A\) for \(\Gamma,A\) without \(\vee,\exists\)
\end{enumerate}
\end{theorem}
\end{document}