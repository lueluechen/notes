% Created 2021-03-11 四 14:41
% Intended LaTeX compiler: pdflatex
\documentclass[11pt]{article}
\usepackage[utf8]{inputenc}
\usepackage[T1]{fontenc}
\usepackage{graphicx}
\usepackage{grffile}
\usepackage{longtable}
\usepackage{wrapfig}
\usepackage{rotating}
\usepackage[normalem]{ulem}
\usepackage{amsmath}
\usepackage{textcomp}
\usepackage{amssymb}
\usepackage{capt-of}
\usepackage{hyperref}
%%%%%%%%%%%%%%%%%%%%%%%%%%%%%%%%%%%%%%
%% TIPS                                 %%
%%%%%%%%%%%%%%%%%%%%%%%%%%%%%%%%%%%%%%
% \substack{a\\b} for multiple lines text

\usepackage[utf8]{inputenc}

\usepackage[B1,T1]{fontenc}

% pdfplots will load xolor automatically without option
\usepackage[dvipsnames]{xcolor}
%%%%%%%%%%%%%%%%%%%%%%%%%%%%%%%%%%%%%%%
%% MATH related pacakge                  %%
%%%%%%%%%%%%%%%%%%%%%%%%%%%%%%%%%%%%%%%
% \usepackage{amsmath} mathtools loads the amsmath
\usepackage{amsmath}
\usepackage{mathtools}


\usepackage{amsthm}
\usepackage{amsbsy}

%\usepackage{commath}

\usepackage{amssymb}
\usepackage{mathrsfs}
%\usepackage{mathabx}
\usepackage{stmaryrd}
\usepackage{empheq}

%for \not\ll
\usepackage{centernot}

\usepackage{scalerel}
\usepackage{stackengine}
\usepackage{stackrel}

\usepackage{nicematrix}
\usepackage{tensor}
\usepackage{blkarray}
\usepackage{siunitx}
\usepackage[f]{esvect}

\usepackage{unicode-math}
\setmainfont{TeX Gyre Pagella}
% \setmathfont{STIX}
%\setmathfont{texgyrepagella-math.otf}
%\setmathfont{Libertinus Math}
\setmathfont{Latin Modern Math}

 
% \setmathfont[range={\smwhtdiamond,\enclosediamond,\varlrtriangle}]{Latin Modern Math}
 \setmathfont[range={\rightrightarrows,\twoheadrightarrow,\leftrightsquigarrow,\triangledown,\vartriangle}]{XITS Math}
 \setmathfont[range={\int,\setminus}]{Libertinus Math}
 \setmathfont[range={\mathalpha}]{TeX Gyre Pagella Math}
% unicode is not good at this!
%\let\nmodels\nvDash


%%%%%%%%%%%%%%%%%%%%%%%%%%%%%%%%%%%%%%%
%% TIKZ related packages                 %%
%%%%%%%%%%%%%%%%%%%%%%%%%%%%%%%%%%%%%%%

\usepackage{pgfplots}
\pgfplotsset{compat=1.15}
\usepackage{tikz}
\usepackage{tikz-cd}
\usepackage{tikz-qtree}

\usetikzlibrary{arrows,positioning,calc,fadings,decorations,matrix,decorations,shapes.misc}
%setting from geogebra
\definecolor{ccqqqq}{rgb}{0.8,0,0}


%%%%%%%%%%%%%%%%%%%%%%%%%%%%%%%%%%%%%%%
%% MISCLELLANEOUS packages               %%
%%%%%%%%%%%%%%%%%%%%%%%%%%%%%%%%%%%%%%%
\usepackage[most]{tcolorbox}
\usepackage{threeparttable}
\usepackage{tabularx}

\usepackage{enumitem}

% wrong with preview
\usepackage{subcaption}
\usepackage{caption}
% {\aunclfamily\Huge}
\usepackage{auncial}

\usepackage{float}

\usepackage{fancyhdr}

\usepackage{ifthen}
\usepackage{xargs}


\usepackage{imakeidx}
\usepackage{hyperref}
\usepackage{soul}


%\usepackage[xetex]{preview}
%%%%%%%%%%%%%%%%%%%%%%%%%%%%%%%%%%%%%%%
%% USEPACKAGES end                       %%
%%%%%%%%%%%%%%%%%%%%%%%%%%%%%%%%%%%%%%%

% \setlist{nosep}
% \numberwithin{equation}{subsection}
% \fancyhead{} % Clear the headers
% \renewcommand{\headrulewidth}{0pt} % Width of line at top of page
% \fancyhead[R]{\slshape\leftmark} % Mark right [R] of page with Chapter name [\leftmark]
% \pagestyle{fancy} % Set default style for all content pages (not TOC, etc)


% \newlength\shlength
% \newcommand\vect[2][0]{\setlength\shlength{#1pt}%
%   \stackengine{-5.6pt}{$#2$}{\smash{$\kern\shlength%
%     \stackengine{7.55pt}{$\mathchar"017E$}%
%       {\rule{\widthof{$#2$}}{.57pt}\kern.4pt}{O}{r}{F}{F}{L}\kern-\shlength$}}%
%       {O}{c}{F}{T}{S}}


\indexsetup{othercode=\small}
\makeindex[columns=2,options={-s /media/wu/file/stuuudy/notes/index_style.ist},intoc]
\makeatletter
\def\@idxitem{\par\hangindent 0pt}
\makeatother


%\newcounter{dummy} \numberwithin{dummy}{section}
\newtheorem{dummy}{dummy}[section]
\theoremstyle{definition}
\newtheorem{definition}[dummy]{Definition}
\theoremstyle{plain}
\newtheorem{corollary}[dummy]{Corollary}
\newtheorem{lemma}[dummy]{Lemma}
\newtheorem{proposition}[dummy]{Proposition}
\newtheorem{theorem}[dummy]{Theorem}
\theoremstyle{definition}
\newtheorem{examplle}{Example}[section]
\theoremstyle{remark}
\newtheorem*{remark}{Remark}
\newtheorem{exercise}{Exercise}[subsection]
\newtheorem{observation}{Observation}[section]


\newenvironment{claim}[1]{\par\noindent\textbf{Claim:}\space#1}{}

\makeatletter
\DeclareFontFamily{U}{tipa}{}
\DeclareFontShape{U}{tipa}{m}{n}{<->tipa10}{}
\newcommand{\arc@char}{{\usefont{U}{tipa}{m}{n}\symbol{62}}}%

\newcommand{\arc}[1]{\mathpalette\arc@arc{#1}}

\newcommand{\arc@arc}[2]{%
  \sbox0{$\m@th#1#2$}%
  \vbox{
    \hbox{\resizebox{\wd0}{\height}{\arc@char}}
    \nointerlineskip
    \box0
  }%
}
\makeatother

\setcounter{MaxMatrixCols}{20}
%%%%%%% ABS
\DeclarePairedDelimiter\abss{\lvert}{\rvert}%
\DeclarePairedDelimiter\normm{\lVert}{\rVert}%

% Swap the definition of \abs* and \norm*, so that \abs
% and \norm resizes the size of the brackets, and the
% starred version does not.
\makeatletter
\let\oldabs\abss
%\def\abs{\@ifstar{\oldabs}{\oldabs*}}
\newcommand{\abs}{\@ifstar{\oldabs}{\oldabs*}}
\newcommand{\norm}[1]{\left\lVert#1\right\rVert}
%\let\oldnorm\normm
%\def\norm{\@ifstar{\oldnorm}{\oldnorm*}}
%\renewcommand{norm}{\@ifstar{\oldnorm}{\oldnorm*}}
\makeatother

% \newcommand\what[1]{\ThisStyle{%
%     \setbox0=\hbox{$\SavedStyle#1$}%
%     \stackengine{-1.0\ht0+.5pt}{$\SavedStyle#1$}{%
%       \stretchto{\scaleto{\SavedStyle\mkern.15mu\char'136}{2.6\wd0}}{1.4\ht0}%
%     }{O}{c}{F}{T}{S}%
%   }
% }

% \newcommand\wtilde[1]{\ThisStyle{%
%     \setbox0=\hbox{$\SavedStyle#1$}%
%     \stackengine{-.1\LMpt}{$\SavedStyle#1$}{%
%       \stretchto{\scaleto{\SavedStyle\mkern.2mu\AC}{.5150\wd0}}{.6\ht0}%
%     }{O}{c}{F}{T}{S}%
%   }
% }

% \newcommand\wbar[1]{\ThisStyle{%
%     \setbox0=\hbox{$\SavedStyle#1$}%
%     \stackengine{.5pt+\LMpt}{$\SavedStyle#1$}{%
%       \rule{\wd0}{\dimexpr.3\LMpt+.3pt}%
%     }{O}{c}{F}{T}{S}%
%   }
% }

\newcommand{\bl}[1] {\boldsymbol{#1}}
\newcommand{\Wt}[1] {\stackrel{\sim}{\smash{#1}\rule{0pt}{1.1ex}}}
\newcommand{\wt}[1] {\widetilde{#1}}
\newcommand{\tf}[1] {\textbf{#1}}


%For boxed texts in align, use Aboxed{}
%otherwise use boxed{}

\DeclareMathSymbol{\widehatsym}{\mathord}{largesymbols}{"62}
\newcommand\lowerwidehatsym{%
  \text{\smash{\raisebox{-1.3ex}{%
    $\widehatsym$}}}}
\newcommand\fixwidehat[1]{%
  \mathchoice
    {\accentset{\displaystyle\lowerwidehatsym}{#1}}
    {\accentset{\textstyle\lowerwidehatsym}{#1}}
    {\accentset{\scriptstyle\lowerwidehatsym}{#1}}
    {\accentset{\scriptscriptstyle\lowerwidehatsym}{#1}}
  }


\newcommand{\cupdot}{\mathbin{\dot{\cup}}}
\newcommand{\bigcupdot}{\mathop{\dot{\bigcup}}}

\usepackage{graphicx}

\usepackage[toc,page]{appendix}

% text on arrow for xRightarrow
\makeatletter
%\newcommand{\xRightarrow}[2][]{\ext@arrow 0359\Rightarrowfill@{#1}{#2}}
\makeatother

% Arbitrary long arrow
\newcommand{\Rarrow}[1]{%
\parbox{#1}{\tikz{\draw[->](0,0)--(#1,0);}}
}

\newcommand{\LRarrow}[1]{%
\parbox{#1}{\tikz{\draw[<->](0,0)--(#1,0);}}
}


\makeatletter
\providecommand*{\rmodels}{%
  \mathrel{%
    \mathpalette\@rmodels\models
  }%
}
\newcommand*{\@rmodels}[2]{%
  \reflectbox{$\m@th#1#2$}%
}
\makeatother







\newcommand{\trcl}[1]{%
  \mathrm{trcl}{(#1)}
}



% Roman numerals
\makeatletter
\newcommand*{\rom}[1]{\expandafter\@slowromancap\romannumeral #1@}
\makeatother
% \\def \\b\([a-zA-Z]\) {\\boldsymbol{[a-zA-z]}}
% \\DeclareMathOperator{\\b\1}{\\textbf{\1}}


\DeclareMathOperator{\bx}{\textbf{x}}
\DeclareMathOperator{\bz}{\textbf{z}}
\DeclareMathOperator{\bff}{\textbf{f}}
\DeclareMathOperator{\ba}{\textbf{a}}
\DeclareMathOperator{\bk}{\textbf{k}}
\DeclareMathOperator{\bs}{\textbf{s}}
\DeclareMathOperator{\bh}{\textbf{h}}
\DeclareMathOperator{\bc}{\textbf{c}}
\DeclareMathOperator{\br}{\textbf{r}}
\DeclareMathOperator{\bi}{\textbf{i}}
\DeclareMathOperator{\bj}{\textbf{j}}
\DeclareMathOperator{\bn}{\textbf{n}}
\DeclareMathOperator{\be}{\textbf{e}}
\DeclareMathOperator{\bo}{\textbf{o}}
\DeclareMathOperator{\bU}{\textbf{U}}
\DeclareMathOperator{\bL}{\textbf{L}}
\DeclareMathOperator{\bV}{\textbf{V}}
\def \bzero {\mathbf{0}}
\def \btwo {\mathbf{2}}
\DeclareMathOperator{\bv}{\textbf{v}}
\DeclareMathOperator{\bp}{\textbf{p}}
\DeclareMathOperator{\bI}{\textbf{I}}
\DeclareMathOperator{\bM}{\textbf{M}}
\DeclareMathOperator{\bN}{\textbf{N}}
\DeclareMathOperator{\bK}{\textbf{K}}
\DeclareMathOperator{\bt}{\textbf{t}}
\DeclareMathOperator{\bb}{\textbf{b}}
\DeclareMathOperator{\bA}{\textbf{A}}
\DeclareMathOperator{\bX}{\textbf{X}}
\DeclareMathOperator{\bu}{\textbf{u}}
\DeclareMathOperator{\bS}{\textbf{S}}
\DeclareMathOperator{\bZ}{\textbf{Z}}
\DeclareMathOperator{\bJ}{\textbf{J}}
\DeclareMathOperator{\by}{\textbf{y}}
\DeclareMathOperator{\bw}{\textbf{w}}
\DeclareMathOperator{\bT}{\textbf{T}}
\DeclareMathOperator{\bF}{\textbf{F}}
\DeclareMathOperator{\bmm}{\textbf{m}}
\DeclareMathOperator{\bW}{\textbf{W}}
\DeclareMathOperator{\bR}{\textbf{R}}
\DeclareMathOperator{\bC}{\textbf{C}}
\DeclareMathOperator{\bD}{\textbf{D}}
\DeclareMathOperator{\bE}{\textbf{E}}
\DeclareMathOperator{\bQ}{\textbf{Q}}
\DeclareMathOperator{\bP}{\textbf{P}}
\DeclareMathOperator{\bY}{\textbf{Y}}
\DeclareMathOperator{\bH}{\textbf{H}}
\DeclareMathOperator{\bB}{\textbf{B}}
\DeclareMathOperator{\bG}{\textbf{G}}
\def \blambda {\symbf{\lambda}}
\def \boldeta {\symbf{\eta}}
\def \balpha {\symbf{\alpha}}
\def \bbeta {\symbf{\beta}}
\def \bgamma {\symbf{\gamma}}
\def \bxi {\symbf{\xi}}
\def \bLambda {\symbf{\Lambda}}
\def \bGamma {\symbf{\Gamma}}

\newcommand{\bto}{{\boldsymbol{\to}}}
\newcommand{\Ra}{\Rightarrow}
\newcommand\und[1]{\underline{#1}}
\newcommand\ove[1]{\overline{#1}}
\def \bPhi {\boldsymbol{\Phi}}
\def \btheta {\boldsymbol{\theta}}
\def \bTheta {\boldsymbol{\Theta}}
\def \bmu {\boldsymbol{\mu}}
\def \bphi {\boldsymbol{\phi}}
\def \bSigma {\boldsymbol{\Sigma}}
\def \lb {\left\{}
\def \rb {\right\}}
\def \la {\langle}
\def \ra {\rangle}
\def \caln {\mathcal{N}}
\def \dissum {\displaystyle\Sigma}
\def \dispro {\displaystyle\prod}
\def \E {\mathbb{E}}
\def \Q {\mathbb{Q}}
\def \N {\mathbb{N}}
\def \V {\mathbb{V}}
\def \R {\mathbb{R}}
\def \P {\mathbb{P}}
\def \A {\mathbb{A}}
\def \F {\mathbb{F}}
\def \Z {\mathbb{Z}}
\def \I {\mathbb{I}}
\def \C {\mathbb{C}}
\def \cala {\mathcal{A}}
\def \cale {\mathcal{E}}
\def \calb {\mathcal{B}}
\def \calq {\mathcal{Q}}
\def \calp {\mathcal{P}}
\def \cals {\mathcal{S}}
\def \calx {\mathcal{X}}
\def \caly {\mathcal{Y}}
\def \calg {\mathcal{G}}
\def \cald {\mathcal{D}}
\def \caln {\mathcal{N}}
\def \calr {\mathcal{R}}
\def \calt {\mathcal{T}}
\def \calm {\mathcal{M}}
\def \calw {\mathcal{W}}
\def \calc {\mathcal{C}}
\def \calv {\mathcal{V}}
\def \calf {\mathcal{F}}
\def \calk {\mathcal{K}}
\def \call {\mathcal{L}}
\def \calu {\mathcal{U}}
\def \calo {\mathcal{O}}
\def \calh {\mathcal{H}}
\def \cali {\mathcal{I}}

\def \bcup {\bigcup}

% set theory

\def \zfcc {\textbf{ZFC}^-}
\def \ac  {\textbf{AC}}
\def \gl  {\textbf{L }}
\def \gll {\textbf{L}}
\newcommand{\zfm}{$\textbf{ZF}^-$}

%\def \zfm {$\textbf{ZF}^-$}
\def \zfmm {\textbf{ZF}^-}
\def \wf {\textbf{WF }}
\def \on {\textbf{On }}
\def \cm {\textbf{M }}
\def \cn {\textbf{N }}
\def \cv {\textbf{V }}
\def \zc {\textbf{ZC }}
\def \zcm {\textbf{ZC}}
\def \zff {\textbf{ZF}}
\def \wfm {\textbf{WF}}
\def \onm {\textbf{On}}
\def \cmm {\textbf{M}}
\def \cnm {\textbf{N}}
\def \cvm {\textbf{V}}
\def \gchh {\textbf{GCH}}
\renewcommand{\restriction}{\mathord{\upharpoonright}}
\def \pred {\text{pred}}

\def \rank {\text{rank}}
\def \con {\text{Con}}
\def \deff {\text{Def}}


\def \uin {\underline{\in}}
\def \oin {\overline{\in}}
\def \uR {\underline{R}}
\def \oR {\overline{R}}
\def \uP {\underline{P}}
\def \oP {\overline{P}}

\def \dsum {\displaystyle\sum}

\def \Ra {\Rightarrow}

\def \e {\enspace}

\def \sgn {\operatorname{sgn}}
\def \gen {\operatorname{gen}}
\def \Hom {\operatorname{Hom}}
\def \hom {\operatorname{hom}}
\def \Sub {\operatorname{Sub}}

\def \supp {\operatorname{supp}}

\def \epiarrow {\twoheadarrow}
\def \monoarrow {\rightarrowtail}
\def \rrarrow {\rightrightarrows}

% \def \minus {\text{-}}
% \newcommand{\minus}{\scalebox{0.75}[1.0]{$-$}}
% \DeclareUnicodeCharacter{002D}{\minus}


\def \tril {\triangleleft}

\def \ACF {\text{ACF}}
\def \GL {\text{GL}}
\def \PGL {\text{PGL}}
\def \equal {=}
\def \deg {\text{deg}}
\def \degree {\text{degree}}
\def \app {\text{App}}
\def \FV {\text{FV}}
\def \conv {\text{conv}}
\def \cont {\text{cont}}
\DeclareMathOperator{\cl}{\textbf{CL}}
\DeclareMathOperator{\sg}{sg}
\DeclareMathOperator{\trdeg}{trdeg}
\def \Ord {\text{Ord}}

\DeclareMathOperator{\cf}{cf}
\DeclareMathOperator{\zfc}{ZFC}

%\DeclareMathOperator{\Th}{Th}
%\def \th {\text{Th}}
% \newcommand{\th}{\text{Th}}
\DeclareMathOperator{\type}{type}
\DeclareMathOperator{\zf}{\textbf{ZF}}
\def \fa {\mathfrak{a}}
\def \fb {\mathfrak{b}}
\def \fc {\mathfrak{c}}
\def \fd {\mathfrak{d}}
\def \fe {\mathfrak{e}}
\def \ff {\mathfrak{f}}
\def \fg {\mathfrak{g}}
\def \fh {\mathfrak{h}}
%\def \fi {\mathfrak{i}}
\def \fj {\mathfrak{j}}
\def \fk {\mathfrak{k}}
\def \fl {\mathfrak{l}}
\def \fm {\mathfrak{m}}
\def \fn {\mathfrak{n}}
\def \fo {\mathfrak{o}}
\def \fp {\mathfrak{p}}
\def \fq {\mathfrak{q}}
\def \fr {\mathfrak{r}}
\def \fs {\mathfrak{s}}
\def \ft {\mathfrak{t}}
\def \fu {\mathfrak{u}}
\def \fv {\mathfrak{v}}
\def \fw {\mathfrak{w}}
\def \fx {\mathfrak{x}}
\def \fy {\mathfrak{y}}
\def \fz {\mathfrak{z}}
\def \fA {\mathfrak{A}}
\def \fB {\mathfrak{B}}
\def \fC {\mathfrak{C}}
\def \fD {\mathfrak{D}}
\def \fE {\mathfrak{E}}
\def \fF {\mathfrak{F}}
\def \fG {\mathfrak{G}}
\def \fH {\mathfrak{H}}
\def \fI {\mathfrak{I}}
\def \fJ {\mathfrak{J}}
\def \fK {\mathfrak{K}}
\def \fL {\mathfrak{L}}
\def \fM {\mathfrak{M}}
\def \fN {\mathfrak{N}}
\def \fO {\mathfrak{O}}
\def \fP {\mathfrak{P}}
\def \fQ {\mathfrak{Q}}
\def \fR {\mathfrak{R}}
\def \fS {\mathfrak{S}}
\def \fT {\mathfrak{T}}
\def \fU {\mathfrak{U}}
\def \fV {\mathfrak{V}}
\def \fW {\mathfrak{W}}
\def \fX {\mathfrak{X}}
\def \fY {\mathfrak{Y}}
\def \fZ {\mathfrak{Z}}

\def \sfA {\textsf{A}}
\def \sfB {\textsf{B}}
\def \sfC {\textsf{C}}
\def \sfD {\textsf{D}}
\def \sfE {\textsf{E}}
\def \sfF {\textsf{F}}
\def \sfG {\textsf{G}}
\def \sfH {\textsf{H}}
\def \sfI {\textsf{I}}
\def \sfj {\textsf{J}}
\def \sfK {\textsf{K}}
\def \sfL {\textsf{L}}
\def \sfM {\textsf{M}}
\def \sfN {\textsf{N}}
\def \sfO {\textsf{O}}
\def \sfP {\textsf{P}}
\def \sfQ {\textsf{Q}}
\def \sfR {\textsf{R}}
\def \sfS {\textsf{S}}
\def \sfT {\textsf{T}}
\def \sfU {\textsf{U}}
\def \sfV {\textsf{V}}
\def \sfW {\textsf{W}}
\def \sfX {\textsf{X}}
\def \sfY {\textsf{Y}}
\def \sfZ {\textsf{Z}}
\def \sfa {\textsf{a}}
\def \sfb {\textsf{b}}
\def \sfc {\textsf{c}}
\def \sfd {\textsf{d}}
\def \sfe {\textsf{e}}
\def \sff {\textsf{f}}
\def \sfg {\textsf{g}}
\def \sfh {\textsf{h}}
\def \sfi {\textsf{i}}
\def \sfj {\textsf{j}}
\def \sfk {\textsf{k}}
\def \sfl {\textsf{l}}
\def \sfm {\textsf{m}}
\def \sfn {\textsf{n}}
\def \sfo {\textsf{o}}
\def \sfp {\textsf{p}}
\def \sfq {\textsf{q}}
\def \sfr {\textsf{r}}
\def \sfs {\textsf{s}}
\def \sft {\textsf{t}}
\def \sfu {\textsf{u}}
\def \sfv {\textsf{v}}
\def \sfw {\textsf{w}}
\def \sfx {\textsf{x}}
\def \sfy {\textsf{y}}
\def \sfz {\textsf{z}}



%\DeclareMathOperator{\ker}{ker}
\DeclareMathOperator{\im}{im}

\DeclareMathOperator{\inn}{Inn}
\DeclareMathOperator{\AC}{\textbf{AC}}
\DeclareMathOperator{\cod}{cod}
\DeclareMathOperator{\dom}{dom}
\DeclareMathOperator{\ran}{ran}
\DeclareMathOperator{\textd}{d}
\DeclareMathOperator{\td}{d}
\DeclareMathOperator{\id}{id}
\DeclareMathOperator{\LT}{LT}
\DeclareMathOperator{\Mat}{Mat}
\DeclareMathOperator{\Eq}{Eq}
\DeclareMathOperator{\irr}{irr}
\DeclareMathOperator{\Fr}{Fr}
\DeclareMathOperator{\Gal}{Gal}
\DeclareMathOperator{\lcm}{lcm}
\DeclareMathOperator{\alg}{\text{alg}}
\DeclareMathOperator{\Th}{Th}

\DeclareMathOperator{\DAG}{DAG}
\DeclareMathOperator{\ODAG}{ODAG}

% \varprod
\DeclareSymbolFont{largesymbolsA}{U}{txexa}{m}{n}
\DeclareMathSymbol{\varprod}{\mathop}{largesymbolsA}{16}
% \DeclareMathSymbol{\tonm}{\boldsymbol{\to}\textbf{Nm}}
\def \tonm {\bto\textbf{Nm}}
\def \tohm {\bto\textbf{Hm}}

% Category theory
\DeclareMathOperator{\Ab}{\textbf{Ab}}
\DeclareMathOperator{\Alg}{\textbf{Alg}}
\DeclareMathOperator{\Rng}{\textbf{Rng}}
\DeclareMathOperator{\Sets}{\textbf{Sets}}
\DeclareMathOperator{\Met}{\textbf{Met}}
\DeclareMathOperator{\BA}{\textbf{BA}}
\DeclareMathOperator{\Mon}{\textbf{Mon}}
\DeclareMathOperator{\Top}{\textbf{Top}}
\DeclareMathOperator{\Aut}{\textbf{Aut}}
\DeclareMathOperator{\RMod}{R-\textbf{Mod}}
\DeclareMathOperator{\RAlg}{R-\textbf{Alg}}
\DeclareMathOperator{\LF}{LF}
\DeclareMathOperator{\op}{op}
% Model theory
\DeclareMathOperator{\tp}{tp}
\DeclareMathOperator{\Diag}{Diag}
\DeclareMathOperator{\el}{el}
\DeclareMathOperator{\depth}{depth}
\DeclareMathOperator{\FO}{FO}
\DeclareMathOperator{\fin}{fin}
\DeclareMathOperator{\qr}{qr}
\DeclareMathOperator{\Mod}{Mod}
\DeclareMathOperator{\TC}{TC}
\DeclareMathOperator{\KH}{KH}
\DeclareMathOperator{\Part}{Part}
\DeclareMathOperator{\Infset}{\textsf{Infset}}
\DeclareMathOperator{\DLO}{\textsf{DLO}}
\DeclareMathOperator{\sfMod}{\textsf{Mod}}
\DeclareMathOperator{\AbG}{\textsf{AbG}}
\DeclareMathOperator{\sfACF}{\textsf{ACF}}
% Computability Theorem
\DeclareMathOperator{\Tot}{Tot}
\DeclareMathOperator{\graph}{graph}
\DeclareMathOperator{\Fin}{Fin}
\DeclareMathOperator{\Cof}{Cof}
\DeclareMathOperator{\lh}{lh}
% Commutative Algebra
\DeclareMathOperator{\ord}{ord}
\DeclareMathOperator{\Idem}{Idem}
\DeclareMathOperator{\zdiv}{z.div}
\DeclareMathOperator{\Frac}{Frac}
\DeclareMathOperator{\rad}{rad}
\DeclareMathOperator{\nil}{nil}
\DeclareMathOperator{\Ann}{Ann}
\DeclareMathOperator{\End}{End}
\DeclareMathOperator{\coim}{coim}
\DeclareMathOperator{\coker}{coker}
\DeclareMathOperator{\Bil}{Bil}
\DeclareMathOperator{\Tril}{Tril}
% Topology
\newcommand{\interior}[1]{%
  {\kern0pt#1}^{\mathrm{o}}%
}

% \makeatletter
% \newcommand{\vect}[1]{%
%   \vbox{\m@th \ialign {##\crcr
%   \vectfill\crcr\noalign{\kern-\p@ \nointerlineskip}
%   $\hfil\displaystyle{#1}\hfil$\crcr}}}
% \def\vectfill{%
%   $\m@th\smash-\mkern-7mu%
%   \cleaders\hbox{$\mkern-2mu\smash-\mkern-2mu$}\hfill
%   \mkern-7mu\raisebox{-3.81pt}[\p@][\p@]{$\mathord\mathchar"017E$}$}

% \newcommand{\amsvect}{%
%   \mathpalette {\overarrow@\vectfill@}}
% \def\vectfill@{\arrowfill@\relbar\relbar{\raisebox{-3.81pt}[\p@][\p@]{$\mathord\mathchar"017E$}}}

% \newcommand{\amsvectb}{%
% \newcommand{\vect}{%
%   \mathpalette {\overarrow@\vectfillb@}}
% \newcommand{\vecbar}{%
%   \scalebox{0.8}{$\relbar$}}
% \def\vectfillb@{\arrowfill@\vecbar\vecbar{\raisebox{-4.35pt}[\p@][\p@]{$\mathord\mathchar"017E$}}}
% \makeatother
% \bigtimes

\DeclareFontFamily{U}{mathx}{\hyphenchar\font45}
\DeclareFontShape{U}{mathx}{m}{n}{
      <5> <6> <7> <8> <9> <10>
      <10.95> <12> <14.4> <17.28> <20.74> <24.88>
      mathx10
      }{}
\DeclareSymbolFont{mathx}{U}{mathx}{m}{n}
\DeclareMathSymbol{\bigtimes}{1}{mathx}{"91}
% \odiv
\DeclareFontFamily{U}{matha}{\hyphenchar\font45}
\DeclareFontShape{U}{matha}{m}{n}{
      <5> <6> <7> <8> <9> <10> gen * matha
      <10.95> matha10 <12> <14.4> <17.28> <20.74> <24.88> matha12
      }{}
\DeclareSymbolFont{matha}{U}{matha}{m}{n}
\DeclareMathSymbol{\odiv}         {2}{matha}{"63}


\newcommand\subsetsim{\mathrel{%
  \ooalign{\raise0.2ex\hbox{\scalebox{0.9}{$\subset$}}\cr\hidewidth\raise-0.85ex\hbox{\scalebox{0.9}{$\sim$}}\hidewidth\cr}}}
\newcommand\simsubset{\mathrel{%
  \ooalign{\raise-0.2ex\hbox{\scalebox{0.9}{$\subset$}}\cr\hidewidth\raise0.75ex\hbox{\scalebox{0.9}{$\sim$}}\hidewidth\cr}}}

\newcommand\simsubsetsim{\mathrel{%
  \ooalign{\raise0ex\hbox{\scalebox{0.8}{$\subset$}}\cr\hidewidth\raise1ex\hbox{\scalebox{0.75}{$\sim$}}\hidewidth\cr\raise-0.95ex\hbox{\scalebox{0.8}{$\sim$}}\cr\hidewidth}}}
\newcommand{\stcomp}[1]{{#1}^{\mathsf{c}}}

\setlength{\baselineskip}{0.8in}

\stackMath
\newcommand\yrightarrow[2][]{\mathrel{%
  \setbox2=\hbox{\stackon{\scriptstyle#1}{\scriptstyle#2}}%
  \stackunder[0pt]{%
    \xrightarrow{\makebox[\dimexpr\wd2\relax]{$\scriptstyle#2$}}%
  }{%
   \scriptstyle#1\,%
  }%
}}
\newcommand\yleftarrow[2][]{\mathrel{%
  \setbox2=\hbox{\stackon{\scriptstyle#1}{\scriptstyle#2}}%
  \stackunder[0pt]{%
    \xleftarrow{\makebox[\dimexpr\wd2\relax]{$\scriptstyle#2$}}%
  }{%
   \scriptstyle#1\,%
  }%
}}
\newcommand\yRightarrow[2][]{\mathrel{%
  \setbox2=\hbox{\stackon{\scriptstyle#1}{\scriptstyle#2}}%
  \stackunder[0pt]{%
    \xRightarrow{\makebox[\dimexpr\wd2\relax]{$\scriptstyle#2$}}%
  }{%
   \scriptstyle#1\,%
  }%
}}
\newcommand\yLeftarrow[2][]{\mathrel{%
  \setbox2=\hbox{\stackon{\scriptstyle#1}{\scriptstyle#2}}%
  \stackunder[0pt]{%
    \xLeftarrow{\makebox[\dimexpr\wd2\relax]{$\scriptstyle#2$}}%
  }{%
   \scriptstyle#1\,%
  }%
}}

\newcommand\altxrightarrow[2][0pt]{\mathrel{\ensurestackMath{\stackengine%
  {\dimexpr#1-7.5pt}{\xrightarrow{\phantom{#2}}}{\scriptstyle\!#2\,}%
  {O}{c}{F}{F}{S}}}}
\newcommand\altxleftarrow[2][0pt]{\mathrel{\ensurestackMath{\stackengine%
  {\dimexpr#1-7.5pt}{\xleftarrow{\phantom{#2}}}{\scriptstyle\!#2\,}%
  {O}{c}{F}{F}{S}}}}

\newenvironment{bsm}{% % short for 'bracketed small matrix'
  \left[ \begin{smallmatrix} }{%
  \end{smallmatrix} \right]}

\newenvironment{psm}{% % short for ' small matrix'
  \left( \begin{smallmatrix} }{%
  \end{smallmatrix} \right)}

\newcommand{\bbar}[1]{\mkern 1.5mu\overline{\mkern-1.5mu#1\mkern-1.5mu}\mkern 1.5mu}

\newcommand{\bigzero}{\mbox{\normalfont\Large\bfseries 0}}
\newcommand{\rvline}{\hspace*{-\arraycolsep}\vline\hspace*{-\arraycolsep}}

\font\zallman=Zallman at 40pt
\font\elzevier=Elzevier at 40pt

\newcommand\isoto{\stackrel{\textstyle\sim}{\smash{\longrightarrow}\rule{0pt}{0.4ex}}}
\newcommand\embto{\stackrel{\textstyle\prec}{\smash{\longrightarrow}\rule{0pt}{0.4ex}}}

% from http://www.actual.world/resources/tex/doc/TikZ.pdf

\tikzset{
modal/.style={>=stealth’,shorten >=1pt,shorten <=1pt,auto,node distance=1.5cm,
semithick},
world/.style={circle,draw,minimum size=0.5cm,fill=gray!15},
point/.style={circle,draw,inner sep=0.5mm,fill=black},
reflexive above/.style={->,loop,looseness=7,in=120,out=60},
reflexive below/.style={->,loop,looseness=7,in=240,out=300},
reflexive left/.style={->,loop,looseness=7,in=150,out=210},
reflexive right/.style={->,loop,looseness=7,in=30,out=330}
}


\makeatletter
\newcommand*{\doublerightarrow}[2]{\mathrel{
  \settowidth{\@tempdima}{$\scriptstyle#1$}
  \settowidth{\@tempdimb}{$\scriptstyle#2$}
  \ifdim\@tempdimb>\@tempdima \@tempdima=\@tempdimb\fi
  \mathop{\vcenter{
    \offinterlineskip\ialign{\hbox to\dimexpr\@tempdima+1em{##}\cr
    \rightarrowfill\cr\noalign{\kern.5ex}
    \rightarrowfill\cr}}}\limits^{\!#1}_{\!#2}}}
\newcommand*{\triplerightarrow}[1]{\mathrel{
  \settowidth{\@tempdima}{$\scriptstyle#1$}
  \mathop{\vcenter{
    \offinterlineskip\ialign{\hbox to\dimexpr\@tempdima+1em{##}\cr
    \rightarrowfill\cr\noalign{\kern.5ex}
    \rightarrowfill\cr\noalign{\kern.5ex}
    \rightarrowfill\cr}}}\limits^{\!#1}}}
\makeatother

% $A\doublerightarrow{a}{bcdefgh}B$

% $A\triplerightarrow{d_0,d_1,d_2}B$


\author{gouziwu}
\date{\today}
\title{Rough Set Theory: A True Landmark in Data Analysis}
\hypersetup{
 pdfauthor={gouziwu},
 pdftitle={Rough Set Theory: A True Landmark in Data Analysis},
 pdfkeywords={},
 pdfsubject={},
 pdfcreator={Emacs 27.1 (Org mode 9.3)}, 
 pdflang={English}}
\begin{document}

\maketitle
\tableofcontents

\section{Rough Sets on Fuzzy Approximation Spaces and Intuitionistic Fuzzy Approximation Spaces}
\label{sec:org857a6d1}
\subsection{Introduction}
\label{sec:orgc2cdbca}
\subsubsection{Fuzzy Sets}
\label{sec:org410a596}
\subsubsection{Intuitionistic Fuzzy Sets}
\label{sec:orgcd38f72}
the membership and nonmembership values of an element with respect to a
collection of elements from a universe may not add up to 1 in all possible
cases   

\begin{definition}[]
An \emph{intuitionistic fuzzy set} \(A\) on a universe \(U\) is defined by two
functions: \emph{membership function} \(\mu_A\) and \emph{non-membership function} \(\nu\) s.t.
\begin{equation*}
\mu_A, \nu_A: U\to [0,1]
\end{equation*}
where \(0\le \mu_A(x)+\nu_A(x)\le 1\) for all \(x\in U\).
\end{definition}

The \emph{hesitation function} \(\Pi_A\) for an intuitionistic fuzzy set is given by
\begin{equation*}
\Pi_A(x)=1-\mu_A(x)-\nu_A(x)
\end{equation*}
\subsubsection{Rough set}
\label{sec:org0f778c9}
A knowledge base is also called an \textbf{approximation space}
\subsubsection{Motivation}
\label{sec:org89145b8}
\subsubsection{Fuzzy proximity relation}
\label{sec:org05982b2}
\begin{definition}[]
Let \(U\) be a universal set and \(X\subseteq U\). Then a \emph{fuzzy relation} on \(X\)
is defined as any fuzzy set defined on \(X\times X\)
\end{definition}
\begin{definition}[]
A fuzzy relation \(R\) is said to be \emph{fuzzy reflexive} on \(X\subseteq U\) if it
satisfies
\begin{equation*}
\mu_R(x,x)=1\quad\text{for all } x
\end{equation*}
\end{definition}
\begin{definition}[]
A fuzzy relation \(R\) is said to be \emph{fuzzy symmetric} on \(X\subseteq U\) if it
satisfies
\begin{equation*}
\mu_R(x,y)=\mu_R(y,x)\quad\text{for all } x,y\in X
\end{equation*}
\end{definition}
\begin{definition}[]
A fuzzy relation on \(X\subseteq U\) is said to be a \emph{fuzzy proximity relation}
if it is fuzzy reflexive and fuzzy symmetric.
\end{definition}
\begin{definition}[]
Let \(X,Y\subseteq U\). A fuzzy relation from \(X\) to \(Y\) is a fuzzy set
defined on \(X\times Y\) characterized by the membership function \(\mu_R:X\times Y\to
    [0,1]\) 
\end{definition}
\begin{definition}[]
For any \(\alpha\in [0,1]\), the \(\alpha\textit{-cut}\) of \(R\), denoted by \(R_\alpha\) is a subset
of \(X\times Y\) given by \(R_\alpha=\lb(x,y):\mu_R(x,y)\ge\alpha\rb\)
\end{definition}
Let \(R\) be a fuzzy proximity relation on \(U\). Then for any \(\alpha\in [0,1]\) the
elements of \(R_\alpha\) are said to be \(\alpha\textit{-similar}\) to each
other. \(xR_\alpha y\). 

Two elements \(x\) and \(y\) in \(U\) are said to be \(\alpha\textit{-identical}\)
w.r.t. \(R\) (\(xR(\alpha)y\)) if either \(x\) and \(y\) are \(\alpha\text{-similar}\) or \(x\)
and \(y\) are \emph{transitively} \(\alpha\textit{-similar}\), that is, there exists a
sequence of elements \(u_1, u_2, \dots,u_n\) in \(U\) s.t. \linebreak
\(xR_\alpha u_1, u_1 R_\alpha u_2
    ,\dots,u_n R_\alpha y\)
\subsubsection{Intuitionistic fuzzy proximity relation}
\label{sec:org5c1cdea}
\begin{definition}[]
An \emph{intuitionistic fuzzy relation} on a universal set \(U\) is an intuitionistic
fuzzy set defined on \(U\times U\)
\end{definition}
\begin{definition}[]
An intuitionistic fuzzy relation \(R\) on a universal set \(U\) is said to be 
\emph{intuitionstic fuzzy reflexive} if 
\begin{equation*}
\mu_R(x,x)=1\text{ and } \nu_R(x,x)=0\quad\text{for all } x\in X
\end{equation*}
\end{definition}
\begin{definition}[]
An intuitionistic fuzzy relation \(R\) on a universal set \(U\) is said to be 
\emph{intuitionistic fuzzy symmetric} if
\begin{equation*}
\mu_R(x,y)=\mu_R(y,x)\text{ and } \nu_R(x,y)=\nu_R(y,x)\quad\text{for all } x,y\in X
\end{equation*}
\end{definition}
\begin{definition}[]
\emph{intuitionistic fuzzy proximity}
\end{definition}
Define
\begin{equation*}
J=\lb(m,n)\mid m,n\in[0,1] \text{ and } 0\le m+n\le 1\rb
\end{equation*}
\begin{definition}[]
Le \(R\) be an IF-proximity relation on \(U\). Then for any \((\alpha,\beta)\in J\) the 
\((\alpha,\beta)\textit{-cut}\) of \(R\), denoted by R\textsubscript{\(\alpha\),\(\beta\)} is 
\begin{equation*}
R_{\alpha,\beta}=\lb(x,y)\mid\mu_R(x,y)\ge\alpha\text{ and } \nu_R(x,y)\le\beta\rb
\end{equation*}
\end{definition}
The relation \(R(\alpha,\beta)\) is an equivalence relation.
\subsection{Rough Sets on Fuzzy Approximation}
\label{sec:org5cdec4e}
\subsubsection{Preliminaries}
\label{sec:org17ac88f}
\begin{definition}[]
For any set of fuzzy proximity relation \(K=(U,\mathfrak{R})\) is called a
\emph{fuzzy approximation space}
\end{definition}

For any fixed \(\alpha\in[0,1]\), \(\fR\) generates a set of equivalence relation
\(\fR(\alpha)\) and we call the associated space \(\bK(\alpha)=(U,\fR(\alpha))\) as the 
\emph{generated approximation space} corresponding to \(\bK\) and \(\alpha\)
\subsubsection{Properties}
\label{sec:orgc20d09e}
\subsubsection{Reduction of Knowledge in Fuzzy Approximation Spaces}
\label{sec:orgb3c8303}
\begin{definition}[]
Let \(\fR\) be a family of fuzzy proximity relations on \(U\) and \(\alpha\in[0,1]\).
For any \(R\in \fR\), we say that \(R\) is \(\alpha\textit{-dispensable}\) or 
\(\alpha\textit{-superfluous}\) in \(\fR\) if and only if 
\(IND(\fR(\alpha))=IND(\fR(\alpha)-R(\alpha))\)
\end{definition}


Consider \(U=\lb x_1,\dots,x_n\rb\). Define the fuzzy proximity relations
\(P,Q,R\) and \(S\) over \(U\) corresponding to the attributes \(a,b,c\) and \(d\)
respectively.

\begin{table}[htbp]
\caption{Fuzzy proximity relation for attribute \(R\)}
\centering
\begin{tabular}{lrrrrr}
\hline
P & x\textsubscript{1} & x\textsubscript{2} & x\textsubscript{3} & x\textsubscript{4} & x\textsubscript{5}\\
\hline
x\textsubscript{1} & 1 & 0.3 & 0.6 & 0.8 & 0.5\\
x\textsubscript{2} & 0.3 & 1 & 0.7 & 0.4 & 0.4\\
x\textsubscript{3} & 0.6 & 0.7 & 1 & 0.2 & 0.8\\
x\textsubscript{4} & 0.8 & 0.4 & 0.2 & 1 & 0.5\\
x\textsubscript{5} & 0.5 & 0.4 & 0.8 & 0.5 & 1\\
\hline
\end{tabular}
\end{table}

\begin{table}[htbp]
\caption{Fuzzy proximity relation for attribute \(Q\)}
\centering
\begin{tabular}{lrrrrr}
\hline
P & x\textsubscript{1} & x\textsubscript{2} & x\textsubscript{3} & x\textsubscript{4} & x\textsubscript{5}\\
\hline
x\textsubscript{1} & 1 & 0.3 & 0.4 & 0.2 & 0.5\\
x\textsubscript{2} & 0.3 & 1 & 0.8 & 0.6 & 0.6\\
x\textsubscript{3} & 0.4 & 0.8 & 1 & 0.3 & 0.9\\
x\textsubscript{4} & 0.2 & 0.6 & 0.3 & 1 & 0.7\\
0.5 & 0.2 & 0.2 & 0.9 & 0.7 & 1\\
\hline
\end{tabular}
\end{table}

\begin{table}[htbp]
\caption{Fuzzy proximity relation for attribute \(R\)}
\centering
\begin{tabular}{lrrrrr}
\hline
R & x\textsubscript{1} & x\textsubscript{2} & x\textsubscript{3} & x\textsubscript{4} & x\textsubscript{5}\\
\hline
x\textsubscript{1} & 1 & 0.3 & 0.2 & 0.8 & 0.7\\
x\textsubscript{2} & 0.3 & 1 & 0.5 & 0.3 & 0.5\\
x\textsubscript{3} & 0.2 & 0.5 & 1 & 0.6 & 0.4\\
x\textsubscript{4} & 0.8 & 0.3 & 0.6 & 1 & 0.9\\
x\textsubscript{5} & 0.7 & 0.5 & 0.4 & 0.9 & 1\\
\hline
\end{tabular}
\end{table}

\begin{table}[htbp]
\caption{Fuzzy proximity relation for attribute \(S\)}
\centering
\begin{tabular}{lrrrrr}
\hline
S & x\textsubscript{1} & x\textsubscript{2} & x\textsubscript{3} & x\textsubscript{4} & x\textsubscript{5}\\
\hline
x\textsubscript{1} & 1 & 0.3 & 0.2 & 0.2 & 0.5\\
x\textsubscript{2} & 0.3 & 1 & 0.5 & 0.3 & 0.2\\
x\textsubscript{3} & 0.2 & 0.5 & 1 & 0.2 & 0.4\\
x\textsubscript{4} & 0.2 & 0.3 & 0.2 & 1 & 0.5\\
x\textsubscript{5} & 0.5 & 0.4 & 0.4 & 0.5 & 1\\
\hline
\end{tabular}
\end{table}

\begin{table}[htbp]
\caption{Fuzzy proximity relation for \(IND(\fR(\alpha))\)}
\centering
\begin{tabular}{crrrrrl}
\hline
\(IND(\fR(\alpha))\) & x\textsubscript{1} & x\textsubscript{2} & x\textsubscript{3} & x\textsubscript{4} & x\textsubscript{5} & \\
\hline
x\textsubscript{1} & 1 & 0.3 & 0.2 & 0.2 & 0.5 & \\
x\textsubscript{2} & 0.3 & 1 & 0.3 & 0.3 & 0.2 & \\
x\textsubscript{3} & 0.2 & 0.3 & 1 & 0.2 & 0.4 & \\
x\textsubscript{4} & 0.2 & 0.3 & 0.2 & 1 & 0.4 & \\
x\textsubscript{5} & 0.5 & 0.2 & 0.4 & 0.4 & 1 & \\
\hline
\end{tabular}
\end{table}

Suppose \(\alpha=0.6\), then we get
\begin{align*}
  &U/P(\alpha)=\{\{x_1,x_2,x_3,x_4,x_5\}\}\\
  &U/Q(\alpha)=\{\{x_1\},\{x_2,x_3,x_4,x_5\}\}\\
  &U/R(\alpha)=\{\{x_1,x_3,x_4,x_5\},\{x_2\}\}\\
  &U/S(\alpha)=\{\{x_1\},\{x_2\},\{x_3\},\{x_4\},\{x_5\}\}\\
\end{align*}
\subsubsection{Relative reducts and relative core of knowledge in fuzzy approximation spaces}
\label{sec:orgc73b688}
\begin{definition}[]
Let \(P\) and \(Q\) be two fuzzy proximity relations over the universe \(U\). For
every fixed \(\alpha\in[0,1]\), the \(\alpha\textit{-positive region}\) of \(P\) w.r.t. \(Q\)
can be defined as
\begin{equation*}
\alpha\text{-}POS_P Q=\displaystyle\bigcup_{X_\alpha\in U/Q} \underline{P}X_\alpha
\end{equation*}
\end{definition}
\begin{definition}[]
Let \(\bP\) and \(\bQ\) be two families of fuzzy proximity relations on \(U\). For
every fixed \(\alpha\in [0,1]\) and \(R\in \bP\) ,\(R\) is
\((\bQ,\alpha)\textit{-dispensable}\) in \(\bP\) if 
\begin{equation*}
\alpha\text{-}POS_{IND} (\bQ)=\alpha\text{-}POS_{IND(\bP-\lb R\rb)}IND(\bQ)
\end{equation*}
If every \(R\in\bP\) is \((\bQ,\alpha)\text{-indispensable}\), then \(\bP\) is 
\((\bQ,\alpha)\textit{-indepedent}\)
\end{definition}
\begin{definition}[]
For every fixed \(\alpha\in[0,1]\), the family \(\bS\subseteq\bP\) is a 
\((\bQ,\alpha)\text{-reduct}\) of \(\bP\) if and only if
\begin{gather*}
\bS\text{ is } (\bQ,\alpha)\text{-indepedent}\\
\alpha\text{-}POS_{\bS}\bQ=\alpha\text{-}POS_{\bP}\bQ
\end{gather*}
\end{definition}

Consider another attribute \(T\), let's find the relative reduct and the
relative core, that is \((T,\alpha)\text{-reduct}\) and
\((T,\alpha)\text{-core}\) of the family of fuzzy proximity relations 
\(\calr=\lb P,Q,R,S\rb\)
\begin{table}[htbp]
\caption{Fuzzy proximity relation for attribute \(T\)}
\centering
\begin{tabular}{lrrrrr}
\hline
T & x\textsubscript{1} & x\textsubscript{2} & x\textsubscript{3} & x\textsubscript{4} & x\textsubscript{5}\\
\hline
x\textsubscript{1} & 1 & 0.4 & 0.5 & 0.6 & 0.8\\
x\textsubscript{2} & 0.4 & 1 & 0.9 & 0.3 & 0.5\\
x\textsubscript{3} & 0.5 & 0.9 & 1 & 0.2 & 0.4\\
x\textsubscript{4} & 0.6 & 0.3 & 0.2 & 1 & 0.5\\
x\textsubscript{5} & 0.8 & 0.5 & 0.4 & 0.5 & 1\\
\hline
\end{tabular}
\end{table}
\begin{gather*}
  U/T(\alpha)=\{\{x_1,x_4,x_5\},\{x_2,x_3\}\}\\
  U/IND\calr(\alpha)=\{\{x_1\},\{x_2\},\{x_3\},\{x_4\},\{x_5\}\}\\
  \alpha\text{-}POS_{\calr(\alpha)}T(\alpha)=
  \displaystyle\bigcup_{X\in U/T(\alpha)}\calr X_\alpha=U
\end{gather*}
\subsubsection{Dependency of knowledge in fuzzy approximation spaces}
\label{sec:orgfaa58c6}
\begin{definition}[]
We say knowledge \(\bQ\) is \(\alpha\textit{-derivable}\) from knowledge \(\bP\) if
the elementary \(\alpha\text{-categories}\) of \(\bQ\) can be defined in terms
of some elementary \(\alpha\text{-categories}\) of knowledge P
\end{definition}
\begin{definition}[]
If \(\bQ\) is \(\alpha\text{-derivable}\) from \(\bP\), we say that knowledge \(\bQ\)
\(\alpha\textit{-depends}\) on knowledge \(\bP\) and we denote it by
\(\bP\xRightarrow{\alpha}\bQ\). So, \(\bP\xRightarrow{\alpha}\bQ\) if and only
if \(IND(\bP(\alpha))\subseteq IND(\bQ(\alpha))\)
\end{definition}
\begin{definition}[]
Knowledge \(\bP\) and \(\bQ\) are \(\alpha\textit{-equivalent}\), denoted by 
\(\bP\stackrel{\alpha}{\equiv}\bQ\) iff \(IND(\bP(\alpha))=IND(\bQ(\alpha))\)
\end{definition}

\subsubsection{partial dependency of knowledge in fuzzy approximation spaces}
\label{sec:org90981a0}
It may happen that the \(\alpha\text{-derivation}\) of one knowledge \(P\) from
another knowledge \(Q\) can be partial. That is only a part of knowledge P can
be \(\alpha\text{-derivation}\) from P 

Suppose \(\calk(\alpha)=(U,\calr(\alpha))\) is a fuzzy approximation space and
\(P,Q\subseteq \calr\). Then \(Q\) \(\alpha\textit{-depends}\) \emph{in a degree}
\(k(\alpha)\), \(0\le k(\alpha)\le 1\) denoted by \(P\xRightarrow{\alpha}_{k}Q\)
if 
\begin{equation*}
k(\alpha)=\gamma_{P(\alpha)}(Q(\alpha))=\frac{\abs{POS_{P(\alpha)}(Q(\alpha))}}{\abs{U}}
\end{equation*}
\(k(\alpha)=1\) then \emph{totally}, \(0<k(\alpha)<1\) \emph{partially(roughly)}

\subsection{Rough Sets in Intuitionistic Fuzzy Approximation Spaces}
\label{sec:orgd782ddf}
\subsubsection{knowledge reduction in IF-Approximation spaces}
\label{sec:orgd76166b}
\section{Granular structures and approximations in rough sets and knowledge spaces}
\label{sec:orgd0d8637}
\subsection{Introduction}
\label{sec:org1343657}
Granular computing is an emerging field of study focusing on structured
thinking, structured problem solving and structured information processing
with multiple levels of granularity 

A primitive notion of granular computing is that of granules. Granules may
be considered as parts of a whole. A granule may be understood as a unit
that we use for describing and representing a problem or a focal point of
our attention at a specific point of time. Granules can be organized based
on their inherent properties and interrelationships. The results are a
multilevel granular structure. Each level is populated by granules of the
similar size or the similar nature. Depending on a particular context,
levels of granularity may be interpreted as levels of abstraction, levels of
details, levels of processing, levels of understanding, levels of
interpretation, levels of control, and many more. An ordering of levels
based on granularity provides a hierarchical granular structure. 
\subsection{Granular spaces}
\label{sec:org88189cd}
\subsubsection{A Set-theoretic interpretation of granules}
\label{sec:org09fce63}
Categorization or classification is one of the fundamental tasks of human
intelligence.

The process of categorization covers two important issues of granulation,
namely, the construction of granules and the naming of granules. Objects in the
same categories must be more similar to each other, and objects in different
granules are more dissimilar to each other.
\subsubsection{A formulation of granules as concept}
\label{sec:org275249d}
The classical view of concepts defines a concept jointly by a set of
objects, called the extension of the concept, and a set of intrinsic
properties common to the set of objects, called the intension of the
concept. Typically, the name of a concept reflects the intension of a
concept. The extension of a concept is the set of objects which are concrete
examples of a concept. One may introduce a logic language so that the
intension of a concept is represented by a formula and the extension is
represented by the set of objects satisfying the formula.

For an individual \(x\in U\), if it satisfies an atomic formula \(p\), we write
\(x\models p\). An individual satisfies a formula if the individual has the
properties as specified by the formula.

If \(\phi\) is a formula, the set \(m(\phi)\) defined by
\begin{equation*}
m(\phi)=\lb x\in U\mid x\models \phi\rb
\end{equation*}
is called the \emph{meaning} of the formula \(\phi\). In other words, \(\phi\) can be
viewed as the description of the set of object \(m(\phi)\). As a result, a
concept can be expressed by a pair \((\phi,m(\phi))\) where \(\phi\in\call\).
\(\phi\) is the intension of a concept while \(m(\phi)\) is the extension of a concept.
\subsubsection{Granular spaces and granular structures}
\label{sec:orga9e2159}
Each atomic formula in \(\cala\) is associated with a subset of \(U\). This
subset may be viewed as an elementary granule in \(U\). Each formula is
obtained by taking logic operations on atomic formulas. The meaning set of
the formula can be obtained from the elementary granules through
set-theoretic operations

A subset or a granule \(X\subseteq U\) is \emph{definable} if and only if there
exists a formula \(\phi\) in the language \(\call\) s.t.
\begin{equation*}
X=m(\phi)
\end{equation*}
Family of all definable granules is given by
\begin{equation*}
Def(\call(\cala,\lb\neg,\wedge,\vee\rb,U))=\lb m(\phi)\mid\phi\in\call
(\cala,\lb\neg,\wedge,\vee\rb,U)\rb
\end{equation*}
which is a subsystem of the power set \(2^U\) closed under set complement,
intersection and union. A \emph{granular space}
\begin{equation*}
(U,\cals_0,\cals)
\end{equation*}
where \(U\) is the universe, \(\cals_0\subseteq 2^U\) is a family of elementary
granules, i.e., \(\cals_0=\lb m(p)\mid p\in\cala\rb\), \(\cals\subseteq 2^U\) is
a family of definbale granules

\(\cals\) is an \(\sigma\text{-algebra}\)
\subsection{Rough Set Analysis}
\label{sec:orge2af3cf}
\subsubsection{Granular spaces and granular structures}
\label{sec:orgbe971db}
Information table
\begin{equation*}
M=(U,At,\lb V_a\mid a\in At\rb,\lb I_a\mid a\in At\rb)
\end{equation*}
where \(I_a:U\to V_a\) is an information function.

For a set of attributes \(P\subseteq At\), we can define an equivalence
relation on the set of objects
\begin{equation*}
xE_py\Longleftrightarrow\forall a\in P(I_a(x)=I_a(y))
\end{equation*}

By taking the union of a family of equivalence classes, we can obtain a
composite granule. The family of all such granules contains the entire set \(U\)
and the empty set \(\emptyset\), and is closed under set complement,
intersection and 
union. More specifically, the family is an \(\sigma\)-algebra, denoted by \(\sigma(U/E)\),
with the basis \(U/E\)

For an attribute-value pair \((a,v)\) ,where \(a\in At, v\in V_a\), we have an
atomic formula \(a=v\). The meaning of \(a=v\) is
\begin{equation*}
m(a=v)=\lb x\in U\mid I_a(x)=v\rb
\end{equation*}
Hence \([x]=\bigvee_{a\in At}a=I_a(x)\) and is a definable granule.
\subsubsection{Rough Set Approximation}
\label{sec:org4eb4cbd}
\begin{align*}
  &\underline{apr}(A)=\bigcup\lb X\in\sigma(U/E)\mid X\subseteq A\rb\\
  &\overline{apr}(A)=\bigcap\lb X\in\sigma(U/E)\mid A\subseteq X\rb\\
\end{align*}
Hence \((\underline{apr}(A),\overline{apr}(A))\) is the tightest approximation.
\subsection{Knowledge space theory}
\label{sec:org855ece0}
In knowledge spaces, we consider a pair \((Q,\calk)\) where \(Q\) is a finite set
of questions and \(\calk\subseteq 2^Q\). Each \(K\subseteq\calk\) is called a
\emph{knowledge state} and \(\calk\) is the set of all possible knowledge state.


Intuitively, the knowledge state of an individual is represented by the set
of questions that he is capable of answering. Each knowledge state can be
considered as a granule. The collection of all the knowledge states together
with the empty set ∅ and the whole set \(Q\) is called a knowledge structure, and
may be viewed as a granular knowledge structure in the terminology of
granular computing 
\subsubsection{Granular spaces associated to surmise relations}
\label{sec:org38b14a1}
A surmise relation on the set \(Q\) of questions is a reflexive and transitive
relation \(S\). By \(aSb\), we can surmise the mastery of \(a\) if a student can
correctly question \(b\). For example, \(aSb\) means that if a knowledge state
contains \(b\), it must also contain \(a\).

Formally, for a surmise relation \(S\) on the finite set \(Q\) of questions, the
associated knowledge structure \(\calk\) is defined by
\begin{equation*}
\calk=\lb K\mid\forall q,q'\in Q((qSq',q'\in K)\Rightarrow q\in K)\rb
\end{equation*}

For each question \(q\) in \(Q\), under a surmise relation, we can find a unique
prerequistie question set \(R_p(q)=\lb q'\mid q'Sq\rb\). The family of the
prerequistie question sets for all the questions is denoted by \(\calb\). By taking
the union of prerequisite sets for a family of questions, we can obtain a
knowledge structure \(\calk\) associated to the surmise relation \(S\). It
defines a granular space \((Q,\calb,\calk)\). All knowledge states are called
granules in \((Q,\calb,\calk)\). 
\section{On Approximation of classifications, Rough Equalities and Rough Equivalences}
\label{sec:org6e2b2ef}
\section{A generic scheme for generating prediction rules using rough sets}
\label{sec:org25b2212}
\subsection{Rough set prediction model}
\label{sec:org0d59467}
\begin{enumerate}
\item \emph{pre-processing phase}
\item \emph{analysis and rule generating phase}
\item \emph{classification and prediction phase}
\end{enumerate}
\subsubsection{Pre-processing phase}
\label{sec:org8720920}
\begin{enumerate}
\item Data completion
\label{sec:org0ebf62d}
\item Data discretization
\label{sec:org028a38e}

When dealing with attributes in concept classification and prediction, it
is obvious that they may have varying importance in the problem being
considered. Their importance can be pre-assumed using auxiliary knowledge
about the problem and expressed by properly chosen weights. However, in the
case of using the rough set approach to concept classification and
prediction, it avoids any additional information aside from what is
included in the information table itself. Basically, the rough set approach
tries to determine from the data available in the information table whether
all the attributes are of the same strength and, if not, how they differ in
respect of the classifier power.
\end{enumerate}
\subsubsection{Classification and prediction phase}
\label{sec:orgaaf6de4}
\end{document}