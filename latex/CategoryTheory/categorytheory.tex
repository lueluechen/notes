% Created 2021-04-22 四 16:36
% Intended LaTeX compiler: pdflatex
\documentclass[11pt]{article}
\usepackage[utf8]{inputenc}
\usepackage[T1]{fontenc}
\usepackage{graphicx}
\usepackage{grffile}
\usepackage{longtable}
\usepackage{wrapfig}
\usepackage{rotating}
\usepackage[normalem]{ulem}
\usepackage{amsmath}
\usepackage{textcomp}
\usepackage{amssymb}
\usepackage{capt-of}
\usepackage{hyperref}
%%%%%%%%%%%%%%%%%%%%%%%%%%%%%%%%%%%%%%
%% TIPS                                 %%
%%%%%%%%%%%%%%%%%%%%%%%%%%%%%%%%%%%%%%
% \substack{a\\b} for multiple lines text

\usepackage[utf8]{inputenc}

\usepackage[B1,T1]{fontenc}

% pdfplots will load xolor automatically without option
\usepackage[dvipsnames]{xcolor}
%%%%%%%%%%%%%%%%%%%%%%%%%%%%%%%%%%%%%%%
%% MATH related pacakge                  %%
%%%%%%%%%%%%%%%%%%%%%%%%%%%%%%%%%%%%%%%
% \usepackage{amsmath} mathtools loads the amsmath
\usepackage{amsmath}
\usepackage{mathtools}


\usepackage{amsthm}
\usepackage{amsbsy}

%\usepackage{commath}

\usepackage{amssymb}
\usepackage{mathrsfs}
%\usepackage{mathabx}
\usepackage{stmaryrd}
\usepackage{empheq}

%for \not\ll
\usepackage{centernot}

\usepackage{scalerel}
\usepackage{stackengine}
\usepackage{stackrel}

\usepackage{nicematrix}
\usepackage{tensor}
\usepackage{blkarray}
\usepackage{siunitx}
\usepackage[f]{esvect}

\usepackage{unicode-math}
\setmainfont{TeX Gyre Pagella}
% \setmathfont{STIX}
%\setmathfont{texgyrepagella-math.otf}
%\setmathfont{Libertinus Math}
\setmathfont{Latin Modern Math}

 
% \setmathfont[range={\smwhtdiamond,\enclosediamond,\varlrtriangle}]{Latin Modern Math}
 \setmathfont[range={\rightrightarrows,\twoheadrightarrow,\leftrightsquigarrow,\triangledown,\vartriangle}]{XITS Math}
 \setmathfont[range={\int,\setminus}]{Libertinus Math}
 \setmathfont[range={\mathalpha}]{TeX Gyre Pagella Math}
% unicode is not good at this!
%\let\nmodels\nvDash


%%%%%%%%%%%%%%%%%%%%%%%%%%%%%%%%%%%%%%%
%% TIKZ related packages                 %%
%%%%%%%%%%%%%%%%%%%%%%%%%%%%%%%%%%%%%%%

\usepackage{pgfplots}
\pgfplotsset{compat=1.15}
\usepackage{tikz}
\usepackage{tikz-cd}
\usepackage{tikz-qtree}

\usetikzlibrary{arrows,positioning,calc,fadings,decorations,matrix,decorations,shapes.misc}
%setting from geogebra
\definecolor{ccqqqq}{rgb}{0.8,0,0}


%%%%%%%%%%%%%%%%%%%%%%%%%%%%%%%%%%%%%%%
%% MISCLELLANEOUS packages               %%
%%%%%%%%%%%%%%%%%%%%%%%%%%%%%%%%%%%%%%%
\usepackage[most]{tcolorbox}
\usepackage{threeparttable}
\usepackage{tabularx}

\usepackage{enumitem}

% wrong with preview
\usepackage{subcaption}
\usepackage{caption}
% {\aunclfamily\Huge}
\usepackage{auncial}

\usepackage{float}

\usepackage{fancyhdr}

\usepackage{ifthen}
\usepackage{xargs}


\usepackage{imakeidx}
\usepackage{hyperref}
\usepackage{soul}


%\usepackage[xetex]{preview}
%%%%%%%%%%%%%%%%%%%%%%%%%%%%%%%%%%%%%%%
%% USEPACKAGES end                       %%
%%%%%%%%%%%%%%%%%%%%%%%%%%%%%%%%%%%%%%%

% \setlist{nosep}
% \numberwithin{equation}{subsection}
% \fancyhead{} % Clear the headers
% \renewcommand{\headrulewidth}{0pt} % Width of line at top of page
% \fancyhead[R]{\slshape\leftmark} % Mark right [R] of page with Chapter name [\leftmark]
% \pagestyle{fancy} % Set default style for all content pages (not TOC, etc)


% \newlength\shlength
% \newcommand\vect[2][0]{\setlength\shlength{#1pt}%
%   \stackengine{-5.6pt}{$#2$}{\smash{$\kern\shlength%
%     \stackengine{7.55pt}{$\mathchar"017E$}%
%       {\rule{\widthof{$#2$}}{.57pt}\kern.4pt}{O}{r}{F}{F}{L}\kern-\shlength$}}%
%       {O}{c}{F}{T}{S}}


\indexsetup{othercode=\small}
\makeindex[columns=2,options={-s /media/wu/file/stuuudy/notes/index_style.ist},intoc]
\makeatletter
\def\@idxitem{\par\hangindent 0pt}
\makeatother


%\newcounter{dummy} \numberwithin{dummy}{section}
\newtheorem{dummy}{dummy}[section]
\theoremstyle{definition}
\newtheorem{definition}[dummy]{Definition}
\theoremstyle{plain}
\newtheorem{corollary}[dummy]{Corollary}
\newtheorem{lemma}[dummy]{Lemma}
\newtheorem{proposition}[dummy]{Proposition}
\newtheorem{theorem}[dummy]{Theorem}
\theoremstyle{definition}
\newtheorem{examplle}{Example}[section]
\theoremstyle{remark}
\newtheorem*{remark}{Remark}
\newtheorem{exercise}{Exercise}[subsection]
\newtheorem{observation}{Observation}[section]


\newenvironment{claim}[1]{\par\noindent\textbf{Claim:}\space#1}{}

\makeatletter
\DeclareFontFamily{U}{tipa}{}
\DeclareFontShape{U}{tipa}{m}{n}{<->tipa10}{}
\newcommand{\arc@char}{{\usefont{U}{tipa}{m}{n}\symbol{62}}}%

\newcommand{\arc}[1]{\mathpalette\arc@arc{#1}}

\newcommand{\arc@arc}[2]{%
  \sbox0{$\m@th#1#2$}%
  \vbox{
    \hbox{\resizebox{\wd0}{\height}{\arc@char}}
    \nointerlineskip
    \box0
  }%
}
\makeatother

\setcounter{MaxMatrixCols}{20}
%%%%%%% ABS
\DeclarePairedDelimiter\abss{\lvert}{\rvert}%
\DeclarePairedDelimiter\normm{\lVert}{\rVert}%

% Swap the definition of \abs* and \norm*, so that \abs
% and \norm resizes the size of the brackets, and the
% starred version does not.
\makeatletter
\let\oldabs\abss
%\def\abs{\@ifstar{\oldabs}{\oldabs*}}
\newcommand{\abs}{\@ifstar{\oldabs}{\oldabs*}}
\newcommand{\norm}[1]{\left\lVert#1\right\rVert}
%\let\oldnorm\normm
%\def\norm{\@ifstar{\oldnorm}{\oldnorm*}}
%\renewcommand{norm}{\@ifstar{\oldnorm}{\oldnorm*}}
\makeatother

% \newcommand\what[1]{\ThisStyle{%
%     \setbox0=\hbox{$\SavedStyle#1$}%
%     \stackengine{-1.0\ht0+.5pt}{$\SavedStyle#1$}{%
%       \stretchto{\scaleto{\SavedStyle\mkern.15mu\char'136}{2.6\wd0}}{1.4\ht0}%
%     }{O}{c}{F}{T}{S}%
%   }
% }

% \newcommand\wtilde[1]{\ThisStyle{%
%     \setbox0=\hbox{$\SavedStyle#1$}%
%     \stackengine{-.1\LMpt}{$\SavedStyle#1$}{%
%       \stretchto{\scaleto{\SavedStyle\mkern.2mu\AC}{.5150\wd0}}{.6\ht0}%
%     }{O}{c}{F}{T}{S}%
%   }
% }

% \newcommand\wbar[1]{\ThisStyle{%
%     \setbox0=\hbox{$\SavedStyle#1$}%
%     \stackengine{.5pt+\LMpt}{$\SavedStyle#1$}{%
%       \rule{\wd0}{\dimexpr.3\LMpt+.3pt}%
%     }{O}{c}{F}{T}{S}%
%   }
% }

\newcommand{\bl}[1] {\boldsymbol{#1}}
\newcommand{\Wt}[1] {\stackrel{\sim}{\smash{#1}\rule{0pt}{1.1ex}}}
\newcommand{\wt}[1] {\widetilde{#1}}
\newcommand{\tf}[1] {\textbf{#1}}


%For boxed texts in align, use Aboxed{}
%otherwise use boxed{}

\DeclareMathSymbol{\widehatsym}{\mathord}{largesymbols}{"62}
\newcommand\lowerwidehatsym{%
  \text{\smash{\raisebox{-1.3ex}{%
    $\widehatsym$}}}}
\newcommand\fixwidehat[1]{%
  \mathchoice
    {\accentset{\displaystyle\lowerwidehatsym}{#1}}
    {\accentset{\textstyle\lowerwidehatsym}{#1}}
    {\accentset{\scriptstyle\lowerwidehatsym}{#1}}
    {\accentset{\scriptscriptstyle\lowerwidehatsym}{#1}}
  }


\newcommand{\cupdot}{\mathbin{\dot{\cup}}}
\newcommand{\bigcupdot}{\mathop{\dot{\bigcup}}}

\usepackage{graphicx}

\usepackage[toc,page]{appendix}

% text on arrow for xRightarrow
\makeatletter
%\newcommand{\xRightarrow}[2][]{\ext@arrow 0359\Rightarrowfill@{#1}{#2}}
\makeatother

% Arbitrary long arrow
\newcommand{\Rarrow}[1]{%
\parbox{#1}{\tikz{\draw[->](0,0)--(#1,0);}}
}

\newcommand{\LRarrow}[1]{%
\parbox{#1}{\tikz{\draw[<->](0,0)--(#1,0);}}
}


\makeatletter
\providecommand*{\rmodels}{%
  \mathrel{%
    \mathpalette\@rmodels\models
  }%
}
\newcommand*{\@rmodels}[2]{%
  \reflectbox{$\m@th#1#2$}%
}
\makeatother







\newcommand{\trcl}[1]{%
  \mathrm{trcl}{(#1)}
}



% Roman numerals
\makeatletter
\newcommand*{\rom}[1]{\expandafter\@slowromancap\romannumeral #1@}
\makeatother
% \\def \\b\([a-zA-Z]\) {\\boldsymbol{[a-zA-z]}}
% \\DeclareMathOperator{\\b\1}{\\textbf{\1}}


\DeclareMathOperator{\bx}{\textbf{x}}
\DeclareMathOperator{\bz}{\textbf{z}}
\DeclareMathOperator{\bff}{\textbf{f}}
\DeclareMathOperator{\ba}{\textbf{a}}
\DeclareMathOperator{\bk}{\textbf{k}}
\DeclareMathOperator{\bs}{\textbf{s}}
\DeclareMathOperator{\bh}{\textbf{h}}
\DeclareMathOperator{\bc}{\textbf{c}}
\DeclareMathOperator{\br}{\textbf{r}}
\DeclareMathOperator{\bi}{\textbf{i}}
\DeclareMathOperator{\bj}{\textbf{j}}
\DeclareMathOperator{\bn}{\textbf{n}}
\DeclareMathOperator{\be}{\textbf{e}}
\DeclareMathOperator{\bo}{\textbf{o}}
\DeclareMathOperator{\bU}{\textbf{U}}
\DeclareMathOperator{\bL}{\textbf{L}}
\DeclareMathOperator{\bV}{\textbf{V}}
\def \bzero {\mathbf{0}}
\def \btwo {\mathbf{2}}
\DeclareMathOperator{\bv}{\textbf{v}}
\DeclareMathOperator{\bp}{\textbf{p}}
\DeclareMathOperator{\bI}{\textbf{I}}
\DeclareMathOperator{\bM}{\textbf{M}}
\DeclareMathOperator{\bN}{\textbf{N}}
\DeclareMathOperator{\bK}{\textbf{K}}
\DeclareMathOperator{\bt}{\textbf{t}}
\DeclareMathOperator{\bb}{\textbf{b}}
\DeclareMathOperator{\bA}{\textbf{A}}
\DeclareMathOperator{\bX}{\textbf{X}}
\DeclareMathOperator{\bu}{\textbf{u}}
\DeclareMathOperator{\bS}{\textbf{S}}
\DeclareMathOperator{\bZ}{\textbf{Z}}
\DeclareMathOperator{\bJ}{\textbf{J}}
\DeclareMathOperator{\by}{\textbf{y}}
\DeclareMathOperator{\bw}{\textbf{w}}
\DeclareMathOperator{\bT}{\textbf{T}}
\DeclareMathOperator{\bF}{\textbf{F}}
\DeclareMathOperator{\bmm}{\textbf{m}}
\DeclareMathOperator{\bW}{\textbf{W}}
\DeclareMathOperator{\bR}{\textbf{R}}
\DeclareMathOperator{\bC}{\textbf{C}}
\DeclareMathOperator{\bD}{\textbf{D}}
\DeclareMathOperator{\bE}{\textbf{E}}
\DeclareMathOperator{\bQ}{\textbf{Q}}
\DeclareMathOperator{\bP}{\textbf{P}}
\DeclareMathOperator{\bY}{\textbf{Y}}
\DeclareMathOperator{\bH}{\textbf{H}}
\DeclareMathOperator{\bB}{\textbf{B}}
\DeclareMathOperator{\bG}{\textbf{G}}
\def \blambda {\symbf{\lambda}}
\def \boldeta {\symbf{\eta}}
\def \balpha {\symbf{\alpha}}
\def \bbeta {\symbf{\beta}}
\def \bgamma {\symbf{\gamma}}
\def \bxi {\symbf{\xi}}
\def \bLambda {\symbf{\Lambda}}
\def \bGamma {\symbf{\Gamma}}

\newcommand{\bto}{{\boldsymbol{\to}}}
\newcommand{\Ra}{\Rightarrow}
\newcommand\und[1]{\underline{#1}}
\newcommand\ove[1]{\overline{#1}}
\def \bPhi {\boldsymbol{\Phi}}
\def \btheta {\boldsymbol{\theta}}
\def \bTheta {\boldsymbol{\Theta}}
\def \bmu {\boldsymbol{\mu}}
\def \bphi {\boldsymbol{\phi}}
\def \bSigma {\boldsymbol{\Sigma}}
\def \lb {\left\{}
\def \rb {\right\}}
\def \la {\langle}
\def \ra {\rangle}
\def \caln {\mathcal{N}}
\def \dissum {\displaystyle\Sigma}
\def \dispro {\displaystyle\prod}
\def \E {\mathbb{E}}
\def \Q {\mathbb{Q}}
\def \N {\mathbb{N}}
\def \V {\mathbb{V}}
\def \R {\mathbb{R}}
\def \P {\mathbb{P}}
\def \A {\mathbb{A}}
\def \F {\mathbb{F}}
\def \Z {\mathbb{Z}}
\def \I {\mathbb{I}}
\def \C {\mathbb{C}}
\def \cala {\mathcal{A}}
\def \cale {\mathcal{E}}
\def \calb {\mathcal{B}}
\def \calq {\mathcal{Q}}
\def \calp {\mathcal{P}}
\def \cals {\mathcal{S}}
\def \calx {\mathcal{X}}
\def \caly {\mathcal{Y}}
\def \calg {\mathcal{G}}
\def \cald {\mathcal{D}}
\def \caln {\mathcal{N}}
\def \calr {\mathcal{R}}
\def \calt {\mathcal{T}}
\def \calm {\mathcal{M}}
\def \calw {\mathcal{W}}
\def \calc {\mathcal{C}}
\def \calv {\mathcal{V}}
\def \calf {\mathcal{F}}
\def \calk {\mathcal{K}}
\def \call {\mathcal{L}}
\def \calu {\mathcal{U}}
\def \calo {\mathcal{O}}
\def \calh {\mathcal{H}}
\def \cali {\mathcal{I}}

\def \bcup {\bigcup}

% set theory

\def \zfcc {\textbf{ZFC}^-}
\def \ac  {\textbf{AC}}
\def \gl  {\textbf{L }}
\def \gll {\textbf{L}}
\newcommand{\zfm}{$\textbf{ZF}^-$}

%\def \zfm {$\textbf{ZF}^-$}
\def \zfmm {\textbf{ZF}^-}
\def \wf {\textbf{WF }}
\def \on {\textbf{On }}
\def \cm {\textbf{M }}
\def \cn {\textbf{N }}
\def \cv {\textbf{V }}
\def \zc {\textbf{ZC }}
\def \zcm {\textbf{ZC}}
\def \zff {\textbf{ZF}}
\def \wfm {\textbf{WF}}
\def \onm {\textbf{On}}
\def \cmm {\textbf{M}}
\def \cnm {\textbf{N}}
\def \cvm {\textbf{V}}
\def \gchh {\textbf{GCH}}
\renewcommand{\restriction}{\mathord{\upharpoonright}}
\def \pred {\text{pred}}

\def \rank {\text{rank}}
\def \con {\text{Con}}
\def \deff {\text{Def}}


\def \uin {\underline{\in}}
\def \oin {\overline{\in}}
\def \uR {\underline{R}}
\def \oR {\overline{R}}
\def \uP {\underline{P}}
\def \oP {\overline{P}}

\def \dsum {\displaystyle\sum}

\def \Ra {\Rightarrow}

\def \e {\enspace}

\def \sgn {\operatorname{sgn}}
\def \gen {\operatorname{gen}}
\def \Hom {\operatorname{Hom}}
\def \hom {\operatorname{hom}}
\def \Sub {\operatorname{Sub}}

\def \supp {\operatorname{supp}}

\def \epiarrow {\twoheadarrow}
\def \monoarrow {\rightarrowtail}
\def \rrarrow {\rightrightarrows}

% \def \minus {\text{-}}
% \newcommand{\minus}{\scalebox{0.75}[1.0]{$-$}}
% \DeclareUnicodeCharacter{002D}{\minus}


\def \tril {\triangleleft}

\def \ACF {\text{ACF}}
\def \GL {\text{GL}}
\def \PGL {\text{PGL}}
\def \equal {=}
\def \deg {\text{deg}}
\def \degree {\text{degree}}
\def \app {\text{App}}
\def \FV {\text{FV}}
\def \conv {\text{conv}}
\def \cont {\text{cont}}
\DeclareMathOperator{\cl}{\textbf{CL}}
\DeclareMathOperator{\sg}{sg}
\DeclareMathOperator{\trdeg}{trdeg}
\def \Ord {\text{Ord}}

\DeclareMathOperator{\cf}{cf}
\DeclareMathOperator{\zfc}{ZFC}

%\DeclareMathOperator{\Th}{Th}
%\def \th {\text{Th}}
% \newcommand{\th}{\text{Th}}
\DeclareMathOperator{\type}{type}
\DeclareMathOperator{\zf}{\textbf{ZF}}
\def \fa {\mathfrak{a}}
\def \fb {\mathfrak{b}}
\def \fc {\mathfrak{c}}
\def \fd {\mathfrak{d}}
\def \fe {\mathfrak{e}}
\def \ff {\mathfrak{f}}
\def \fg {\mathfrak{g}}
\def \fh {\mathfrak{h}}
%\def \fi {\mathfrak{i}}
\def \fj {\mathfrak{j}}
\def \fk {\mathfrak{k}}
\def \fl {\mathfrak{l}}
\def \fm {\mathfrak{m}}
\def \fn {\mathfrak{n}}
\def \fo {\mathfrak{o}}
\def \fp {\mathfrak{p}}
\def \fq {\mathfrak{q}}
\def \fr {\mathfrak{r}}
\def \fs {\mathfrak{s}}
\def \ft {\mathfrak{t}}
\def \fu {\mathfrak{u}}
\def \fv {\mathfrak{v}}
\def \fw {\mathfrak{w}}
\def \fx {\mathfrak{x}}
\def \fy {\mathfrak{y}}
\def \fz {\mathfrak{z}}
\def \fA {\mathfrak{A}}
\def \fB {\mathfrak{B}}
\def \fC {\mathfrak{C}}
\def \fD {\mathfrak{D}}
\def \fE {\mathfrak{E}}
\def \fF {\mathfrak{F}}
\def \fG {\mathfrak{G}}
\def \fH {\mathfrak{H}}
\def \fI {\mathfrak{I}}
\def \fJ {\mathfrak{J}}
\def \fK {\mathfrak{K}}
\def \fL {\mathfrak{L}}
\def \fM {\mathfrak{M}}
\def \fN {\mathfrak{N}}
\def \fO {\mathfrak{O}}
\def \fP {\mathfrak{P}}
\def \fQ {\mathfrak{Q}}
\def \fR {\mathfrak{R}}
\def \fS {\mathfrak{S}}
\def \fT {\mathfrak{T}}
\def \fU {\mathfrak{U}}
\def \fV {\mathfrak{V}}
\def \fW {\mathfrak{W}}
\def \fX {\mathfrak{X}}
\def \fY {\mathfrak{Y}}
\def \fZ {\mathfrak{Z}}

\def \sfA {\textsf{A}}
\def \sfB {\textsf{B}}
\def \sfC {\textsf{C}}
\def \sfD {\textsf{D}}
\def \sfE {\textsf{E}}
\def \sfF {\textsf{F}}
\def \sfG {\textsf{G}}
\def \sfH {\textsf{H}}
\def \sfI {\textsf{I}}
\def \sfj {\textsf{J}}
\def \sfK {\textsf{K}}
\def \sfL {\textsf{L}}
\def \sfM {\textsf{M}}
\def \sfN {\textsf{N}}
\def \sfO {\textsf{O}}
\def \sfP {\textsf{P}}
\def \sfQ {\textsf{Q}}
\def \sfR {\textsf{R}}
\def \sfS {\textsf{S}}
\def \sfT {\textsf{T}}
\def \sfU {\textsf{U}}
\def \sfV {\textsf{V}}
\def \sfW {\textsf{W}}
\def \sfX {\textsf{X}}
\def \sfY {\textsf{Y}}
\def \sfZ {\textsf{Z}}
\def \sfa {\textsf{a}}
\def \sfb {\textsf{b}}
\def \sfc {\textsf{c}}
\def \sfd {\textsf{d}}
\def \sfe {\textsf{e}}
\def \sff {\textsf{f}}
\def \sfg {\textsf{g}}
\def \sfh {\textsf{h}}
\def \sfi {\textsf{i}}
\def \sfj {\textsf{j}}
\def \sfk {\textsf{k}}
\def \sfl {\textsf{l}}
\def \sfm {\textsf{m}}
\def \sfn {\textsf{n}}
\def \sfo {\textsf{o}}
\def \sfp {\textsf{p}}
\def \sfq {\textsf{q}}
\def \sfr {\textsf{r}}
\def \sfs {\textsf{s}}
\def \sft {\textsf{t}}
\def \sfu {\textsf{u}}
\def \sfv {\textsf{v}}
\def \sfw {\textsf{w}}
\def \sfx {\textsf{x}}
\def \sfy {\textsf{y}}
\def \sfz {\textsf{z}}



%\DeclareMathOperator{\ker}{ker}
\DeclareMathOperator{\im}{im}

\DeclareMathOperator{\inn}{Inn}
\DeclareMathOperator{\AC}{\textbf{AC}}
\DeclareMathOperator{\cod}{cod}
\DeclareMathOperator{\dom}{dom}
\DeclareMathOperator{\ran}{ran}
\DeclareMathOperator{\textd}{d}
\DeclareMathOperator{\td}{d}
\DeclareMathOperator{\id}{id}
\DeclareMathOperator{\LT}{LT}
\DeclareMathOperator{\Mat}{Mat}
\DeclareMathOperator{\Eq}{Eq}
\DeclareMathOperator{\irr}{irr}
\DeclareMathOperator{\Fr}{Fr}
\DeclareMathOperator{\Gal}{Gal}
\DeclareMathOperator{\lcm}{lcm}
\DeclareMathOperator{\alg}{\text{alg}}
\DeclareMathOperator{\Th}{Th}

\DeclareMathOperator{\DAG}{DAG}
\DeclareMathOperator{\ODAG}{ODAG}

% \varprod
\DeclareSymbolFont{largesymbolsA}{U}{txexa}{m}{n}
\DeclareMathSymbol{\varprod}{\mathop}{largesymbolsA}{16}
% \DeclareMathSymbol{\tonm}{\boldsymbol{\to}\textbf{Nm}}
\def \tonm {\bto\textbf{Nm}}
\def \tohm {\bto\textbf{Hm}}

% Category theory
\DeclareMathOperator{\Ab}{\textbf{Ab}}
\DeclareMathOperator{\Alg}{\textbf{Alg}}
\DeclareMathOperator{\Rng}{\textbf{Rng}}
\DeclareMathOperator{\Sets}{\textbf{Sets}}
\DeclareMathOperator{\Met}{\textbf{Met}}
\DeclareMathOperator{\BA}{\textbf{BA}}
\DeclareMathOperator{\Mon}{\textbf{Mon}}
\DeclareMathOperator{\Top}{\textbf{Top}}
\DeclareMathOperator{\Aut}{\textbf{Aut}}
\DeclareMathOperator{\RMod}{R-\textbf{Mod}}
\DeclareMathOperator{\RAlg}{R-\textbf{Alg}}
\DeclareMathOperator{\LF}{LF}
\DeclareMathOperator{\op}{op}
% Model theory
\DeclareMathOperator{\tp}{tp}
\DeclareMathOperator{\Diag}{Diag}
\DeclareMathOperator{\el}{el}
\DeclareMathOperator{\depth}{depth}
\DeclareMathOperator{\FO}{FO}
\DeclareMathOperator{\fin}{fin}
\DeclareMathOperator{\qr}{qr}
\DeclareMathOperator{\Mod}{Mod}
\DeclareMathOperator{\TC}{TC}
\DeclareMathOperator{\KH}{KH}
\DeclareMathOperator{\Part}{Part}
\DeclareMathOperator{\Infset}{\textsf{Infset}}
\DeclareMathOperator{\DLO}{\textsf{DLO}}
\DeclareMathOperator{\sfMod}{\textsf{Mod}}
\DeclareMathOperator{\AbG}{\textsf{AbG}}
\DeclareMathOperator{\sfACF}{\textsf{ACF}}
% Computability Theorem
\DeclareMathOperator{\Tot}{Tot}
\DeclareMathOperator{\graph}{graph}
\DeclareMathOperator{\Fin}{Fin}
\DeclareMathOperator{\Cof}{Cof}
\DeclareMathOperator{\lh}{lh}
% Commutative Algebra
\DeclareMathOperator{\ord}{ord}
\DeclareMathOperator{\Idem}{Idem}
\DeclareMathOperator{\zdiv}{z.div}
\DeclareMathOperator{\Frac}{Frac}
\DeclareMathOperator{\rad}{rad}
\DeclareMathOperator{\nil}{nil}
\DeclareMathOperator{\Ann}{Ann}
\DeclareMathOperator{\End}{End}
\DeclareMathOperator{\coim}{coim}
\DeclareMathOperator{\coker}{coker}
\DeclareMathOperator{\Bil}{Bil}
\DeclareMathOperator{\Tril}{Tril}
% Topology
\newcommand{\interior}[1]{%
  {\kern0pt#1}^{\mathrm{o}}%
}

% \makeatletter
% \newcommand{\vect}[1]{%
%   \vbox{\m@th \ialign {##\crcr
%   \vectfill\crcr\noalign{\kern-\p@ \nointerlineskip}
%   $\hfil\displaystyle{#1}\hfil$\crcr}}}
% \def\vectfill{%
%   $\m@th\smash-\mkern-7mu%
%   \cleaders\hbox{$\mkern-2mu\smash-\mkern-2mu$}\hfill
%   \mkern-7mu\raisebox{-3.81pt}[\p@][\p@]{$\mathord\mathchar"017E$}$}

% \newcommand{\amsvect}{%
%   \mathpalette {\overarrow@\vectfill@}}
% \def\vectfill@{\arrowfill@\relbar\relbar{\raisebox{-3.81pt}[\p@][\p@]{$\mathord\mathchar"017E$}}}

% \newcommand{\amsvectb}{%
% \newcommand{\vect}{%
%   \mathpalette {\overarrow@\vectfillb@}}
% \newcommand{\vecbar}{%
%   \scalebox{0.8}{$\relbar$}}
% \def\vectfillb@{\arrowfill@\vecbar\vecbar{\raisebox{-4.35pt}[\p@][\p@]{$\mathord\mathchar"017E$}}}
% \makeatother
% \bigtimes

\DeclareFontFamily{U}{mathx}{\hyphenchar\font45}
\DeclareFontShape{U}{mathx}{m}{n}{
      <5> <6> <7> <8> <9> <10>
      <10.95> <12> <14.4> <17.28> <20.74> <24.88>
      mathx10
      }{}
\DeclareSymbolFont{mathx}{U}{mathx}{m}{n}
\DeclareMathSymbol{\bigtimes}{1}{mathx}{"91}
% \odiv
\DeclareFontFamily{U}{matha}{\hyphenchar\font45}
\DeclareFontShape{U}{matha}{m}{n}{
      <5> <6> <7> <8> <9> <10> gen * matha
      <10.95> matha10 <12> <14.4> <17.28> <20.74> <24.88> matha12
      }{}
\DeclareSymbolFont{matha}{U}{matha}{m}{n}
\DeclareMathSymbol{\odiv}         {2}{matha}{"63}


\newcommand\subsetsim{\mathrel{%
  \ooalign{\raise0.2ex\hbox{\scalebox{0.9}{$\subset$}}\cr\hidewidth\raise-0.85ex\hbox{\scalebox{0.9}{$\sim$}}\hidewidth\cr}}}
\newcommand\simsubset{\mathrel{%
  \ooalign{\raise-0.2ex\hbox{\scalebox{0.9}{$\subset$}}\cr\hidewidth\raise0.75ex\hbox{\scalebox{0.9}{$\sim$}}\hidewidth\cr}}}

\newcommand\simsubsetsim{\mathrel{%
  \ooalign{\raise0ex\hbox{\scalebox{0.8}{$\subset$}}\cr\hidewidth\raise1ex\hbox{\scalebox{0.75}{$\sim$}}\hidewidth\cr\raise-0.95ex\hbox{\scalebox{0.8}{$\sim$}}\cr\hidewidth}}}
\newcommand{\stcomp}[1]{{#1}^{\mathsf{c}}}

\setlength{\baselineskip}{0.8in}

\stackMath
\newcommand\yrightarrow[2][]{\mathrel{%
  \setbox2=\hbox{\stackon{\scriptstyle#1}{\scriptstyle#2}}%
  \stackunder[0pt]{%
    \xrightarrow{\makebox[\dimexpr\wd2\relax]{$\scriptstyle#2$}}%
  }{%
   \scriptstyle#1\,%
  }%
}}
\newcommand\yleftarrow[2][]{\mathrel{%
  \setbox2=\hbox{\stackon{\scriptstyle#1}{\scriptstyle#2}}%
  \stackunder[0pt]{%
    \xleftarrow{\makebox[\dimexpr\wd2\relax]{$\scriptstyle#2$}}%
  }{%
   \scriptstyle#1\,%
  }%
}}
\newcommand\yRightarrow[2][]{\mathrel{%
  \setbox2=\hbox{\stackon{\scriptstyle#1}{\scriptstyle#2}}%
  \stackunder[0pt]{%
    \xRightarrow{\makebox[\dimexpr\wd2\relax]{$\scriptstyle#2$}}%
  }{%
   \scriptstyle#1\,%
  }%
}}
\newcommand\yLeftarrow[2][]{\mathrel{%
  \setbox2=\hbox{\stackon{\scriptstyle#1}{\scriptstyle#2}}%
  \stackunder[0pt]{%
    \xLeftarrow{\makebox[\dimexpr\wd2\relax]{$\scriptstyle#2$}}%
  }{%
   \scriptstyle#1\,%
  }%
}}

\newcommand\altxrightarrow[2][0pt]{\mathrel{\ensurestackMath{\stackengine%
  {\dimexpr#1-7.5pt}{\xrightarrow{\phantom{#2}}}{\scriptstyle\!#2\,}%
  {O}{c}{F}{F}{S}}}}
\newcommand\altxleftarrow[2][0pt]{\mathrel{\ensurestackMath{\stackengine%
  {\dimexpr#1-7.5pt}{\xleftarrow{\phantom{#2}}}{\scriptstyle\!#2\,}%
  {O}{c}{F}{F}{S}}}}

\newenvironment{bsm}{% % short for 'bracketed small matrix'
  \left[ \begin{smallmatrix} }{%
  \end{smallmatrix} \right]}

\newenvironment{psm}{% % short for ' small matrix'
  \left( \begin{smallmatrix} }{%
  \end{smallmatrix} \right)}

\newcommand{\bbar}[1]{\mkern 1.5mu\overline{\mkern-1.5mu#1\mkern-1.5mu}\mkern 1.5mu}

\newcommand{\bigzero}{\mbox{\normalfont\Large\bfseries 0}}
\newcommand{\rvline}{\hspace*{-\arraycolsep}\vline\hspace*{-\arraycolsep}}

\font\zallman=Zallman at 40pt
\font\elzevier=Elzevier at 40pt

\newcommand\isoto{\stackrel{\textstyle\sim}{\smash{\longrightarrow}\rule{0pt}{0.4ex}}}
\newcommand\embto{\stackrel{\textstyle\prec}{\smash{\longrightarrow}\rule{0pt}{0.4ex}}}

% from http://www.actual.world/resources/tex/doc/TikZ.pdf

\tikzset{
modal/.style={>=stealth’,shorten >=1pt,shorten <=1pt,auto,node distance=1.5cm,
semithick},
world/.style={circle,draw,minimum size=0.5cm,fill=gray!15},
point/.style={circle,draw,inner sep=0.5mm,fill=black},
reflexive above/.style={->,loop,looseness=7,in=120,out=60},
reflexive below/.style={->,loop,looseness=7,in=240,out=300},
reflexive left/.style={->,loop,looseness=7,in=150,out=210},
reflexive right/.style={->,loop,looseness=7,in=30,out=330}
}


\makeatletter
\newcommand*{\doublerightarrow}[2]{\mathrel{
  \settowidth{\@tempdima}{$\scriptstyle#1$}
  \settowidth{\@tempdimb}{$\scriptstyle#2$}
  \ifdim\@tempdimb>\@tempdima \@tempdima=\@tempdimb\fi
  \mathop{\vcenter{
    \offinterlineskip\ialign{\hbox to\dimexpr\@tempdima+1em{##}\cr
    \rightarrowfill\cr\noalign{\kern.5ex}
    \rightarrowfill\cr}}}\limits^{\!#1}_{\!#2}}}
\newcommand*{\triplerightarrow}[1]{\mathrel{
  \settowidth{\@tempdima}{$\scriptstyle#1$}
  \mathop{\vcenter{
    \offinterlineskip\ialign{\hbox to\dimexpr\@tempdima+1em{##}\cr
    \rightarrowfill\cr\noalign{\kern.5ex}
    \rightarrowfill\cr\noalign{\kern.5ex}
    \rightarrowfill\cr}}}\limits^{\!#1}}}
\makeatother

% $A\doublerightarrow{a}{bcdefgh}B$

% $A\triplerightarrow{d_0,d_1,d_2}B$


\DeclareMathOperator{\Groups}{\textbf{Groups}}
\author{Steve Awodey}
\date{\today}
\title{Category Theory}
\hypersetup{
 pdfauthor={Steve Awodey},
 pdftitle={Category Theory},
 pdfkeywords={},
 pdfsubject={},
 pdfcreator={Emacs 27.1 (Org mode 9.3)}, 
 pdflang={English}}
\begin{document}

\maketitle
\tableofcontents

\section{Categories}
\label{sec:org0821802}
\subsection{Examples of categories}
\label{sec:org1c07c55}
\begin{definition}[]
A \textbf{functor}
\begin{equation*}
F:\bC\to\bC
\end{equation*}
between categories \(\bC\) and \(\bD\) is a mapping of objects to objects and
arrows to arrows, in such a way that
\begin{enumerate}
\item \(F(f:A\to B)=F(f):F(A)\to F(B)\)
\item \(F(1_A)=1_{F(A)}\)
\item \(F(g\circ f)=F(g)\circ F(f)\)
\end{enumerate}
\end{definition}
\subsection{Free categories}
\label{sec:orgfbc560b}
The "Kleene closure" of \(A\) is defined to be the set
\begin{equation*}
A^*=\{\text{words over $A$}\}
\end{equation*}
Also
\begin{equation*}
i:A\to A^*
\end{equation*}
defined by \(i(a)=a\) and called the "intersection of generators"

A monoid \(M\) is \textbf{freely generated} by a subset \(A\) of \(M\) if the
following conditions hold:
\begin{enumerate}
\item Every element \(m\in M\) can be written as a product of elements of \(A\)
\item No "nontrivial" relations hold in \(M\), that is, if \(a_1\dots
      a_j=a_1'\dots a_k'\) ,then this is required by the axioms for monoids
\end{enumerate}


Every monoid \(N\) has an underlying set \(\abs{N}\), and every monoid
homomorphism \(f:N\to M\) has an underlying function
\(\abs{f}:\abs{N}\to\abs{M}\). The free monoid \(M(A)\) on a set \(A\) is by
definition "the" monoid with the following UMP

\emph{Universal Mapping Property of \(M(A)\)}

There is a function \(i:A\to\abs{M(A)}\), and given any monoid \(N\) and any
function \(f:A\to\abs{N}\), there is a \textbf{unique} monoid homomorphism
\(\ove{f}:M(A)\to N\) s.t. \(\abs{\over{f}}\circ i=f\)

\emph{in} \(\Mon\)
\begin{center}\begin{tikzcd}
M(A)\arrow[r,"\ove{f}",dotted]&N
\end{tikzcd}\end{center}

\emph{in} \(\Sets\)
\begin{center}\begin{tikzcd}
\abs{M(A)}\arrow[r,"\abs{\ove{f}}"]&\abs{N}\\
A\arrow[u,"i"]\arrow[ur,"f"']&
\end{tikzcd}\end{center}
\begin{proposition}[]
\(A^*\) has the UMP of the free monoid on \(A\)
\end{proposition}

\begin{proof}
Given \(f:A\to\abs{N}\), define \(\ove{f}:A^*\to N\) by
\begin{gather*}
\ove{f}(-)=u_N,\quad\text{ the unit of } N\\
\ove{f}(a_1\dots a_i)=f(a_1)\cdot_N\dots\cdot_N f(a_i)
\end{gather*}
\end{proof}


\section{Abstract structures}
\label{sec:orgefd690f}

\subsection{Initial and terminal objects}
\label{sec:org73d809c}
\begin{examplle}[]
A \textbf{Boolean algebra} is a poset \(B\) equipped with distinguished elements 0,1,
binary operations \(a\vee b\) of join and \(a\wedge b\) of meet, and a unary
operation \(\neg b\) of complementation. These are required to satisfy the
conditions
\begin{align*}
0&\le a\\
a&\le 1\\
a\le c \quad\text{ and }\quad b\le c \quad&\text{ iff }\quad a\vee b\le c\\
c\le a \quad\text{ and }\quad c\le b \quad&\text{ iff }\quad c\le a\wedge b\\
a\le\neg b \quad&\text{ iff }\quad a\wedge b=0\\
\neg\neg a&=a
\end{align*}

\(\textbf{2}=\{0,1\}\) is an initial elements of \(\BA\). \(\BA\) has as
arrows the Boolean homomorphisms that \(h(0)=0,h(a\vee b)=h(a)\vee h(b)\), etc.
\end{examplle}
\subsection{Products}
\label{sec:org4a583bb}
\begin{definition}[]
In any category \(\bC\), a \textbf{product diagram} for the objects \(A\) and \(B\)
consists of an object \(P\) and arrows
\begin{center}\begin{tikzcd}
A&P\arrow[l,"p_1"']\arrow[r,"p_2"]&B
\end{tikzcd}\end{center}

satisfying the following UMP:

Given any diagram of the form
\begin{center}\begin{tikzcd}
A&X\arrow[l,"x_1"']\arrow[r,"x_2"]&B
\end{tikzcd}\end{center}

there exists a unique \(u:X\to P\) making the diagram
\begin{center}\begin{tikzcd}
&X\arrow[dl,"x_1"']\arrow[dr,"x_2"]\arrow[d,dashed,"u"]\\
A&P\arrow[l,"p_1"]\arrow[r,"p_2"']&B
\end{tikzcd}\end{center}
\end{definition}

\subsection{Categories with products}
\label{sec:org3d51972}
Let \(\bC\) be a category that has a product diagram for every pair of
objects. Suppose we have objects and arrows
\begin{center}\begin{tikzcd}
A\arrow[d,"f"']&A\times A'\arrow[l,"p_1"']\arrow[r,"p_2"]&A'\arrow[d,"f'"]\\
B&B\times B'\arrow[l,"q_1"]\arrow[r,"q_2"']&B'
\end{tikzcd}\end{center}
with indicated products. Then we write
\begin{equation*}
f\times f':A\times A'\to B\times B
\end{equation*}
for \(f\times f'=\la f\circ p_1,f'\circ p_2\ra\)
\begin{center}\begin{tikzcd}
A\arrow[d,"f"']&A\times A'\arrow[l,"p_1"']\arrow[r,"p_2"]
\arrow[d,dashed,"f\times f'"]
&A'\arrow[d,"f'"]\\
B&B\times B'\arrow[l,"q_1"]\arrow[r,"q_2"']&B'
\end{tikzcd}\end{center}     In this way, if we choose a product for each
pair of objects, we get a functor
\begin{equation*}
\times:\bC\times\bC\to\bC
\end{equation*}

\begin{center}\begin{tikzcd}
A\arrow[d,"f"']&A\times A'\arrow[l,"p_1"']\arrow[r,"p_2"]\arrow[d,dashed,"f\times f'"]
\arrow[dd,dashed,bend right=60]
&A'\arrow[d,"f'"]\\
B\arrow[d,"g"']&B\times B'\arrow[l,"q_1"']\arrow[r,"q_2"]\arrow[d,dashed,"g\times g'"]&B'\arrow[d,"g'"]\\
C&C\times C'\arrow[l,"o_1"']\arrow[r,"o_2"]&C'\\
\end{tikzcd}\end{center}
\((g\circ f)\times(g'\circ f')=(f\times f')\circ(g\times g')\)



To prove
\begin{equation*}
(A\times B)\times C\cong A\times (B\times C)
\end{equation*}
Consider
\begin{center}\begin{tikzcd}
&A\times (B\times C)\arrow[r]\arrow[ldd]&B\times C\arrow[ld]\arrow[rdd]\\
&B&&\\
A&A\times B\arrow[u]\arrow[l]\arrow[rd]
&(A\times B)\times C\arrow[l]\arrow[r]\arrow[uu,dashed]\arrow[luu,dashed,"g"']
&C\\
&&B\\
&A\times(B\times C)\arrow[r]\arrow[luu]\arrow[uu,dashed]\arrow[ruu,dashed,"f"]&
B\times C\arrow[u]\arrow[ruu]
\end{tikzcd}\end{center}


Given no objects, there is an object 1 with no maps, and give nany other
object \(X\) and no maps, there is a unique arrow
\begin{equation*}
!:X\to 1
\end{equation*}

\begin{definition}[]
A category \(\bC\) is said to \textbf{have all finite products}, if it has a terminal
object and all binary products (and therewith products of any finite
cardinality). The category \(\bC\) \textbf{has all (small) products} if every set of
objects in \(\bC\) has a product
\end{definition}


\subsection{Hom-sets}
\label{sec:org3ea047a}
In this section, we assume that all categories are locally small

Given any objects \(A\) and \(B\)  in category \(\bC\),we write
\begin{equation*}
\Hom(A,B)=\{f\in\bC\mid f:A\to B\}
\end{equation*}
and call such a set of arrows a \textbf{Hom-set}

Note that any arrow \(g:B\to B'\) in \(\bC\) induces a function
\begin{gather*}
\Hom(A,g):\Hom(A,B)\to\Hom(A,B')\\
(f:A\to B)\mapsto(g\circ f:A\to B\to B')
\end{gather*}

Let's show that this determines a functor
\begin{equation*}
\Hom(A,-):\bC\to\Sets
\end{equation*}
called the (covariant) \textbf{representable functor} of \(A\). We need to show that
\begin{equation*}
\Hom(A,1_X)=1_{\Hom(A,X)}
\end{equation*}
and that
\begin{equation*}
\Hom(A,g\circ f)=\Hom(A,g)\circ\Hom(A,f)
\end{equation*}

For any object \(P\), a pair of arrows \(p_1:P\to A\) and \(p_2:P\to B\)
determine an element \((p_1,p_2)\) of the set
\begin{equation*}
\Hom(P,A)\times\Hom(P,B)
\end{equation*}
Now given any arrow
\begin{equation*}
x:X\to P
\end{equation*}
composing with \(p_1\) and \(p_2\) gives a pair of arrows
\(x_1=p_1\circ x:X\to A\) and \(x_2=p_2\circ x:X\to B\)

In this way, we have a function
\begin{equation*}
\theta_X=(\Hom(X,p_1),\Hom(X,p_2)):\Hom(X,P)\to\Hom(X,A)\times\Hom(X,B)
\end{equation*}
defined by
\begin{equation*}
\theta_X(x)=(x_1,x_2)
\end{equation*}
\begin{proposition}[]
A diagram of the form
\begin{center}\begin{tikzcd}
A&P\arrow[l,"p_1"]\arrow[r,"p_2"']&B
\end{tikzcd}\end{center}
is a product for \(A\) and \(B\) iff for every object \(X\), the canonical
function \(\theta_X\) is an isomorphism
\begin{equation*}
\theta_X:\Hom(X,P)\cong\Hom(X,A)\times\Hom(X,B)
\end{equation*}
\end{proposition}

\begin{proof}
Note that we are talking about isomorphism on the set
\end{proof}

\begin{definition}[]
Let \(\bC,\bD\) be categories with binary products. A functor
\(F:\bC\to\bD\) is said to \textbf{preserve binary products} if it takes every
product diagram
\begin{center}\begin{tikzcd}
A&A\times B\arrow[l,"p_1"]\arrow[r,"p_2"']&B
\end{tikzcd}\end{center}
to a product diagram
\begin{center}\begin{tikzcd}
FA&F(A\times B)\arrow[l,"Fp_1"]\arrow[r,"Fp_2"']&FB
\end{tikzcd}\end{center}
\end{definition}
\(F\) preserves products just if
\begin{equation*}
F(A\times B)\cong FA\times FB
\end{equation*}
\begin{corollary}[]
For any object \(X\) in a category \(\bC\) with products, the (covariant)
representable functor
\begin{equation*}
\Hom_{\bC}(X,-):\bC\to\Sets
\end{equation*}
preserves products
\end{corollary}

\section{Duality}
\label{sec:org7b36338}

\subsection{Coproducts}
\label{sec:orgedb378e}
\begin{definition}[]
A diagram
\begin{tikzcd}
A\arrow[r,"q_1"]&Q&B\arrow[l,"q_2"']
\end{tikzcd}
is a coproduct of \(A,B\) if for any \(Z\) and
\begin{tikzcd}
A\arrow[r,"z_1"]&Z&B\arrow[l,"z_2"']
\end{tikzcd}
there is a unique \(u:Q\to Z\) with \(u\circ q_i=z_i\)
\begin{center}\begin{tikzcd}
&Z\\
A\arrow[ur,"z_1"]\arrow[r,"q_1"]&Q\arrow[u,dotted,"u"]&B\arrow[l,"q_2"]\arrow[ul,"z_2"']
\end{tikzcd}\end{center}
written as \(A+B\)
\end{definition}

In \(\Sets\), every finite set \(A\) is a coproduct
\begin{equation*}
A\cong1+1+\dots+1\quad(n\text{-times})
\end{equation*}
\begin{examplle}[]
If \(M(A)\) and \(M(B)\) are free monoids on sets \(A\) and \(B\), then in
\(\Mon\) we can construct their coproduct as
\begin{equation*}
M(A)+M(B)\cong M(A+B)
\end{equation*}

\begin{center}\begin{tikzcd}
&N\\
M(A)\arrow[r]\arrow[ur]&M(A+B)\arrow[u,dotted]&M(B)\arrow[l]\arrow[ul]\\
A\arrow[u,"\eta_A"]\arrow[r]&A+B\arrow[u,"\eta_{A+B}"]&B\arrow[l]\arrow[u,"\eta_B"']
\end{tikzcd}\end{center}
Here we are working in two different categories. Half below is in \(\Sets\),
the other is \(\Mon\)
\end{examplle}

Product of two powerset Boolean algebras \(\calp(A)\) and \(\calp(B)\) is
also a powerset
\begin{equation*}
\calp(A)\times\calp(B)\cong\calp(A+B)
\end{equation*}  
\begin{examplle}[]
Two monoids \(M(\abs{A}+\abs{B})\) is strings over the disjoint union
\(\abs{A}+\abs{B}\) of the underlying sets. That is, the elements of \(A\)
and \(B\) and the equivalence relation \(v\sim w\) is the least one
containing all instances of the following equations
\begin{align*}
(\dots xu_Ay\dots)&=(\dots xy\dots)\\
(\dots xu_By\dots)&=(\dots xy\dots)\\
(\dots aa'\dots)&=(\dots a\cdot_A a'\dots)\\
(\dots bb'\dots)&=(\dots b\cdot_B b'\dots)
\end{align*}
The unit is the equivalence class \([-]\) of the empty word. Multiplication
is
\begin{equation*}
[x\dots y]\cdot[x'\dots y']=[x\dots yx'\dots y']
\end{equation*}
The coproduct injections \(i_A:A\to A+B\) and \(i_B:B\to A+B\) are
\begin{equation*}
i_A(a)=[a],\quad i_B(b)=[b]
\end{equation*}
Given any homomorphisms \(f:A\to M\) and \(g:B\to M\) into a monoid, the
unique homomorphism
\begin{equation*}
[f,g]:A+B\to M
\end{equation*}
is defined by first extending the function
\([\abs{f},\abs{g}]:\abs{A}+\abs{B}\to\abs{M}\) to one \([f,g]'\) on the free
monoid \(M(\abs{A}+\abs{B})\)
\begin{center}\begin{tikzcd}
\abs{A}+\abs{B}\arrow[r,"{[\abs{f}+\abs{g}]}"]&\abs{M}\\
M(\abs{A}+\abs{B})\arrow[r,"{[f,g]}'"]\arrow[d,twoheadrightarrow]&M\\
M(\abs{A}+\abs{B})/\sim\arrow[ur,dotted,"{[f,g]}"']
\end{tikzcd}\end{center}
If \(v\sim w\) in \(M(\abs{A}+\abs{B})/\sim\) then \([f,g]'(v)=[f,g]'(w)\).
Thus \([f,g]'\) extends to the quotient to yield the desired map
\([f,g]:M(\abs{A}+\abs{B})/\sim\to M\)

This construction also works to give coproducts in \(\Groups\), where it is
called the \textbf{free product} of \(A\) and \(B\) and written as \(A\oplus B\).
\end{examplle}

\begin{proposition}[]
In the category \(\Ab\) of abelian groups, there is a canonical isomorphism
between the binary coproduct and product
\begin{equation*}
A+B\cong A\times B
\end{equation*}
\end{proposition}

\begin{proof}
Take \(1_A:A\to A\) and \(O_B:A\to B\) . we get
\begin{equation*}
\theta=[\la 1_A,0_B\ra,\la 0_A,1_B\ra]:A+B\to A\times B
\end{equation*}
Then given any \((a,b)\in A+B\), we have
\begin{align*}
\theta(a,b)&=
[\la 1_A,0_B\ra,\la0_A,1_B\ra](a,b)\\
&=\la 1_A,0_B\ra(a)+\la0_A,1_B\ra(b)\\
&=(1_A(a),0_B(A))+(0_A(b),1_B(b))\\
&=(a,0_B)+(0_A,b)\\
&=(a+0_A,0_B+b)\\
&=(a,b)
\end{align*}
\end{proof}

\begin{proposition}[]
Coproducts are unique up to isomorphism
\end{proposition}

\subsection{Equalizers}
\label{sec:org0391206}
\begin{definition}[]
In any category \(\bC\), given parallel arrows
\begin{center}\begin{tikzcd}
A\arrow[r,yshift=0.7ex,"f"]\arrow[r,yshift=-0.7ex,"g"']&B
\end{tikzcd}\end{center}
an \textbf{equalizer} of \(f\) and \(g\) consists of an object \(E\) and an arrow
\(e:E\to A\), universal s.t.
\begin{equation*}
f\circ e=g\circ e
\end{equation*}
That is, given any \(z:Z\to A\) with \(f\circ z=g\circ z\), there is a \textbf{unique}
\(u:Z\to E\) with \(e\circ u=z\), all as in the diagram
\begin{center}\begin{tikzcd}
E\arrow[r,"e"]&A\arrow[r,yshift=0.7ex,"f"]\arrow[r,yshift=-0.7ex,"g"']&B\\
Z\arrow[u,dotted,"u"]\arrow[ur,"z"']
\end{tikzcd}\end{center}
\end{definition}

\begin{examplle}[]
Suppose we have the functions \(f,g:\R^2\rightrightarrows\R\), where
\begin{align*}
&f(x,y)=x^2+y^2\\
&g(x,y)=1
\end{align*}
and we take the equalizer, say in \(\Top\). This is the subspace
\begin{equation*}
S=\{(x,y)\in\R^2\mid x^2+y^2=1\}\hookrightarrow\R^2
\end{equation*}
For, given any "generalized element" \(z:Z\to\R^2\) we get a pair of such
"elements" \(z_1,z_2:Z\to\R\) just by composing with the two projections,
\(z=\la z_1,z_2\ra\) and for these we then have
\begin{align*}
f(z)=g(z) &\quad\text{ iff }\quad z_1^2+z_2^2=1\\
&\quad\text{ iff }\quad \la z_1,z_2\ra=z\in S
\end{align*}
where the last line really means that there is a factorization
\(z=\ove{z}\circ i\) of \(z\) through the inclusion
\(i:S\hookrightarrow\R^2\), as indicated in the following diagram
\begin{center}\begin{tikzcd}
S\arrow[r,hook,"i"]&\R^2\arrow[r,yshift=0.7ex,"x^2+y^2"]\arrow[r,yshift=-0.7ex,"1"']&\R\\
Z\arrow[u,dotted,"\ove{z}"]\arrow[ur,"z"']
\end{tikzcd}\end{center}
Since the inclusion \(i\) is monic, such a factorization, if it exists, is
necessarily unique, and thus \(S\hookrightarrow\R^2\) is indeed the equalizer
of \(f\) and \(g\)
\end{examplle}

\begin{examplle}[]
In \(\Sets\), given any functions \(f,g:A\rightrightarrows B\), their
equalizer is the inclusion into \(A\) of the equationally defined subset
\begin{equation*}
\{x\in A\mid f(x)=g(x)\}\hookrightarrow A
\end{equation*}

Let
\begin{equation*}
2=\{\top,\bot\}
\end{equation*}
Then consider the \textbf{characteristic function}
\begin{equation*}
\chi_U:A\to 2
\end{equation*}
defined for \(x\in A\) by
\begin{equation*}
\chi_U(x)=
\begin{cases}
\top&x\in U\\
\bot&x\not\in U
\end{cases}
\end{equation*}
So the following is an equalizer
\begin{center}\begin{tikzcd}
   U\arrow[r]&A\arrow[r,yshift=0.7ex,"\top!"]\arrow[r,\yshift=-.7ex,"\chi_U"']&2
\end{tikzcd}\end{center}

where \(\top!=\top\circ!:U\xrightarrow{!}1\xrightarrow{\top}2\)

Moreover, for every function,
\begin{equation*}
\varphi:A\to 2
\end{equation*}
we can form the variety
\begin{equation*}
V_{\varphi}=\{x\in A\mid \varphi(x)=\top\}
\end{equation*}
as an equalizer.

It is easy to see that these operations \(\chi_U\) and \(V_\varphi\) are
mutually inverse
\begin{align*}
 V_{\chi_U}&=\{x\in A\mid \chi_U(x)=\top\}\\
 &=\{x\in A\mid x\in U\}\\
 &=U
\end{align*}
for any \(U\subseteq A\), and given any \(\varphi:A\to2\)
\begin{align*}
\chi_{V_\varphi}(x)&=
\begin{cases}
\top&x\in V_\varphi\\
\bot&x\not\in V_\varphi\\
\end{cases}\\
&=
\begin{cases}
\top&\varphi(x)=\top\\
\bot&\varphi(x)=\bot
\end{cases}\\
&=\varphi(x)
\end{align*}
Thus we have the familiar isomorphism
\begin{equation*}
\Hom(A,2)\cong P(A)
\end{equation*}
mediated by taking equalizers
\end{examplle}

\begin{proposition}[]
In any category, if \(e:E\to A\) is an equalizer of some pair of arrows, then
\(e\) is monic
\end{proposition}

\begin{proof}
Consider
\begin{center}\begin{tikzcd}
E\arrow[r,"e"]&A\arrow[r,yshift=.7ex,"f"]\arrow[r,yshift=-.7ex,"g"']&B\\
Z\arrow[u,shift right=-.7ex,"x"]\arrow[u,shift right=.7ex,"y"']\arrow[ur,"z"']
\end{tikzcd}\end{center}
Suppose \(ex=ey\), we want to show \(x=y\). Put \(z=ex=ey\). Then
\(fz=fex=gex=gz\), so there is a unique \(u:Z\to E\) s.t. \(eu=z\). So \(x=u=y\)
\end{proof}

\begin{examplle}[]
In abelian groups, one has an alternate description of the equalizer, using
the fact that
\begin{equation*}
f(x)=g(x) \quad\text{ iff }\quad (f-g)(x)=0
\end{equation*}
\end{examplle}


\subsection{Coequalizers}
\label{sec:org951afc2}
\begin{definition}[]
For any parallel arrows \(f,g:A\to B\) in a category \(\bC\), a \textbf{coequalizer}
consists of \(Q\) and \(q:B\to Q\), universal with the property \(qf=qg\) as
in
\begin{center}\begin{tikzcd}
A\arrow[r,yshift=.7ex,"f"]\arrow[r,-.7ex,"g"']&B\arrow[r,"q"]\arrow[rd,"z"]&Q
\arrow[d,dotted,"u"]\\
&&Z
\end{tikzcd}\end{center}
That is, given any \(Z\) and \(z:B\to Z\) if \(zf=zg\), then there exists a
unique \(u:Q\to Z\) s.t. \(uq=z\)
\end{definition}

\begin{proposition}[]
If \(q:B\to Q\) is a coequalizer of some pair of arrows, then \(q\) is epic
\end{proposition}

We can therefore think of a coequalizer \(q:B\twoheadrightarrow Q\) as a
"collapse" of \(B\) by "identifying" all pairs \(f(a)=g(a)\)

\begin{examplle}[]
Let \(R\subseteq X\times X\) be an equivalence relation on a set \(X\), and
consider the diagram
\begin{center}\begin{tikzcd}
R\arrow[r,yshift=.7ex,"r_1"]\arrow[r,yshift=-.7ex,"f_2"']&X
\end{tikzcd}\end{center}
where the \(r\)'s are the two projections of the inclusion
\(R\subseteq X\times X\)
\begin{center}\begin{tikzcd}
&R\arrow[ld,"r_1"']\arrow[rd,"r_2"]\arrow[d,hook]\\
X&X\times X\arrow[l,"p_1"]\arrow[r,"p_2"']&X
\end{tikzcd}\end{center}
The quotient projection
\begin{equation*}
\pi:X\to X/R
\end{equation*}
defined by \(x\mapsto[x]\) is then a coequalizer of \(r_1\) and \(r_2\). For
given an \(f:X\to Y\) as in
\begin{center}\begin{tikzcd}
R\arrow[r,yshift=.7ex,"r_1"]\arrow[r,yshift=-.7ex,"r_2"']&X\arrow[r,"\pi"]\arrow[rd,"f"']
&X/R\arrow[d,dotted,"\ove{f}"]\\
&&Y
\end{tikzcd}\end{center}
there exists a function \(\ove{f}\) s.t.
\begin{equation*}
\ove{f}\pi (x)=f(x)
\end{equation*}
whenever \(f\) respects \(R\) in the sense that \((x,x')\in R\) implies
\(f(x)=f(x')\). But this condition just says that \(f\circ r_1=f\circ
   r_2\) since \(f\circ r_1(x,x')=f(x')\) and \(f\circ r_2(x,x')=f(x')\) for all
\((x,x')\in R\). Moreover, if it exists, such a function \(\ove{f}\) is then
necessarily unique, since \(\pi\) is an epimorphism

The coequalizer in \(\Sets\) of an arbitrary parallel pair of function
\(f,g:A\twoheadrightarrow B\) can be constructed by quotienting \(B\) by the
equivalence relation generated by the equations \(f(x)=g(x)\)for all \(x\in
   A\)

Consider
\begin{center}\begin{tikzcd}
A\arrow[r,yshift=0.7ex,"f"]\arrow[r,yshift=-.7ex,"g"']&B\arrow[r]&Q=B/(f=g)
\end{tikzcd}\end{center}
where the equivalence relation \(R\) on \(b\) is generated by the pairs
\((f(x),g(x))\) for all \(x\in A\). That is, \(R\) is the intersection of all
equivalence relations on \(B\) containing all such pairs
\end{examplle}

\begin{examplle}[]
Taken in posets
\begin{center}\begin{tikzcd}
1\arrow[r,yshift=.7ex,"0_P"]\arrow[r,yshift=-.7ex,"0_Q"']&P+Q\arrow[r]
&P+Q/(0_P=0_Q)
\end{tikzcd}\end{center}
\((0_P=0_Q)\) is the equivalent closure of \((0_P(1),0_Q(1))\).
\end{examplle}

\begin{examplle}[Presentations of algebras]
Suppose we are given
\begin{align*}
&\text{Generators: }\quad x,y,z\\
&\text{Relations: }\quad xy=z,y^2=1
\end{align*}
To build an algebra on these generators and satisfying these relations, start
with the free algebra
\begin{equation*}
F(3)=F(x,y,z)
\end{equation*}
and then "force" the relation \(xy=z\) to hold by taking a coequalizer of the
maps
\begin{center}\begin{tikzcd}
F(1)\arrow[r,yshift=.7ex,"xy"]\arrow[r,yshift=-.7ex,"z"']&F(3)\arrow[r,"q"]&Q
\end{tikzcd}\end{center}
We use the fact that maps \(F(1)\to A\) correspond to elements \(a\in A\) by
\(v\mapsto a\), where \(v\) is the single generator of \(F(1)\). Now
similarly for the equation \(y^2=1\), taking the coequalizer
\begin{center}\begin{tikzcd}
F(1)\arrow[r,yshift=.7ex,"q(y^2)"]\arrow[r,yshift=-.7ex,"q(1)"']&Q\arrow[r]&Q'
\end{tikzcd}\end{center}
These two steps can actually be done simultaneously. Let
\begin{gather*}
 F(2)=F(1)+F(1)\\
 F(2)\doublerightarrow{f}{g}F(3)
\end{gather*}
where \(f=[xy,y^2]\) and \(g=[z,1]\)
\end{examplle}
\end{document}