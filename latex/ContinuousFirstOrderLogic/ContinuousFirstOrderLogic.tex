% Created 2021-06-22 Tue 20:42
% Intended LaTeX compiler: pdflatex
\documentclass[11pt]{article}
\usepackage[utf8]{inputenc}
\usepackage[T1]{fontenc}
\usepackage{graphicx}
\usepackage{grffile}
\usepackage{longtable}
\usepackage{wrapfig}
\usepackage{rotating}
\usepackage[normalem]{ulem}
\usepackage{amsmath}
\usepackage{textcomp}
\usepackage{amssymb}
\usepackage{capt-of}
\usepackage{hyperref}
%%%%%%%%%%%%%%%%%%%%%%%%%%%%%%%%%%%%%%
%% TIPS                                 %%
%%%%%%%%%%%%%%%%%%%%%%%%%%%%%%%%%%%%%%
% \substack{a\\b} for multiple lines text

\usepackage[utf8]{inputenc}

\usepackage[B1,T1]{fontenc}

% pdfplots will load xolor automatically without option
\usepackage[dvipsnames]{xcolor}
%%%%%%%%%%%%%%%%%%%%%%%%%%%%%%%%%%%%%%%
%% MATH related pacakge                  %%
%%%%%%%%%%%%%%%%%%%%%%%%%%%%%%%%%%%%%%%
% \usepackage{amsmath} mathtools loads the amsmath
\usepackage{amsmath}
\usepackage{mathtools}


\usepackage{amsthm}
\usepackage{amsbsy}

%\usepackage{commath}

\usepackage{amssymb}
\usepackage{mathrsfs}
%\usepackage{mathabx}
\usepackage{stmaryrd}
\usepackage{empheq}

%for \not\ll
\usepackage{centernot}

\usepackage{scalerel}
\usepackage{stackengine}
\usepackage{stackrel}

\usepackage{nicematrix}
\usepackage{tensor}
\usepackage{blkarray}
\usepackage{siunitx}
\usepackage[f]{esvect}

\usepackage{unicode-math}
\setmainfont{TeX Gyre Pagella}
% \setmathfont{STIX}
%\setmathfont{texgyrepagella-math.otf}
%\setmathfont{Libertinus Math}
\setmathfont{Latin Modern Math}

 
% \setmathfont[range={\smwhtdiamond,\enclosediamond,\varlrtriangle}]{Latin Modern Math}
 \setmathfont[range={\rightrightarrows,\twoheadrightarrow,\leftrightsquigarrow,\triangledown,\vartriangle}]{XITS Math}
 \setmathfont[range={\int,\setminus}]{Libertinus Math}
 \setmathfont[range={\mathalpha}]{TeX Gyre Pagella Math}
% unicode is not good at this!
%\let\nmodels\nvDash


%%%%%%%%%%%%%%%%%%%%%%%%%%%%%%%%%%%%%%%
%% TIKZ related packages                 %%
%%%%%%%%%%%%%%%%%%%%%%%%%%%%%%%%%%%%%%%

\usepackage{pgfplots}
\pgfplotsset{compat=1.15}
\usepackage{tikz}
\usepackage{tikz-cd}
\usepackage{tikz-qtree}

\usetikzlibrary{arrows,positioning,calc,fadings,decorations,matrix,decorations,shapes.misc}
%setting from geogebra
\definecolor{ccqqqq}{rgb}{0.8,0,0}


%%%%%%%%%%%%%%%%%%%%%%%%%%%%%%%%%%%%%%%
%% MISCLELLANEOUS packages               %%
%%%%%%%%%%%%%%%%%%%%%%%%%%%%%%%%%%%%%%%
\usepackage[most]{tcolorbox}
\usepackage{threeparttable}
\usepackage{tabularx}

\usepackage{enumitem}

% wrong with preview
\usepackage{subcaption}
\usepackage{caption}
% {\aunclfamily\Huge}
\usepackage{auncial}

\usepackage{float}

\usepackage{fancyhdr}

\usepackage{ifthen}
\usepackage{xargs}


\usepackage{imakeidx}
\usepackage{hyperref}
\usepackage{soul}


%\usepackage[xetex]{preview}
%%%%%%%%%%%%%%%%%%%%%%%%%%%%%%%%%%%%%%%
%% USEPACKAGES end                       %%
%%%%%%%%%%%%%%%%%%%%%%%%%%%%%%%%%%%%%%%

% \setlist{nosep}
% \numberwithin{equation}{subsection}
% \fancyhead{} % Clear the headers
% \renewcommand{\headrulewidth}{0pt} % Width of line at top of page
% \fancyhead[R]{\slshape\leftmark} % Mark right [R] of page with Chapter name [\leftmark]
% \pagestyle{fancy} % Set default style for all content pages (not TOC, etc)


% \newlength\shlength
% \newcommand\vect[2][0]{\setlength\shlength{#1pt}%
%   \stackengine{-5.6pt}{$#2$}{\smash{$\kern\shlength%
%     \stackengine{7.55pt}{$\mathchar"017E$}%
%       {\rule{\widthof{$#2$}}{.57pt}\kern.4pt}{O}{r}{F}{F}{L}\kern-\shlength$}}%
%       {O}{c}{F}{T}{S}}


\indexsetup{othercode=\small}
\makeindex[columns=2,options={-s /media/wu/file/stuuudy/notes/index_style.ist},intoc]
\makeatletter
\def\@idxitem{\par\hangindent 0pt}
\makeatother


%\newcounter{dummy} \numberwithin{dummy}{section}
\newtheorem{dummy}{dummy}[section]
\theoremstyle{definition}
\newtheorem{definition}[dummy]{Definition}
\theoremstyle{plain}
\newtheorem{corollary}[dummy]{Corollary}
\newtheorem{lemma}[dummy]{Lemma}
\newtheorem{proposition}[dummy]{Proposition}
\newtheorem{theorem}[dummy]{Theorem}
\theoremstyle{definition}
\newtheorem{examplle}{Example}[section]
\theoremstyle{remark}
\newtheorem*{remark}{Remark}
\newtheorem{exercise}{Exercise}[subsection]
\newtheorem{observation}{Observation}[section]


\newenvironment{claim}[1]{\par\noindent\textbf{Claim:}\space#1}{}

\makeatletter
\DeclareFontFamily{U}{tipa}{}
\DeclareFontShape{U}{tipa}{m}{n}{<->tipa10}{}
\newcommand{\arc@char}{{\usefont{U}{tipa}{m}{n}\symbol{62}}}%

\newcommand{\arc}[1]{\mathpalette\arc@arc{#1}}

\newcommand{\arc@arc}[2]{%
  \sbox0{$\m@th#1#2$}%
  \vbox{
    \hbox{\resizebox{\wd0}{\height}{\arc@char}}
    \nointerlineskip
    \box0
  }%
}
\makeatother

\setcounter{MaxMatrixCols}{20}
%%%%%%% ABS
\DeclarePairedDelimiter\abss{\lvert}{\rvert}%
\DeclarePairedDelimiter\normm{\lVert}{\rVert}%

% Swap the definition of \abs* and \norm*, so that \abs
% and \norm resizes the size of the brackets, and the
% starred version does not.
\makeatletter
\let\oldabs\abss
%\def\abs{\@ifstar{\oldabs}{\oldabs*}}
\newcommand{\abs}{\@ifstar{\oldabs}{\oldabs*}}
\newcommand{\norm}[1]{\left\lVert#1\right\rVert}
%\let\oldnorm\normm
%\def\norm{\@ifstar{\oldnorm}{\oldnorm*}}
%\renewcommand{norm}{\@ifstar{\oldnorm}{\oldnorm*}}
\makeatother

% \newcommand\what[1]{\ThisStyle{%
%     \setbox0=\hbox{$\SavedStyle#1$}%
%     \stackengine{-1.0\ht0+.5pt}{$\SavedStyle#1$}{%
%       \stretchto{\scaleto{\SavedStyle\mkern.15mu\char'136}{2.6\wd0}}{1.4\ht0}%
%     }{O}{c}{F}{T}{S}%
%   }
% }

% \newcommand\wtilde[1]{\ThisStyle{%
%     \setbox0=\hbox{$\SavedStyle#1$}%
%     \stackengine{-.1\LMpt}{$\SavedStyle#1$}{%
%       \stretchto{\scaleto{\SavedStyle\mkern.2mu\AC}{.5150\wd0}}{.6\ht0}%
%     }{O}{c}{F}{T}{S}%
%   }
% }

% \newcommand\wbar[1]{\ThisStyle{%
%     \setbox0=\hbox{$\SavedStyle#1$}%
%     \stackengine{.5pt+\LMpt}{$\SavedStyle#1$}{%
%       \rule{\wd0}{\dimexpr.3\LMpt+.3pt}%
%     }{O}{c}{F}{T}{S}%
%   }
% }

\newcommand{\bl}[1] {\boldsymbol{#1}}
\newcommand{\Wt}[1] {\stackrel{\sim}{\smash{#1}\rule{0pt}{1.1ex}}}
\newcommand{\wt}[1] {\widetilde{#1}}
\newcommand{\tf}[1] {\textbf{#1}}


%For boxed texts in align, use Aboxed{}
%otherwise use boxed{}

\DeclareMathSymbol{\widehatsym}{\mathord}{largesymbols}{"62}
\newcommand\lowerwidehatsym{%
  \text{\smash{\raisebox{-1.3ex}{%
    $\widehatsym$}}}}
\newcommand\fixwidehat[1]{%
  \mathchoice
    {\accentset{\displaystyle\lowerwidehatsym}{#1}}
    {\accentset{\textstyle\lowerwidehatsym}{#1}}
    {\accentset{\scriptstyle\lowerwidehatsym}{#1}}
    {\accentset{\scriptscriptstyle\lowerwidehatsym}{#1}}
  }


\newcommand{\cupdot}{\mathbin{\dot{\cup}}}
\newcommand{\bigcupdot}{\mathop{\dot{\bigcup}}}

\usepackage{graphicx}

\usepackage[toc,page]{appendix}

% text on arrow for xRightarrow
\makeatletter
%\newcommand{\xRightarrow}[2][]{\ext@arrow 0359\Rightarrowfill@{#1}{#2}}
\makeatother

% Arbitrary long arrow
\newcommand{\Rarrow}[1]{%
\parbox{#1}{\tikz{\draw[->](0,0)--(#1,0);}}
}

\newcommand{\LRarrow}[1]{%
\parbox{#1}{\tikz{\draw[<->](0,0)--(#1,0);}}
}


\makeatletter
\providecommand*{\rmodels}{%
  \mathrel{%
    \mathpalette\@rmodels\models
  }%
}
\newcommand*{\@rmodels}[2]{%
  \reflectbox{$\m@th#1#2$}%
}
\makeatother







\newcommand{\trcl}[1]{%
  \mathrm{trcl}{(#1)}
}



% Roman numerals
\makeatletter
\newcommand*{\rom}[1]{\expandafter\@slowromancap\romannumeral #1@}
\makeatother
% \\def \\b\([a-zA-Z]\) {\\boldsymbol{[a-zA-z]}}
% \\DeclareMathOperator{\\b\1}{\\textbf{\1}}


\DeclareMathOperator{\bx}{\textbf{x}}
\DeclareMathOperator{\bz}{\textbf{z}}
\DeclareMathOperator{\bff}{\textbf{f}}
\DeclareMathOperator{\ba}{\textbf{a}}
\DeclareMathOperator{\bk}{\textbf{k}}
\DeclareMathOperator{\bs}{\textbf{s}}
\DeclareMathOperator{\bh}{\textbf{h}}
\DeclareMathOperator{\bc}{\textbf{c}}
\DeclareMathOperator{\br}{\textbf{r}}
\DeclareMathOperator{\bi}{\textbf{i}}
\DeclareMathOperator{\bj}{\textbf{j}}
\DeclareMathOperator{\bn}{\textbf{n}}
\DeclareMathOperator{\be}{\textbf{e}}
\DeclareMathOperator{\bo}{\textbf{o}}
\DeclareMathOperator{\bU}{\textbf{U}}
\DeclareMathOperator{\bL}{\textbf{L}}
\DeclareMathOperator{\bV}{\textbf{V}}
\def \bzero {\mathbf{0}}
\def \btwo {\mathbf{2}}
\DeclareMathOperator{\bv}{\textbf{v}}
\DeclareMathOperator{\bp}{\textbf{p}}
\DeclareMathOperator{\bI}{\textbf{I}}
\DeclareMathOperator{\bM}{\textbf{M}}
\DeclareMathOperator{\bN}{\textbf{N}}
\DeclareMathOperator{\bK}{\textbf{K}}
\DeclareMathOperator{\bt}{\textbf{t}}
\DeclareMathOperator{\bb}{\textbf{b}}
\DeclareMathOperator{\bA}{\textbf{A}}
\DeclareMathOperator{\bX}{\textbf{X}}
\DeclareMathOperator{\bu}{\textbf{u}}
\DeclareMathOperator{\bS}{\textbf{S}}
\DeclareMathOperator{\bZ}{\textbf{Z}}
\DeclareMathOperator{\bJ}{\textbf{J}}
\DeclareMathOperator{\by}{\textbf{y}}
\DeclareMathOperator{\bw}{\textbf{w}}
\DeclareMathOperator{\bT}{\textbf{T}}
\DeclareMathOperator{\bF}{\textbf{F}}
\DeclareMathOperator{\bmm}{\textbf{m}}
\DeclareMathOperator{\bW}{\textbf{W}}
\DeclareMathOperator{\bR}{\textbf{R}}
\DeclareMathOperator{\bC}{\textbf{C}}
\DeclareMathOperator{\bD}{\textbf{D}}
\DeclareMathOperator{\bE}{\textbf{E}}
\DeclareMathOperator{\bQ}{\textbf{Q}}
\DeclareMathOperator{\bP}{\textbf{P}}
\DeclareMathOperator{\bY}{\textbf{Y}}
\DeclareMathOperator{\bH}{\textbf{H}}
\DeclareMathOperator{\bB}{\textbf{B}}
\DeclareMathOperator{\bG}{\textbf{G}}
\def \blambda {\symbf{\lambda}}
\def \boldeta {\symbf{\eta}}
\def \balpha {\symbf{\alpha}}
\def \bbeta {\symbf{\beta}}
\def \bgamma {\symbf{\gamma}}
\def \bxi {\symbf{\xi}}
\def \bLambda {\symbf{\Lambda}}
\def \bGamma {\symbf{\Gamma}}

\newcommand{\bto}{{\boldsymbol{\to}}}
\newcommand{\Ra}{\Rightarrow}
\newcommand\und[1]{\underline{#1}}
\newcommand\ove[1]{\overline{#1}}
\def \bPhi {\boldsymbol{\Phi}}
\def \btheta {\boldsymbol{\theta}}
\def \bTheta {\boldsymbol{\Theta}}
\def \bmu {\boldsymbol{\mu}}
\def \bphi {\boldsymbol{\phi}}
\def \bSigma {\boldsymbol{\Sigma}}
\def \lb {\left\{}
\def \rb {\right\}}
\def \la {\langle}
\def \ra {\rangle}
\def \caln {\mathcal{N}}
\def \dissum {\displaystyle\Sigma}
\def \dispro {\displaystyle\prod}
\def \E {\mathbb{E}}
\def \Q {\mathbb{Q}}
\def \N {\mathbb{N}}
\def \V {\mathbb{V}}
\def \R {\mathbb{R}}
\def \P {\mathbb{P}}
\def \A {\mathbb{A}}
\def \F {\mathbb{F}}
\def \Z {\mathbb{Z}}
\def \I {\mathbb{I}}
\def \C {\mathbb{C}}
\def \cala {\mathcal{A}}
\def \cale {\mathcal{E}}
\def \calb {\mathcal{B}}
\def \calq {\mathcal{Q}}
\def \calp {\mathcal{P}}
\def \cals {\mathcal{S}}
\def \calx {\mathcal{X}}
\def \caly {\mathcal{Y}}
\def \calg {\mathcal{G}}
\def \cald {\mathcal{D}}
\def \caln {\mathcal{N}}
\def \calr {\mathcal{R}}
\def \calt {\mathcal{T}}
\def \calm {\mathcal{M}}
\def \calw {\mathcal{W}}
\def \calc {\mathcal{C}}
\def \calv {\mathcal{V}}
\def \calf {\mathcal{F}}
\def \calk {\mathcal{K}}
\def \call {\mathcal{L}}
\def \calu {\mathcal{U}}
\def \calo {\mathcal{O}}
\def \calh {\mathcal{H}}
\def \cali {\mathcal{I}}

\def \bcup {\bigcup}

% set theory

\def \zfcc {\textbf{ZFC}^-}
\def \ac  {\textbf{AC}}
\def \gl  {\textbf{L }}
\def \gll {\textbf{L}}
\newcommand{\zfm}{$\textbf{ZF}^-$}

%\def \zfm {$\textbf{ZF}^-$}
\def \zfmm {\textbf{ZF}^-}
\def \wf {\textbf{WF }}
\def \on {\textbf{On }}
\def \cm {\textbf{M }}
\def \cn {\textbf{N }}
\def \cv {\textbf{V }}
\def \zc {\textbf{ZC }}
\def \zcm {\textbf{ZC}}
\def \zff {\textbf{ZF}}
\def \wfm {\textbf{WF}}
\def \onm {\textbf{On}}
\def \cmm {\textbf{M}}
\def \cnm {\textbf{N}}
\def \cvm {\textbf{V}}
\def \gchh {\textbf{GCH}}
\renewcommand{\restriction}{\mathord{\upharpoonright}}
\def \pred {\text{pred}}

\def \rank {\text{rank}}
\def \con {\text{Con}}
\def \deff {\text{Def}}


\def \uin {\underline{\in}}
\def \oin {\overline{\in}}
\def \uR {\underline{R}}
\def \oR {\overline{R}}
\def \uP {\underline{P}}
\def \oP {\overline{P}}

\def \dsum {\displaystyle\sum}

\def \Ra {\Rightarrow}

\def \e {\enspace}

\def \sgn {\operatorname{sgn}}
\def \gen {\operatorname{gen}}
\def \Hom {\operatorname{Hom}}
\def \hom {\operatorname{hom}}
\def \Sub {\operatorname{Sub}}

\def \supp {\operatorname{supp}}

\def \epiarrow {\twoheadarrow}
\def \monoarrow {\rightarrowtail}
\def \rrarrow {\rightrightarrows}

% \def \minus {\text{-}}
% \newcommand{\minus}{\scalebox{0.75}[1.0]{$-$}}
% \DeclareUnicodeCharacter{002D}{\minus}


\def \tril {\triangleleft}

\def \ACF {\text{ACF}}
\def \GL {\text{GL}}
\def \PGL {\text{PGL}}
\def \equal {=}
\def \deg {\text{deg}}
\def \degree {\text{degree}}
\def \app {\text{App}}
\def \FV {\text{FV}}
\def \conv {\text{conv}}
\def \cont {\text{cont}}
\DeclareMathOperator{\cl}{\textbf{CL}}
\DeclareMathOperator{\sg}{sg}
\DeclareMathOperator{\trdeg}{trdeg}
\def \Ord {\text{Ord}}

\DeclareMathOperator{\cf}{cf}
\DeclareMathOperator{\zfc}{ZFC}

%\DeclareMathOperator{\Th}{Th}
%\def \th {\text{Th}}
% \newcommand{\th}{\text{Th}}
\DeclareMathOperator{\type}{type}
\DeclareMathOperator{\zf}{\textbf{ZF}}
\def \fa {\mathfrak{a}}
\def \fb {\mathfrak{b}}
\def \fc {\mathfrak{c}}
\def \fd {\mathfrak{d}}
\def \fe {\mathfrak{e}}
\def \ff {\mathfrak{f}}
\def \fg {\mathfrak{g}}
\def \fh {\mathfrak{h}}
%\def \fi {\mathfrak{i}}
\def \fj {\mathfrak{j}}
\def \fk {\mathfrak{k}}
\def \fl {\mathfrak{l}}
\def \fm {\mathfrak{m}}
\def \fn {\mathfrak{n}}
\def \fo {\mathfrak{o}}
\def \fp {\mathfrak{p}}
\def \fq {\mathfrak{q}}
\def \fr {\mathfrak{r}}
\def \fs {\mathfrak{s}}
\def \ft {\mathfrak{t}}
\def \fu {\mathfrak{u}}
\def \fv {\mathfrak{v}}
\def \fw {\mathfrak{w}}
\def \fx {\mathfrak{x}}
\def \fy {\mathfrak{y}}
\def \fz {\mathfrak{z}}
\def \fA {\mathfrak{A}}
\def \fB {\mathfrak{B}}
\def \fC {\mathfrak{C}}
\def \fD {\mathfrak{D}}
\def \fE {\mathfrak{E}}
\def \fF {\mathfrak{F}}
\def \fG {\mathfrak{G}}
\def \fH {\mathfrak{H}}
\def \fI {\mathfrak{I}}
\def \fJ {\mathfrak{J}}
\def \fK {\mathfrak{K}}
\def \fL {\mathfrak{L}}
\def \fM {\mathfrak{M}}
\def \fN {\mathfrak{N}}
\def \fO {\mathfrak{O}}
\def \fP {\mathfrak{P}}
\def \fQ {\mathfrak{Q}}
\def \fR {\mathfrak{R}}
\def \fS {\mathfrak{S}}
\def \fT {\mathfrak{T}}
\def \fU {\mathfrak{U}}
\def \fV {\mathfrak{V}}
\def \fW {\mathfrak{W}}
\def \fX {\mathfrak{X}}
\def \fY {\mathfrak{Y}}
\def \fZ {\mathfrak{Z}}

\def \sfA {\textsf{A}}
\def \sfB {\textsf{B}}
\def \sfC {\textsf{C}}
\def \sfD {\textsf{D}}
\def \sfE {\textsf{E}}
\def \sfF {\textsf{F}}
\def \sfG {\textsf{G}}
\def \sfH {\textsf{H}}
\def \sfI {\textsf{I}}
\def \sfj {\textsf{J}}
\def \sfK {\textsf{K}}
\def \sfL {\textsf{L}}
\def \sfM {\textsf{M}}
\def \sfN {\textsf{N}}
\def \sfO {\textsf{O}}
\def \sfP {\textsf{P}}
\def \sfQ {\textsf{Q}}
\def \sfR {\textsf{R}}
\def \sfS {\textsf{S}}
\def \sfT {\textsf{T}}
\def \sfU {\textsf{U}}
\def \sfV {\textsf{V}}
\def \sfW {\textsf{W}}
\def \sfX {\textsf{X}}
\def \sfY {\textsf{Y}}
\def \sfZ {\textsf{Z}}
\def \sfa {\textsf{a}}
\def \sfb {\textsf{b}}
\def \sfc {\textsf{c}}
\def \sfd {\textsf{d}}
\def \sfe {\textsf{e}}
\def \sff {\textsf{f}}
\def \sfg {\textsf{g}}
\def \sfh {\textsf{h}}
\def \sfi {\textsf{i}}
\def \sfj {\textsf{j}}
\def \sfk {\textsf{k}}
\def \sfl {\textsf{l}}
\def \sfm {\textsf{m}}
\def \sfn {\textsf{n}}
\def \sfo {\textsf{o}}
\def \sfp {\textsf{p}}
\def \sfq {\textsf{q}}
\def \sfr {\textsf{r}}
\def \sfs {\textsf{s}}
\def \sft {\textsf{t}}
\def \sfu {\textsf{u}}
\def \sfv {\textsf{v}}
\def \sfw {\textsf{w}}
\def \sfx {\textsf{x}}
\def \sfy {\textsf{y}}
\def \sfz {\textsf{z}}



%\DeclareMathOperator{\ker}{ker}
\DeclareMathOperator{\im}{im}

\DeclareMathOperator{\inn}{Inn}
\DeclareMathOperator{\AC}{\textbf{AC}}
\DeclareMathOperator{\cod}{cod}
\DeclareMathOperator{\dom}{dom}
\DeclareMathOperator{\ran}{ran}
\DeclareMathOperator{\textd}{d}
\DeclareMathOperator{\td}{d}
\DeclareMathOperator{\id}{id}
\DeclareMathOperator{\LT}{LT}
\DeclareMathOperator{\Mat}{Mat}
\DeclareMathOperator{\Eq}{Eq}
\DeclareMathOperator{\irr}{irr}
\DeclareMathOperator{\Fr}{Fr}
\DeclareMathOperator{\Gal}{Gal}
\DeclareMathOperator{\lcm}{lcm}
\DeclareMathOperator{\alg}{\text{alg}}
\DeclareMathOperator{\Th}{Th}

\DeclareMathOperator{\DAG}{DAG}
\DeclareMathOperator{\ODAG}{ODAG}

% \varprod
\DeclareSymbolFont{largesymbolsA}{U}{txexa}{m}{n}
\DeclareMathSymbol{\varprod}{\mathop}{largesymbolsA}{16}
% \DeclareMathSymbol{\tonm}{\boldsymbol{\to}\textbf{Nm}}
\def \tonm {\bto\textbf{Nm}}
\def \tohm {\bto\textbf{Hm}}

% Category theory
\DeclareMathOperator{\Ab}{\textbf{Ab}}
\DeclareMathOperator{\Alg}{\textbf{Alg}}
\DeclareMathOperator{\Rng}{\textbf{Rng}}
\DeclareMathOperator{\Sets}{\textbf{Sets}}
\DeclareMathOperator{\Met}{\textbf{Met}}
\DeclareMathOperator{\BA}{\textbf{BA}}
\DeclareMathOperator{\Mon}{\textbf{Mon}}
\DeclareMathOperator{\Top}{\textbf{Top}}
\DeclareMathOperator{\Aut}{\textbf{Aut}}
\DeclareMathOperator{\RMod}{R-\textbf{Mod}}
\DeclareMathOperator{\RAlg}{R-\textbf{Alg}}
\DeclareMathOperator{\LF}{LF}
\DeclareMathOperator{\op}{op}
% Model theory
\DeclareMathOperator{\tp}{tp}
\DeclareMathOperator{\Diag}{Diag}
\DeclareMathOperator{\el}{el}
\DeclareMathOperator{\depth}{depth}
\DeclareMathOperator{\FO}{FO}
\DeclareMathOperator{\fin}{fin}
\DeclareMathOperator{\qr}{qr}
\DeclareMathOperator{\Mod}{Mod}
\DeclareMathOperator{\TC}{TC}
\DeclareMathOperator{\KH}{KH}
\DeclareMathOperator{\Part}{Part}
\DeclareMathOperator{\Infset}{\textsf{Infset}}
\DeclareMathOperator{\DLO}{\textsf{DLO}}
\DeclareMathOperator{\sfMod}{\textsf{Mod}}
\DeclareMathOperator{\AbG}{\textsf{AbG}}
\DeclareMathOperator{\sfACF}{\textsf{ACF}}
% Computability Theorem
\DeclareMathOperator{\Tot}{Tot}
\DeclareMathOperator{\graph}{graph}
\DeclareMathOperator{\Fin}{Fin}
\DeclareMathOperator{\Cof}{Cof}
\DeclareMathOperator{\lh}{lh}
% Commutative Algebra
\DeclareMathOperator{\ord}{ord}
\DeclareMathOperator{\Idem}{Idem}
\DeclareMathOperator{\zdiv}{z.div}
\DeclareMathOperator{\Frac}{Frac}
\DeclareMathOperator{\rad}{rad}
\DeclareMathOperator{\nil}{nil}
\DeclareMathOperator{\Ann}{Ann}
\DeclareMathOperator{\End}{End}
\DeclareMathOperator{\coim}{coim}
\DeclareMathOperator{\coker}{coker}
\DeclareMathOperator{\Bil}{Bil}
\DeclareMathOperator{\Tril}{Tril}
% Topology
\newcommand{\interior}[1]{%
  {\kern0pt#1}^{\mathrm{o}}%
}

% \makeatletter
% \newcommand{\vect}[1]{%
%   \vbox{\m@th \ialign {##\crcr
%   \vectfill\crcr\noalign{\kern-\p@ \nointerlineskip}
%   $\hfil\displaystyle{#1}\hfil$\crcr}}}
% \def\vectfill{%
%   $\m@th\smash-\mkern-7mu%
%   \cleaders\hbox{$\mkern-2mu\smash-\mkern-2mu$}\hfill
%   \mkern-7mu\raisebox{-3.81pt}[\p@][\p@]{$\mathord\mathchar"017E$}$}

% \newcommand{\amsvect}{%
%   \mathpalette {\overarrow@\vectfill@}}
% \def\vectfill@{\arrowfill@\relbar\relbar{\raisebox{-3.81pt}[\p@][\p@]{$\mathord\mathchar"017E$}}}

% \newcommand{\amsvectb}{%
% \newcommand{\vect}{%
%   \mathpalette {\overarrow@\vectfillb@}}
% \newcommand{\vecbar}{%
%   \scalebox{0.8}{$\relbar$}}
% \def\vectfillb@{\arrowfill@\vecbar\vecbar{\raisebox{-4.35pt}[\p@][\p@]{$\mathord\mathchar"017E$}}}
% \makeatother
% \bigtimes

\DeclareFontFamily{U}{mathx}{\hyphenchar\font45}
\DeclareFontShape{U}{mathx}{m}{n}{
      <5> <6> <7> <8> <9> <10>
      <10.95> <12> <14.4> <17.28> <20.74> <24.88>
      mathx10
      }{}
\DeclareSymbolFont{mathx}{U}{mathx}{m}{n}
\DeclareMathSymbol{\bigtimes}{1}{mathx}{"91}
% \odiv
\DeclareFontFamily{U}{matha}{\hyphenchar\font45}
\DeclareFontShape{U}{matha}{m}{n}{
      <5> <6> <7> <8> <9> <10> gen * matha
      <10.95> matha10 <12> <14.4> <17.28> <20.74> <24.88> matha12
      }{}
\DeclareSymbolFont{matha}{U}{matha}{m}{n}
\DeclareMathSymbol{\odiv}         {2}{matha}{"63}


\newcommand\subsetsim{\mathrel{%
  \ooalign{\raise0.2ex\hbox{\scalebox{0.9}{$\subset$}}\cr\hidewidth\raise-0.85ex\hbox{\scalebox{0.9}{$\sim$}}\hidewidth\cr}}}
\newcommand\simsubset{\mathrel{%
  \ooalign{\raise-0.2ex\hbox{\scalebox{0.9}{$\subset$}}\cr\hidewidth\raise0.75ex\hbox{\scalebox{0.9}{$\sim$}}\hidewidth\cr}}}

\newcommand\simsubsetsim{\mathrel{%
  \ooalign{\raise0ex\hbox{\scalebox{0.8}{$\subset$}}\cr\hidewidth\raise1ex\hbox{\scalebox{0.75}{$\sim$}}\hidewidth\cr\raise-0.95ex\hbox{\scalebox{0.8}{$\sim$}}\cr\hidewidth}}}
\newcommand{\stcomp}[1]{{#1}^{\mathsf{c}}}

\setlength{\baselineskip}{0.8in}

\stackMath
\newcommand\yrightarrow[2][]{\mathrel{%
  \setbox2=\hbox{\stackon{\scriptstyle#1}{\scriptstyle#2}}%
  \stackunder[0pt]{%
    \xrightarrow{\makebox[\dimexpr\wd2\relax]{$\scriptstyle#2$}}%
  }{%
   \scriptstyle#1\,%
  }%
}}
\newcommand\yleftarrow[2][]{\mathrel{%
  \setbox2=\hbox{\stackon{\scriptstyle#1}{\scriptstyle#2}}%
  \stackunder[0pt]{%
    \xleftarrow{\makebox[\dimexpr\wd2\relax]{$\scriptstyle#2$}}%
  }{%
   \scriptstyle#1\,%
  }%
}}
\newcommand\yRightarrow[2][]{\mathrel{%
  \setbox2=\hbox{\stackon{\scriptstyle#1}{\scriptstyle#2}}%
  \stackunder[0pt]{%
    \xRightarrow{\makebox[\dimexpr\wd2\relax]{$\scriptstyle#2$}}%
  }{%
   \scriptstyle#1\,%
  }%
}}
\newcommand\yLeftarrow[2][]{\mathrel{%
  \setbox2=\hbox{\stackon{\scriptstyle#1}{\scriptstyle#2}}%
  \stackunder[0pt]{%
    \xLeftarrow{\makebox[\dimexpr\wd2\relax]{$\scriptstyle#2$}}%
  }{%
   \scriptstyle#1\,%
  }%
}}

\newcommand\altxrightarrow[2][0pt]{\mathrel{\ensurestackMath{\stackengine%
  {\dimexpr#1-7.5pt}{\xrightarrow{\phantom{#2}}}{\scriptstyle\!#2\,}%
  {O}{c}{F}{F}{S}}}}
\newcommand\altxleftarrow[2][0pt]{\mathrel{\ensurestackMath{\stackengine%
  {\dimexpr#1-7.5pt}{\xleftarrow{\phantom{#2}}}{\scriptstyle\!#2\,}%
  {O}{c}{F}{F}{S}}}}

\newenvironment{bsm}{% % short for 'bracketed small matrix'
  \left[ \begin{smallmatrix} }{%
  \end{smallmatrix} \right]}

\newenvironment{psm}{% % short for ' small matrix'
  \left( \begin{smallmatrix} }{%
  \end{smallmatrix} \right)}

\newcommand{\bbar}[1]{\mkern 1.5mu\overline{\mkern-1.5mu#1\mkern-1.5mu}\mkern 1.5mu}

\newcommand{\bigzero}{\mbox{\normalfont\Large\bfseries 0}}
\newcommand{\rvline}{\hspace*{-\arraycolsep}\vline\hspace*{-\arraycolsep}}

\font\zallman=Zallman at 40pt
\font\elzevier=Elzevier at 40pt

\newcommand\isoto{\stackrel{\textstyle\sim}{\smash{\longrightarrow}\rule{0pt}{0.4ex}}}
\newcommand\embto{\stackrel{\textstyle\prec}{\smash{\longrightarrow}\rule{0pt}{0.4ex}}}

% from http://www.actual.world/resources/tex/doc/TikZ.pdf

\tikzset{
modal/.style={>=stealth’,shorten >=1pt,shorten <=1pt,auto,node distance=1.5cm,
semithick},
world/.style={circle,draw,minimum size=0.5cm,fill=gray!15},
point/.style={circle,draw,inner sep=0.5mm,fill=black},
reflexive above/.style={->,loop,looseness=7,in=120,out=60},
reflexive below/.style={->,loop,looseness=7,in=240,out=300},
reflexive left/.style={->,loop,looseness=7,in=150,out=210},
reflexive right/.style={->,loop,looseness=7,in=30,out=330}
}


\makeatletter
\newcommand*{\doublerightarrow}[2]{\mathrel{
  \settowidth{\@tempdima}{$\scriptstyle#1$}
  \settowidth{\@tempdimb}{$\scriptstyle#2$}
  \ifdim\@tempdimb>\@tempdima \@tempdima=\@tempdimb\fi
  \mathop{\vcenter{
    \offinterlineskip\ialign{\hbox to\dimexpr\@tempdima+1em{##}\cr
    \rightarrowfill\cr\noalign{\kern.5ex}
    \rightarrowfill\cr}}}\limits^{\!#1}_{\!#2}}}
\newcommand*{\triplerightarrow}[1]{\mathrel{
  \settowidth{\@tempdima}{$\scriptstyle#1$}
  \mathop{\vcenter{
    \offinterlineskip\ialign{\hbox to\dimexpr\@tempdima+1em{##}\cr
    \rightarrowfill\cr\noalign{\kern.5ex}
    \rightarrowfill\cr\noalign{\kern.5ex}
    \rightarrowfill\cr}}}\limits^{\!#1}}}
\makeatother

% $A\doublerightarrow{a}{bcdefgh}B$

% $A\triplerightarrow{d_0,d_1,d_2}B$


\def \diam {\text{diam}}
\def \Card {\text{Card}}
\author{Shichang Song}
\date{\today}
\title{Continuous First Order Logic}
\hypersetup{
 pdfauthor={Shichang Song},
 pdftitle={Continuous First Order Logic},
 pdfkeywords={},
 pdfsubject={},
 pdfcreator={Emacs 27.2 (Org mode 9.5)}, 
 pdflang={English}}
\begin{document}

\maketitle
\tableofcontents

\section{Day 1}
\label{sec:org6ddcaf9}
Continuou first order logic a.k.a. continuous logic, continuous model theory, model theory for
metric structures

\begin{center}
\begin{tabular}{ll}
Logic & Mathematics\\
1930s: Compactness thm & 1960s: diaphantine problems over local fields\\
ultraproducts, saturation & nonstandard analysis\\
1970s: shelah: classification theorem & 1996: mordell-lang conjecture\\
stability theory & \\
1980s: o-minimal theory & 2011: pila André-oort conjecture\\
\end{tabular}
\end{center}

background:

1960s: Applications of ultraproducts to Banach spaces. Krivine

1976: Ward Henson nonstandard hulls of Banach spaces: ``Henson's logic'' positive bounded formulas
with an approximate semantics

Later, 2002 with Iovino. Ultraproducts in analysis

2003, Ben Yaacov, compact abstract theories, Positive model theory and compact abstract theories

2005 model theory for metric structures

many valued logic: Łukasiewicz, Chang-Keisler.

\begin{center}
\begin{tabular}{ll}
Truth values & \([0,1]\)\\
quantifiers & \(\inf,\sup\)\\
equality & metric \(d(\cdot,\cdot)\)\\
\end{tabular}
\end{center}

Analogy betweeen CFO and FOL

Let \((M,d)\) be a complete, bounded metric space

\(d:M\times M\to\R^{\ge0}\) is a \textbf{metric} on \(M\)
\begin{enumerate}
\item \(d(x,y)\ge0\), \(d(x,y)=0\) iff \(x=y\)
\item \(d(x,y)=d(y,x)\)
\item \(d(x,y)\le d(x,z)+d(z,y)\)
\end{enumerate}


\((M,d)\) is \textbf{bounded} if \(\exists k>0,\forall x,y\in M\), \(d(x,y)<k\) or
\(\diam(M,d):=\sup_{x,y\in M}d(x,y)<k\)

\((M,d)\) is \textbf{complete} if every cauchy sequence convergencs




A \textbf{predicate} on \(M\) is a \textbf{uniformly continuous} function:
\(M^n\to[a,b]\subseteq\R\) for some \(n\ge1\)

\(P:M^n\to(0,1)\) is \textbf{uniformly continuous} if \(\forall \epsilon>0\exists\delta>0\forall x,y\in M^n\)
\(d(x,y)<\delta\to\abs{p(x)-p(y)}<\epsilon\)

A function on \(M\) is a uniformly continuous function: \(M^n\to M\) fro some \(n\ge1\).

For simplicity, \((M,d)\) is bounded by 1, predicates have values on \([0,1]\). 0 is true and 1 is
false.

\begin{remark}
In FOL, predicates: \(M^n\to\{0,1\}\)

In CFO, predicates: \(M^n\to[0,1]\)
\end{remark}

A \textbf{metric structure} \(\calm\) based on \((M,d)\) consists of a family \((R_i\mid i\in I)\) of
predicates on \(M\), a family \((F_j\mid j\in J)\) of functions on \(M\) and a
family \((a_k\mid k\in K)\) of distinguished elements of \(M\)

We denote a metric structure as
\begin{equation*}
  \calm=(M,R_i,F_j,a_k\mid i\in I,j\in J,k\in K)
\end{equation*}
\uline{Key restrictions}:
\begin{enumerate}
\item complete bounded
\item bounded interval (\(R_i\))
\item uniformly continuous \((R_i,F_j)\)
\end{enumerate}


\begin{examplle}[]
\begin{enumerate}
\item A complete bounded metric space \((M,d)\) with no additional structure
\item Given a first-order structure \(\calm\), define a discrete metric on it.
\begin{equation*}
d(a,b)=
\begin{cases}
1&\text{if }a\neq b\\
0
\end{cases}
\end{equation*}
predicates taking values in \(\{0,1\}\). Then \((\calm,d)\)  is a metric structure. Thus CFO is
a generalization of FOL
\item Probability algebras are boolean algebras of events in probabilities space.
\end{enumerate}
\end{examplle}

For each metric structure \(\calm=(M,R_i,F_j,a_k\mid i\in I,j\in J,k\in K)\), we
associate a \textbf{signature}
\begin{equation*}
 L=\{p_i,f_i,c_k\mid i\in I,j\in J,k\in K\}
\end{equation*}
where each \(p_i\) is a \textbf{predicate symbol} corresponding to predicate \(R_i\) on \(\calm\),
each \(f_j\) is a \textbf{function symbol} corresponding to predicate \(F_j\) on \(\calm\) ,
and each \(c_k\) is a \textbf{constant symbol} corresponding to predicate \(a_k\) on \(\calm\).

We associate to each symbol its arity like in FOL.

Moreover, in CFO, we associate to each predicate symbol \(p\) a closed bounded interval \(I_p\)
(usually \(I_p=[0,1]\)) and a \textbf{modulus of uniform continuity} \(\Delta_p\), i.e., a
function \(\Delta_p:(0,1]\to(0,1])\) satisfify \(\forall \epsilon>0\forall x,y\in M^n\),
if \(d(x,y)<\Delta_p(\epsilon)\) (defines a \(\delta\) by \(\Delta_p\)) then
\begin{equation*}
 \abs{p^{\calm}(x)-p^{\calm}(y)}<\epsilon
\end{equation*}
This \(\Delta_p\) does \textbf{not} depend on \(\calm\)

We associate to each function a modulus of uniformly continuity \(\Delta_f\), i.e. a
function \(\Delta_f:(0,1]\to(0,1]\) satisfying  \(\forall \epsilon>0\forall x,y\in M^n\),
if \(d(x,y)<\Delta_f(\epsilon)\) then
\begin{equation*}
 d(f^{\calm}(x,f^{\calm}(y)))<\epsilon
\end{equation*}

Finally, \(L\) provides \(D_L\) which is a bound on the diameter of \((M,d)\). For
simplicity, \(D_L=1\) and \(I_p=[0,1]\).

We say \(\calm\) an \(L\)-structure

Suppose \(\calm\) and \(\caln\) are \(L\)-structures. An \textbf{embedding} from \(\calm\) to \(\caln\) is
a metric space \textbf{isometry} \(T:(\calm,d^{\calm})\to(N,d^{\caln})\) s.t.
\begin{enumerate}
\item \(\forall x,y\in M\), \(d^{\caln}(T(x),T(y))=d^{\calm}(x,y)\)
\item for each \(n\)-ary predicate symbol \(p\) of \(L\)
\begin{equation*}
\forall a_1,\dots,a_n\in M,p^{\caln}(T(a_1),\dots,T(a_n))=p^{\calm}(a_1,\dots,a_n)
\end{equation*}
\item for each \(n\)-ary function symbol \(f\) of  \(L\)
\begin{equation*}
\forall a_1,\dots,a_n\in M,f^{\caln}(T(a_1),\dots,T(a_n))=T(f^{\calm}(a_1,\dots,a_n))
\end{equation*}
\item for each constant symbol \(c\) of \(L\)
\begin{equation*}
c^{\caln}=T(c^{\calm})
\end{equation*}
\end{enumerate}


An \textbf{isomorphism} is a surjective embedding. We say that \(\calm\) and \(\caln\) are isomorphic and
write \(\calm\cong\caln\) if there exists an isomorphism between \(\calm\) and \(\caln\)

If \(M\subseteq N\) and the inclusion map is an embedding of \(\calm\) into \(\caln\), then we
say \(\calm\) is a \textbf{substructure} of \(\caln\) and write \(\calm\subseteq\caln\)

More on modulus of uniform continuity. See [BBHU] appendix to section 2 on pages 322-327

Fix a signature \(L\) for metric structure

Symbols of \(L\)
\begin{itemize}
\item nonlogical symbols: predicates, functions, constants
\item logical symbols: \(d\)-binary predicate (=), \(V_L\) - an infinite set of
variables, \(u:[0,1]^n\to[0,1]\) continuous connectives (since continus \(\Rightarrow\)
uniformly continuous), \(\sup,\inf\) (quantifiers)
\end{itemize}


The \textbf{cardinality} of \(L\), denoted by \(\Card(L)\) is the smallest infinite cardinal
\(\ge\#\{\text{nonlogical symbols}\}\)

\textbf{terms} of \(L\)
\begin{enumerate}
\item variables and constant symbols
\item \(f(t_1,\dots,t_n)\)
\end{enumerate}


\textbf{Atomic formulas} of \(L\)
\begin{enumerate}
\item \(P(t_1,\dots,t_n)\) wher e\(P\) is an \(n\)-ary predicate symbol
\item \(d(t_1,t_2)\) like ``='' in FOL
\end{enumerate}


\textbf{Formulas} of \(L\)
\begin{enumerate}
\item atomic formulas
\item if \(u:[0,1]^n\to[0,1]\) is continuous and \(\varphi_1,\dots,\varphi_n\) are \(L\)-formulas,
then \(u(\varphi_1,\dots,\varphi_n)\) is an \(L\)-formula
\item if \(\varphi\) is an \(L\)-formula and \(x\) is a variable, then \(\sup_x\varphi\) and \(\inf_x\varphi\)
are \(L\)-formulas
\end{enumerate}


This definition is not a  good one
\begin{enumerate}
\item too general, uncountably continuous functions. we only need to concern a dense subset of it
\item too restrict, in order to develop a good nition of ``definability'', need formulas closed under
certain kinds of limits
\end{enumerate}


We write \(t(x_1,\dots,x_n)\) or \(\varphi(x_1,\dots,x_n)\) to indicate is free variables are
among \(x_1,\dots,x_n\)

\begin{examplle}[]
Let \(D_0\) denote the set of repeating decimals. Then \((D_0,d_0)\) (\(d_0\) is subtraction) is
a \textbf{pseudometric} space. Because \(d_0(0.\dot{9},1)=0\) but \(0.\dot{9}\neq1\). Consider its
quotient \((D,d)=(D_0,d_0)/\sim\) where \(x\sim y\) if \(d_0(x,y)=0\). Then \((D,d)\) is a metric
space, but it is not complete. Actually \((D,d)=(\Q,d)\). Take its completion, we
get \((\barD,\bard)=(\R,d)\).

pseudometric space -> metric space -> complete metric space
\end{examplle}

Fix a signature \(L\). Let \((M_0,d_0)\) be a pseudometric space
satisfying \(\diam(M_0,d_0)\le D_L\). An \textbf{\(L\)-prestructure} \(\calm_0\) based on \((M_0,d_0)\) is a
structure satisfying
\begin{enumerate}
\item for each predicate symbol \(p\) of \(L\) \(p^{\calm_0}:M_0^n\to I_p\) has \(\Delta_p\) as a
modulus of uniform continuity
\item for each function symbol \(f\) of \(L\), \(f^{\calm_0}:M_0^n\to M_0\) has \(\Delta_f\) as a modulus
of uniform continuity
\item for each constant symbol \(c\) of  \(L\), \(c^{\calm_0}\in M_0\)
\end{enumerate}


Given \(L\)-prestructure \(\calm_0\), we define its \textbf{quotient} prestructure as follows:

Let \((M,d)=(M_0,d_0)/\sim\), where \(x\sim y\) iff \(d_0(x,y)=0\). Let \(\pi:M_0\to M\) be the
quotient map. Then
\begin{enumerate}
\item for each predicate symbol \(p\) of \(L\), define \(p^{\calm}:M^n\to I_p\) by
  \begin{equation*}
p^{\calm}(\pi(x_1),\dots,\pi(x_n))=p^{\calm_0}(x_1,\dots,x_n)
  \end{equation*}
\item for each function symbol \(f\) of \(L\), define \(f^{\calm}:M^n\to M\) by
  \begin{equation*}
f^{\calm}(\pi(x_1),\dots,\pi(x_n))=\pi(f^{\calm_0}(x_1,\dots,x_n))
  \end{equation*}
\item for each constant symbol \(c\) of \(L\), define \(c^{\calm}=\pi(c^{\calm_0})\)
\end{enumerate}


Clearly,
\begin{enumerate}
\item \(\diam(M,d)=\diam(M_0,d)\)
\item \(p^{\calm}\) is well-defined and has \(\Delta_p\) as its modulus of uniform continuity
\item \(f^{\calm}\) is well-defined and has \(\Delta_f\) as its modulus of uniform continuity (these
2 proofs are in the appendix)
\end{enumerate}


Thus \((M,d)\) is an \(L\)-prestructure based on a possibly incomplete metric space.

Finally we take a \textbf{completion} of \(\calm\), denoted by \(L\)-structure \(\caln\)
\begin{enumerate}
\item for each predicate symbol \(p\). define \(p^{\caln}:N^n\to I_p\) as the unique extension
of \(p^{\calm}\) with the same \(\Delta_p\) (Check!)
\item for each \(f\), \(f^{\caln}:N^n\to N\) is the unique extension of \(f^{\calm}\) with the same \(\Delta_f\)
\item for each constant \(c\), \(c^{\caln}=c^{\calm}\)
\end{enumerate}


Let \(\calm\) be an \(L\)-prestructure and let \(A\subset M\). We extend \(L\) to a
signature \(L(A)\) by adding a new constant symbol \(c(a)\) to \(L\) for
each \(a\in A\). \((c(a))^{\calm=a}\). We call \(c(a)\) the \textbf{name} of \(a\) in \(L(A)\).
Consider \(L(M)\)-terms \(t(x_1,\dots,x_n)\), define exactly as in FOL
\begin{equation*}
 t^{\calm}:M^n\to M
\end{equation*}
The interpretation of \(t\) in \(\calm\)

\uline{Key definitions of semantics in CFO}
\begin{enumerate}
\item \((d(t_1,t_2))^{\calm}=d^{\calm}(t_1^{\calm},t_2^{\calm})\) for all \(t_1,t_2\)
\item \((p(t_1,\dots,t_n))^{\calm}=p^{\calm}(t_1^{\calm},\dots,t_n^{\calm})\) for all \(n\)-ary
predicate symbol \(p\) and all \(t_1,\dots,t_n\)
\item for all \(L(M)\)-sentences \(\sigma_1,\dots,\sigma_n\) and all continuous
function \(\mu:[0,1]^n\to[0,1]\)
  \begin{equation*}
(\mu[\sigma_1,\dots,\sigma_n])^{\calm}=\mu(\sigma_1^{\calm},\dots,\sigma_n^{\calm})
  \end{equation*}
\item for all \(L(M)\)-formulas \(\varphi(x)\)
  \begin{align*}
(\sup_x(\varphi(x)))^{\calm}=\sup_{a\in M}(\varphi(a))^{\calm}\in[0,1]
  \end{align*}
\end{enumerate}


Given \(L(M)\)-formula \(\varphi(x_1,\dots,x_n)\), we let \(\varphi^{\calm}\) denote the function
\(M^n\to[0,1]\) defined by
\begin{equation*}
 \varphi^{\calm}(a_1,\dots,a_n)=(\varphi(a_1,\dots,a_n))^{\calm}
\end{equation*}

Fact: \(\varphi^{\calm}\) is a uniformly continuous function

\begin{theorem}[]
\label{thmA.1}
Let \(t(x_1,\dots,x_m)\) be an \(L\)-term and \(\varphi(x_1,\dots,x_n)\) an \(L\)-formula. Then there
exists functions \(\Delta_t\) and \(\Delta_{\varphi} :(0,1]\to(0,1]\) s.t.
for every \(L\)-prestructure \(\calm\), \(\Delta_t\) is a modulus of uniform continuity for the
function \(t^{\calm}:M^n\to M\) and \(\Delta_\varphi\) is a modulus of uniform continuity for the
predicate \(\varphi^{\calm}:M^n\to[0,1]\)
\end{theorem}

\begin{proof}
Induction
\end{proof}

\begin{theorem}[]
\label{thmA.2}
pseudometric space \((M_0,d_0)\) \(\to\) quotient \((M,d)\) \(\to\) completion \((N,d)\)

Let \(t(x_1,\dots,x_n)\) be an \(L\)-term and \(\varphi(x_1,\dots,x_n)\) be an \(L\)-formula. Then
\begin{enumerate}
\item \(t^{\calm}(\pi(a_1),\dots,\pi(a_n))=t^{\calm_0}(a_1,\dots,a_n)\)
\item \(t^{\caln}(b_1,\dots,b_n)=t^{\calm}(b_1,\dots,b_n)\)
\item \(\varphi^{\calm}(\pi(a_1),\dots,\pi(a_n))=\varphi^{\calm_0}(a_1,\dots,a_n)\)
\item \(\varphi^{\caln}(b_1,\dots,b_n)=\varphi^{\calm}(b_1,\dots,b_n)\)
\end{enumerate}
\end{theorem}

\begin{proof}
in 3. key step is that \(\pi\) is surjective

in 4, key step is that \(\varphi^{\caln}\) is continuous and \(M\) is dense in \(N\)
\end{proof}


Two \(L\)-formulas \(\varphi(x_1,\dots,x_n)\) and \(\psi(x_1,\dots,x_n)\) are \textbf{logically equivalent} if
\begin{equation*}
 \varphi^{\calm}(a_1,\dots,a_n)=\psi^{\calm}(a_1,\dots,a_n)
\end{equation*}
for every \(L\)-structure \(\calm\)

The \textbf{logical distance} \(d_L\) between \(\varphi\) and \(\psi\) is
\begin{equation*}
 d_L(\varphi,\psi)=\sup_{\calm}\sup_{a_1,\dots,a_n\in M}\abs{\varphi^{\calm}(a_1,\dots,a_n)-\varphi^{\calm}(a_1,\dots,a_n)}
\end{equation*}


\begin{remark}
\begin{enumerate}
\item This defines a pseudometric
\item \(d_L(\varphi,\psi)=0\) iff \(\varphi\sim_L\psi\)
\end{enumerate}
\end{remark}

The space of \(L\)-formulas is too big. \textbf{density character} is the smallest dense subset w.r.t.
logical distance.

By Weierstrass theorem, there is a countable set of functions that is dense in the set of all
continuous functions w.r.t \(\sup\)-distance. We may use this countable set of functions to build
connectives. Then
\begin{enumerate}
\item the total number of constructed formulas is \(\Card(L)\)
\item every \(L\)-formulas can be approximated arbitrarily closely in logical distance by a formula
constructed using restricted connectives
\end{enumerate}
\section{Day 2}
\label{sec:org4a69f3a}
\begin{definition}[]
An \textbf{\(L\)-condition} \(E\) is of the form \(\varphi=0\), where \(\varphi\) is an \(L\)-formula. We call \(E\)
\textbf{closed} if \(\varphi\) is closed, i.e., \(\varphi\) is an \(L\)-sentence

If \(E\) is the \(L(M)\)-condition \(\varphi(x_1,\dots,x_n)=0\) and \(a_1,\dots,a_n\in M\), we say \(E\)
is \textbf{true of \(a_1,\dots,a_n\)} in \(\calm\) and we write \(\calm\vDash E[a_1,\dots,a_n]\) if
\(\varphi^{\calm}(a_1,\dots,a_n)=0\)

Let \(E_i\) be the \(L\)-condition \(\varphi_i(x_1,\dots,x_n)=0\). We say \(E_1\) and \(E_2\) are
\textbf{logically equivalent} if for every \(L\)-structure \(\calm\) and every \(a_1,\dots,a_n\) we have
\begin{equation*}
  \calm\vDash E_1[a_1,\dots,a_n] \quad\text{ iff }\quad
  \calm\vDash E_2[a_1,\dots,a_n]
\end{equation*}

\(\varphi=\psi\) is an abbreviation for the condition \(\abs{\varphi-\psi}=0\), where
\(\abs{\cdot}:[0,1]^2\to[0,1]\), \((t_1,t_2)\mapsto\abs{t_1-t-2}\) is a connective.

\(\varphi\le\psi\) iff \(\varphi\dot-\psi=0\)
\end{definition}

In \([0,1]\)-valued logic, \(\varphi\le\psi\) is like \(\varphi\to\psi\) in FOL. Since
from \(\psi\le r\) we have \(\varphi\le r\) for all \(r\in[0,1]\)

Fix a signature \(L\) for metric structure.

\begin{definition}[]
A \textbf{theory} \(T\) is a set of closed \(L\)-conditions. We say \(\calm\) is a model of \(T\) and
write \(\calm\vDash T\)  if \(\calm\vDash E\) for every condition \(E\in T\).

Let \(\Mod_L(T)\) be the collection of all models of \(T\)

The \textbf{theory of \(\calm\)}, denoted by \(\Th(\calm)\), is the set of closed \(L\)-conditions that are
true in \(\calm\).

If \(T\) is a theory of this form, then \(T\) is \textbf{complete}.

We say \(E\) is a \textbf{logical consequence} of \(T\) and write \(T\vDash E\) if \(\calm\vDash E\) holds
for every model \(\calm\) of \(T\).
\end{definition}

\begin{remark}
\begin{enumerate}
\item models are complete metric spaces.
\item Let \(\calm_0\) be an \(L\)-prestructure s.t. \(\varphi^{\calm_0}=0\) for every
condition \(\varphi=0\) in \(T\). Then by Theorem \ref{thmA.2}, the completion of the canonical
quotient of \(\calm_0\) is a model of \(T\). (\(\calm_0\) is a \textbf{premodel})
\end{enumerate}
\end{remark}

\begin{definition}[]
\begin{enumerate}
\item We say \(\calm\) and \(\caln\) are \textbf{elementary equivalent} and write \(\calm\equiv\caln\)
if \(\sigma^{\calm}=\sigma^{\caln}\) for all \(L\)-sentences \(\sigma\)
\item If \(\calm\subseteq\caln\), we say that \(\calm\) is an \textbf{elementary substructure} of \(\caln\)
and write \(\calm\preceq\caln\)
if whenever \(\varphi(x_1,\dots,x_n)\) is an \(L\)-formula and \(a_1,\dots,a_n\in M\) we have
 \begin{equation*}
\varphi^{\calm}(a_1,\dots,a_n)=\varphi^{\caln}(a_1,\dots,a_n)
 \end{equation*}
We also say \(\caln\) is an \textbf{elementary extension} of \(\calm\)
\item \(F:A\subseteq M\to N\) is an \textbf{elementary map} if whenever \(\varphi(x_1,\dots,x_n)\) is
an \(L\)-formula and \(a_1,\dots,a_n\in\dom(F)\) we have
 \begin{equation*}
\varphi^{\calm}(a_1,\dots,a_n)=\varphi^{\caln}(F(a_1),\dots,F(a_n))
 \end{equation*}
\item An \textbf{elementary embedding} of \(\calm\) into \(\caln\) is a function \(M\to N\) that is an
elementary map from \(\calm\) into \(\caln\)
\end{enumerate}
\end{definition}

\begin{remark}
\begin{enumerate}
\item elementary map is distance preserving, and thus is an embedding
\item \(\calm\cong\caln\Rightarrow\calm\equiv\caln\)
\end{enumerate}
\end{remark}

We say \(S\) of \(L\)-formulas is \textbf{dense w.r.t. logical distance} if for
every \(L\)-formula \(\varphi(x_1,\dots,x_n)\) and every \(\epsilon>0\) there is \(\psi(x_1,\dots,x_n)\) in \(S\)
s.t. for every \(L\)-structure \(\calm\) and all \(a_1,\dots,a_n\in M\) we have
\begin{equation*}
  \abs{\varphi^{\calm}(a_1,\dots,a_n)-\psi^{\calm}(a_1,\dots,a_n)}\le\epsilon
\end{equation*}

\begin{proposition}[Tarski-Vaught Test for $\preceq$]
\label{propB.1}
Let \(S\) be dense w.r.t. logical distance. Suppose \(\calm\) and \(\caln\) are \(L\)-structures
with \(\calm\subseteq\caln\). Then the following are equivalent
\begin{enumerate}
\item \(\calm\preceq\caln\)
\item For every \(L\)-formula e\(\varphi(x_1,\dots,x_n,y)\) in \(S\) and all \(a\in M^n\)
\begin{equation}
\inf\{\varphi^{\caln}(a,b)\mid b\in N\}=\inf\{\varphi^{\caln}(a,c)\mid c\in M\}\tag{$\star$}
\end{equation}
\end{enumerate}
\end{proposition}

\begin{proof}
\(1\to 2\). By 1, we have
\begin{align*}
  \inf\{\varphi^{\caln}(a_1,\dots,a_n,b)\mid b\in N\}&=
  \left(\inf_y\varphi(a_1,\dots,a_n,y)\right)^{\caln}\\
  &=\left((\inf_y\varphi(a_1,\dots,a_n,y))\right)^{\calm}\\
  &=\inf\{\varphi^{\calm}(a_1,\dots,a_n,c)\mid c\in M\}\\
  &=\inf\{\varphi^{\caln}(a_1,\dots,a_n,c)\mid c\in M\}
\end{align*}

\(2\to1\). First we show \(\star\) holds for all \(L\)-formulas \(\varphi(x_1,\dots,x_n,y)\).
\(\forall\epsilon>0\), take \(\varphi(x_1,\dots,x_n,y)\in S\) s.t.
\begin{equation*}
  \sup_{\calm}\sup_{a_1,\dots,a_n\in M}
  \abs{\varphi^{\calm}(a_1,\dots,a_n,b)-\psi^{\calm}(a_1,\dots,a_n,b)}\le\epsilon
\end{equation*}
Let \(a_1,\dots,a_n\in M\) then we have
\begin{align*}
  \inf\{\varphi^{\caln}(a_1,\dots,a_n,b)\mid b\in M\}&\le
  \inf\{\psi^{\caln}(a_1,\dots,a_n,b)\mid b\in M\}+\epsilon\\
  &=\inf\{\psi^{\caln}(a_1,\dots,a_n,c)\mid c\in N\}+\epsilon\\
  &\le\inf\{\varphi^{\caln}(a_1,\dots,a_n,c)\mid c\in N\}+2\epsilon
\end{align*}
Let \(\epsilon\to0\), then
\begin{equation*}
  \inf\{\varphi^{\caln}(a_1,\dots,a_n,b)\mid b\in M\}\le
  \inf\{\varphi^{\caln}(a_1,\dots,a_n,c)\mid c\in N\}
\end{equation*}
Hence \(\star\) holds for all \(L\)-formulas \(\varphi\).

Then by incution on the complexities of \(\varphi\) and \(\star\) we have
\(\varphi^{\calm}(a_1,\dots,a_n)=\varphi^{\caln}(a_1,\dots,a_n)\) for all \(a_1,\dots,a_n\in M\)
\end{proof}

\begin{definition}[]
Let \(I\) be a nonempty set. A \textbf{filter} on \(I\) is a collection \(F\) of subsets of \(I\) satisfies
\begin{enumerate}
\item \(\emptyset\not\in F\) and \(I\in F\)
\item for all \(A,B\in F\), \(A\cap B\in F\)
\item for all \(A\in F\), if \(A\subseteq B\subseteq I\) then \(B\in F\)
\end{enumerate}


A filter \(F\) is an \textbf{ultrafilter} if it is maximal under \(\subseteq\) among filters on \(I\)

\(F\) is \textbf{principal} if there is a subset \(A\subseteq I\) s.t. \(F\) is exactly the collection of
all sets \(B\) that satisfy \(A\subseteq B\subseteq I\).

non-principal is also called as \textbf{free}
\end{definition}

\begin{definition}[]
\(S\) is a collection of \(I\). We say that \(S\) has \textbf{finite intersection property} (FIP)
if \(\forall n\in\N\), \(\forall\) finite subset collection \(\{A_1,\dots,A_n\}\)
of \(S\), \(A_1\cap\dots\cap A_n\neq\emptyset\)
\end{definition}

\begin{lemma}[]
\label{lemmaB.2}
Let \(I\) be a nonempty set and let \(S\) be a collection of subsets of \(I\). There exists a
filter \(F\) on \(I\) which contains \(S\) iff \(S\) has the FIP
\end{lemma}

\begin{remark}
The smallest filter on \(I\) containing \(S\) is called the \textbf{filter generated by \(S\)}
\end{remark}

\begin{lemma}[]
\label{lemmaB.3}
Let \(F\) be a filter on a nonempty set \(I\). Then \(F\) is an ultrafilter iff \(\forall A\subset I\),
either \(A\in F\) or \(A^c\in F\).
\end{lemma}

\begin{remark}
principal ultrafilters are trivial
\end{remark}

\begin{theorem}[]
\label{thmB.4}
Let \(I\) be a nonempty set. Then every filter on \(I\) is contained in an ultrafilter on \(I\).
\end{theorem}

\begin{proof}
Zorn's lemma
\end{proof}

\begin{corollary}[]
\label{corB.5}
Let \(I\) be a nonempty set and let \(S\) be a collection of subset of \(I\). If \(S\) has the
FIP, then there is an ultrafilter on \(I\) that contains \(S\).
\end{corollary}

Fix a first order signature \(L\). Let \(I\) be a nonempty set and let \(U\) be a fixed
ultrafilter on \(I\). Consider an \(I\)-indexed family of \(L\)-structures \(\cala_i\).
Let \(A=\prod_{i\in I}A_i\) be the Cartesian product of the sets \(A_i\). Let \(f,g\in A\). We
define a relation on \(A\)
\begin{equation*}
  f\sim g \quad\text{ iff }\quad
  \{i\mid f(i)=g(i)\}\in U
\end{equation*}
\begin{lemma}[]
\label{lemmaB.6}
The relation \(\sim\) is an equivalence relation on \(A\)
\end{lemma}

Then \(A/\sim\) is the ultraproduct of the set \(A_i\) w.r.t. the ultrafilter \(U\) on \(I\)

We let \(\prod_UA_i\) denote \(A/\sim\), the collection of all equivalence classes
\(\{[f]\mid f\in\prod_{i\in I}A_i\}\)

\begin{definition}[]
The \textbf{ultraproduct} \(\prod_{U}\cala_i\) is defined to be the \(L\)-structure
\begin{enumerate}
\item the universe of \(\prod_{U}\cala_i\) is \(\prod_UA_i\)
\item for each constant \(c\) in \(L\), define \(f\in A\) by \(f(i)=c^{\cala_i}\)
 \begin{equation*}
c^{\prod_U\cala_i}=f/\sim
 \end{equation*}
\item for each predicate \(P\) in \(L\)
 \begin{equation*}
P^{\prod_U\cala_i}(f_1/\sim,\dots,f_n/\sim)\quad\text{ iff }\quad
\{i\in I\mid P^{\cala_i}(f_1(i),\dots,f_n(i))\}\in U
 \end{equation*}
\item for each function \(F\) in \(L\)
 \begin{equation*}
F^{\prod_U\cala_i}(f_1/\sim,\dots,f_n/\sim)=f/\sim
 \end{equation*}
where \(f\in A\) is defined by \(f(i)=F^{\cala_i}(f_1(i),\dots,f_n(i))\)
\end{enumerate}


Need to check they are well-defined

An \textbf{ultrapower} of \(\cala\) is an ultraproduct \(\prod_U\cala_i\) with \(\cala_i=\cala\) for
all \(i\in I\)
\end{definition}

\begin{theorem}[Łoś's theorem (fundamental theorem for ultraproducts)]
\label{thmB.7}
For every \(L\)-formula \(\varphi(x_1,\dots,x_n)\) and every \(f/\sim=(f_1/\sim,\dots,f_n/\sim)\), we
have
\begin{equation*}
  \prod_U\cala_i\vDash\varphi[f/\sim] \quad\text{ iff }\quad
  \{i\in I\mid\cala_i\vDash\varphi[f_1(i),\dots,f_n(i)]\}\in U
\end{equation*}
\end{theorem}

\begin{corollary}[]
\label{corB.8}
if \(\sigma\) is an \(L\)-sentence, then \(\prod_U\cala_i\vDash\sigma\) iff
\(\{i\in I\mid\cala_i\vDash\sigma\}\in U\)
\end{corollary}

Let \(X\) be a topological space and let \((x_i)_{i\in I}\) be a family of elements of \(X\).
If \(D\) is an ultrafilter on \(I\) and \(x\in X\), we write \(\lim_{i,D}x_i=x\) (ultra limit) and
say \(x\) is the \textbf{\(D\)-limit of \((x_i)_{i\in I}\)} if \(\forall\)
open \(U\ni x\), \(\{i\in I\mid x_i\in U\}\in D\)

\textbf{Fact}: \(X\) is a compact Hausdorff space (e.g. \(X=[0,1]\)) iff for every family \((x_i)_{i\in I}\) in \(X\) and
 every ultrafilter \(D\) on \(I\) the \(D\)-limit of \((x_i)_{i\in I}\) exists and is unique.

\begin{lemma}[]
\label{lemmaB.9}
Suppose \(X,X'\) are topological spaces and \(F:X\to X'\) is continuous. For every
family \((x_i)_{i\in I}\) from \(X\) and every ultrafilter \(D\) on \(I\), we have
\begin{equation*}
 \lim_{i, D}x_i=x\Rightarrow\lim_{i,D}F(x_i)=F(x)
\end{equation*}
where the ultralimits are taken in \(X\) and \(X'\) respectively
\end{lemma}

\begin{proof}
Take open \(U\ni F(x)\) in \(X'\). Since \(F\) is continuous, \(F^{-1}(U)\) is open in \(X\).
And \(F^{-1}(U)\ni x\). If \(x\) is the \(D\)-limit of \((x_i)_{i\in I}\) there is \(A\in D\)
s.t. \(x_i\in F^{-1}(U)\) for all \(i\in A\) and thus \(F(x_i)\in U\)
\end{proof}

\begin{definition}[]
Let \(((M_i,d_i)\mid i\in I)\) be a family of bounded metric spaces with diameter \(\le k\).
Let \(D\) be an ultrafilter on \(I\). Define \(d\) on \(\prod_{i\in I}M_i\)
by \(d(x,y)=\lim_{i,D}d_i(x_i,y_i)\), when \(x=(x_i)_{i\in I}\) and \(y=(y_i)_{i\in I}\).

Check: \(d\) is a pseudometric on \(\prod_{i\in I}M_i\)

For \(x,y\in\prod_{i\in I}M_i\), define \(x\sim_Dy\) iff \(d(x,y)=0\)

Then \(\sim_D\) is an equivalence relation, so we may define
\begin{equation*}
 \left(\prod_{i\in I}M_i\right)_D=\left(\prod_{i\in I}M_i\right)/\sim_D
\end{equation*}
Later we will see its complete

The pseudometric \(d\) on \(\prod_{i\in I}M_i\) induces a metric \(d\)
on \((\prod_{i\in I}M_i)_D\)

The space \(((\prod_{i\in I}M_i)_D,d)\) is the \textbf{\(D\)-ultraproduct} of \(((M_i,d_i)\mid i\in I)\).

We denote \((x_i)_{i\in I}/\sim_D\) by \(((x_i)_{i\in I})_D\)

If \((M_i,d_i)=(M,d)\) \(\forall i\in I\). The space \((\prod_{i\in I}M_i)_D\) is called the
\textbf{\(D\)-ultrapower} of \(M\) and denoted by \((M)_D\)

\(T:M\to(M)_D\), \(x\mapsto((x_i)_{i\in I})_D\), where \(\forall i\in I\), \(x_i=x\) is a \textbf{diagonal
embedding}. its an isometric embedding
\end{definition}

If \((M,d)\) is compact, then it is easy to show \(((x_i)_{i\in I})_D=T(x)\), i.e., the diagonal
embedding is surjective.

\textbf{Fact}: every ultrapower of a closed bounded interval may be canonically identified wiht the
 interval itself. e.g. \(([0,1])_D=[0,1]\)

\begin{proposition}[]
\label{propB.10}
Let \(((M_i,d_i)\mid i\in I)\) be a family of complete, uniformly bounded metric space.
Let \(D\) be an ultrafilter on \(I\) and let \((M,d)\) be the \(D\)-ultraproduct
of \(((M_i,d_i)\mid i\in I)\). The metric space \((M,d)\) is complete
\end{proposition}

\begin{proof}
let \((x^k)_{k\ge1}\)  be a Cauchy sequence in \((M,d)\). WLOG, we assume that
\(d(x^k,x^{k+1})<\frac{1}{2^k}\) holds for all \(k\ge1\). We want to show it has a limit.

For each \(k\ge1\) let \(x^k\) be represente by the family \((x_i^k)_{i\in I}\). For
each \(m\ge1\), let \(A_m=\{i\in I\mid d_i(x_i^k,x_i^{k+1})<\frac{1}{2k}\forall k\le m\}\).
Note \(A_1\supseteq A_2\supseteq\dots A_n\neq\emptyset\supseteq\dots\) and each \(A_i\in D\).

We define a family \((y_i)_{i\in I}\) as follows. If \(i\not\in A_1\) then we take \(y_i\)
arbitrarily. If \(i\in A_m\setminus A_{m+1}\) for some \(m\ge1\), then we set \(y_i=x_i^{m+1}\).
If \(\forall m\ge1\), \(i\in A_m\), then \((x_i^m)_{m\ge1}\) is a Cauchy sequence, we take \(y_i\) to
be its limit.

Then for each \(m\ge1\) each \(i\in A_m\) we have \(d_i(x_i^m,y_i)\le 2^{-m+1}\). It follows that
\(((y_i)_{i\in I})_{D}\) is the limit of \((x^k)_{k\ge1}\) in \((M,d)\)
\end{proof}
\end{document}
